\documentclass[12pt]{cmuthesis}

%for a more compact document, add the option openany to avoid
%starting all chapters on odd numbered pages

\usepackage{comment}
% \usepackage[ruled,vlined]{algorithm2e}
\usepackage[boxed,ruled,vlined,figure]{algorithm2e}
\usepackage{url}
\usepackage{ifthen}
\usepackage{framed}
\usepackage{enumitem}
\usepackage{multirow}
\usepackage{subcaption}
\usepackage{times}
\usepackage{amssymb}
\usepackage[export]{adjustbox}
\usepackage{rotating}
\usepackage{times}
\usepackage{fullpage}
\usepackage{amsmath}
\usepackage[numbers,sort]{natbib}
\usepackage[backref,pageanchor=true,plainpages=false, pdfpagelabels, bookmarks,bookmarksnumbered]{hyperref}
%pdfborder=0 0 0,  %removes outlines around hyper links in online display
% \usepackage{subfigure}
\usepackage{color}
\usepackage{soul}
\usepackage{xspace}
\usepackage{makecell}
\usepackage{booktabs}
\usepackage{graphicx}
\usepackage{epsfig}
\usepackage{siunitx}

%% paramgraph

\setlength{\parskip}{6pt}

%%%
%%%  Comments
%%%
\newcommand{\showComments}{yes}
\newcommand{\note}[2]{
	\ifthenelse{\equal{\showComments}{yes}}{\textcolor{#1}{#2}}{}
}

\newcommand{\showRevision}{no}
\newcommand{\revision}[1]{
	\ifthenelse{\equal{\showRevision}{yes}}{\textcolor{red}{#1}}{#1}
}


\newcommand{\TODO}[1]{\note{red}{#1}}


% FOR SUBMISSION ONLY: Eliminate copyright box at bottom of first page
% Use \toappear{} if you want a blank ACM copyright box instead of suppressing it
\usepackage{etoolbox}
\makeatletter
\patchcmd{\maketitle}{\@copyrightspace}{}{}{}
\makeatother


% Itemize and Enumerate with less white space wasted
\newenvironment{smitemize}%
  {\begin{list}{$\bullet$}%
     {\setlength{\parsep}{1pt}%
      \setlength{\topsep}{1pt}%
      \setlength{\itemsep}{1pt}}}%
  {\end{list}}

\newenvironment{smenumerate}{
\begin{enumerate}
  \setlength{\itemsep}{2.5pt}
  \setlength{\parskip}{0pt}
  \setlength{\parsep}{0pt}
}{\end{enumerate}}

\newenvironment{mypara}{
   \vspace{0.1in}
   \textbf}
{\plain}


\newcommand{\lc}[1]{\lowercase{#1}}
\newcommand{\uc}[1]{\uppercase{#1}}
\newcommand{\xc}[1]{\small\sc #1}


\newenvironment{captiontext}{%
   \par\vspace{0.1in}\renewcommand{\baselinestretch}{0.9}\footnotesize}%
   {\renewcommand{\baselinestretch}{1.0}\par}


%
% defining the \BibTeX command - from Oren Patashnik's original BibTeX documentation.
\def\BibTeX{{\rm B\kern-.05em{\sc i\kern-.025em b}\kern-.08emT\kern-.1667em\lower.7ex\hbox{E}\kern-.125emX}}

\usepackage{setspace} % for \onehalfspacing and \singlespacing macros
\usepackage{etoolbox}
\AtBeginEnvironment{quote}{\singlespacing\small}

% \settopmatter{printacmref=false} % Removes citation information below abstract
% \renewcommand\footnotetextcopyrightpermission[1]{} % removes footnote with conference information in first column
\pagestyle{plain}


% Abbreviations


% Approximately 1" margins, more space on binding side
%\usepackage[letterpaper,twoside,vscale=.8,hscale=.75,nomarginpar]{geometry}
%for general printing (not binding)
\usepackage[letterpaper,twoside,vscale=.8,hscale=.75,nomarginpar,hmarginratio=1:1]{geometry}

% Provides a draft mark at the top of the document. 
\draftstamp{\today}{DRAFT}


\begin{document}

\frontmatter

\title{
    {\bf Scaling Wearable Cognitive Assistance}
}
\author{Junjue Wang}
\date{May 2020}
\Year{2020}
\trnumber{CMU-CS-20-107}

\vspace{3cm}

\committee{
    Mahadev Satyanarayanan (Satya) (Chair), Carnegie Mellon University\\
    Daniel Siewiorek, Carnegie Mellon University \\
    Martial Hebert, Carnegie Mellon University \\
    Roberta Klatzky, Carnegie Mellon University \\
    Padmanabhan Pillai, Intel Labs
}

\support{This research was supported by the National Science Foundation (NSF)
    under grant number CNS-1518865. Additional support was provided by Intel,
    Vodafone, Deutsche Telekom, Verizon, Crown Castle, Seagate, VMware,
    MobiledgeX, InterDigital, and the Conklin Kistler family fund.}
% \disclaimer{}

% copyright notice generated automatically from Year and author.
% permission added if \permission{} given.

\keywords{Wearable Cognitive Assistance, Scalability, Adaptation, Wearable Computing, Augmented Reality, Edge Computing, Cloudlet}

\maketitle

\begin{dedication}
    To the future of machines.
\end{dedication}

%% Obviously, it's probably a good idea to break the various sections of your thesis
%% into different files and input them into this file...

\begin{abstract}
    It has been a long endeavour to augment human cognition with machine
    intelligence. Recently, a new genre of applications, named Wearable
    Cognitive Assistance, has advanced the boundaries of augmented cognition.
    These applications continuously process data from body-worn sensors and
    provide just-in-time guidance to help a user complete a specific task. While
    previous research has demonstrated the technical feasibility of wearable
    cognitive assistants, this dissertation addresses the problem of
    \textit{scalability}. We identify two critical challenges to the widespread
    deployment of these applications to be 1) the need to operate cloudlets and
    wireless network at low utilization to achieve acceptable end-to-end latency
    2) the level of specialized skills and the long development time needed to
    create new applications. To address these challenges, we first design and
    evaluate adaptation-centric optimizations that reduce resource consumption
    and improve resource management in contentious systems while maintaining
    acceptable end-to-end latency. We then propose and implement a new
    prototyping methodology and a suite of development tools to lower the
    barrier of application development.
\end{abstract}

\begin{acknowledgments}
    I have been extremely fortunate to be inspired and supported by a group of
    talented and caring individuals throughout my PhD research. This
    dissertation would not be possible without them.

    I am deeply indebted to my advisor Mahadev Satyanarayanan (Satya), who
    played a paramount role in guiding me through this dissertation research. Satya
    taught me how to design, implement, and evaluate mobile and distributed systems.
    His knowledge and encouragement have been a constant source of inspiration for
    me to work harder and produce more impactful system research. In addition to an
    excellent academic advisor, Satya has also been a visionary mentor whom I look
    up to. It has been invaluable for me to observe and learn from his leadership in
    shaping new technologies and bringing people together for a shared vision.
    Another person who has inspired and helped me tremendously is Padmanabhan (Babu)
    Pillai. He is an exemplary researcher with broad and deep knowledge, kindness,
    and patience. In addition, I would like to thank my thesis committee members,
    Daniel Siewiorek, Martial Hebert, and Roberta (Bobby) Klatzky. As this
    dissertation is at the intersection of multiple fields, they have given me many
    insightful feedback and guidance on human-computer interaction and computer
    vision. Their comments from our weekly meetings have been vital for me to stay
    on the correct course.

    Brilliant and kind individuals in Satya's research group have
    provided many rapport and joy during my PhD study. In addition, many ideas
    and experimental results presented in this dissertation come from close
    collaborative work with them. They have also provided many system
    infrastructure support to perform the experiments. I cannot dream of a
    better team and environment to be in for a PhD student. My day-to-day
    interactions with them helped me grow as a better researcher and a critical
    thinker. Especially, I would like to give my appreciation to, in an
    alphabetical order, Yoshihisa Abe, Brandon Amos, Jim Blakley, Zhuo Chen, Tom
    Eiszler, Ziqiang (Edmond) Feng, Shilpa George, Kiryong Ha, Jan Harkes, Wenlu
    Hu, Roger Iyengar, Natalie Janosik, and Haithem Turki. I also want to thank
    Chase Klingensmith for his outstanding administrative assistance for the
    group.

    My PhD research has also benefited significantly from the larger research
    community. I was fortunate enough to collaborate with many exceptional
    academic researchers and industry professionals in the past few years. Thank
    you all for the inspiration you have given me. In particular, thank you,
    Pauline Anthonysamy, Mihir Bala, Richard Buchler, Kevin Christensen, Anupam
    Das, Nigel Davies, Debidatta Dwibedi, Khalid Elgazzar, Wei Gao, Ying Gao,
    James Gross, Alex Guerrero, Guenter Klas, Zico Kolter, Michael Kozuch,
    Hongkun Leng, Grace Lewis, Tan Li, Haodong Liu, Yuqi Liu, Lin Ma, Mateusz
    Mikusz, Manuel Olguín, Sai Teja Peddinti, Truong An Pham, Norman Sadeh, Rolf
    Schuster, Asim Smailagic, Nihar B. Shah, Nina Taft, Joseph Wang, Guanhang
    Wu, Yu Xiao, Shao-Wen Yang, Canbo (Albert) Ye, and Siyan Zhao.

    My beloved family and friends have provided enormous understanding,
    sympathy, and support throughout my PhD study. I could not have finished
    this thesis research without knowing they are on my back. In particular, I
    have always felt extremely lucky and gracious to be born to the most
    attentive, considerate, and wise parents in the world. They instilled the
    value of perseverance and compassion early in my life through examples.
    Without their trust and encouragement, I would never dream of completing a
    doctorate degree halfway around the globe at one of the finest computer
    science institutions in the world. Furthermore, as lucky as a man could be,
    I met my dearest partner, Han, while pursuing my PhD. Since then, she has
    been my closest friend and most cherished counselor who has provided an
    immeasurable amount of love, strength, and comfort. I am able to become a
    better version of myself because of her. As much as this dissertation is
    written for the future of machines, it is the unwritten words about people
    that will be treasured for the rest of my life.

\end{acknowledgments}

\tableofcontents
\listoffigures
\listoftables
\mainmatter

%% Double space document for easy review:
%\renewcommand{\baselinestretch}{1.66}\normalsize

% The other requirements Catherine has:
%
%  - avoid large margins.  She wants the thesis to use fewer pages, 
%    especially if it requires colour printing.
%
%  - The thesis should be formatted for double-sided printing.  This
%    means that all chapters, acknowledgements, table of contents, etc.
%    should start on odd numbered (right facing) pages.
%
%  - You need to use the department standard tech report title page.  I
%    have tried to ensure that the title page here conforms to this
%    standard.
%
%  - Use a nice serif font, such as Times Roman.  Sans serif looks bad.
%
% Other than that, just make it look good...

\chapter{Introduction}

It has been a long endeavour to augment human cognition with machine
intelligence. As early as in 1945, Vannevar Bush envisioned a machine Memex that
provides "enlarged intimate supplement to one's memory" and can be "consulted
with exceeding speed and flexibility" in the seminal article \textit{As We May
Think}~\cite{bush1945we}. This vision has been brought closer to reality by years
of research in computing hardware, artificial intelligence, and human-computer
interaction. In late 90s to early 2000s, Smailagic et al
~\cite{smailagic1993case}~\cite{smailagic1998very}~\cite{smailagic2002application}
created prototypes of wearable computers to assist several cognitive tasks, for
example, displaying inspection manuals in a head-up screen to facilitate
aircraft maintenance. Around the same time, Loomis el
al~\cite{loomis1998navigation}~\cite{loomis1994personal} explored using
computers carried in a backpack to help the blind navigate through auditory
cues. Davis et al~\cite{davies1998developing} developed a context-sensitive
intelligent visitor guide leveraging hand-portable multimedia systems. While
these research work pioneered human cognition assistance, they are limited by
the technologies of their time.

More recently, as many underlying technologies experience fundamental changes,
new genres of several applications have been 

advancement in machine learning, especially in machine
learning, computer hardware, and distributed computing. Many applications that
have been built for

(Challenges of making them happen)

Despite the ventures of these research work, we are still far from the reality
that human augmentation is prevalent in society. A mature cognitive system needs
three key components. (algorithm, compute, networking)

(Recent wonderful advancement has made the dream closer)
In recent five years, significant improvements have made WCA feasible. (DNN,
edge computing, and wearable h/w).

(At the intersection of all three, WCA has been shown feasiblity emerged)
Wearable Cognitive Assistance has emerged as a new genre of applications that
pushes the boundaries of augmented cognition. These applications continuously
process data from body-worn sensors and provide just-in-time guidance to help a
user complete a specific task. For example, an IKEA Lamp
assistant~\cite{chen2018application} has been built to assist the assembly of a
table lamp. To use the application, a user wears a head-mounted smart glass that
continuously captures her actions and surroundings from a first-person
viewpoint. In real-time, the camera stream is analyzed to identify the state of
the assembly. Audiovisual instructions are generated based on the detected
state. The instructions either demonstrate a subsequent procedure or alert and
correct a mistake.

(WCA has its root in edge computing)
Edge system

(Problems that have not been addressed)

(scalability at the edge)

(scalability for development)
Fundamental changes in development. uncertainty. Fundamental changes in how
people deploy applications.

Although Wearable Cognitive Assistance shares the vision of cognition
enhancement with many previous research
efforts~\cite{kidd1999aware}~\cite{loomis1998navigation}
~\cite{cheverst2000developing}~\cite{tanuwidjaja2014chroma},
its design goals advance the frontier of mobile computing in multiple aspects.
First, wearable devices, particularly head-mounted smart glasses, are used to
reduce the discomfort caused by carrying a bulky computation device. Users are
freed from holding a smartphone and therefore able to interact with the physical
world using both hands. The convenience of this interaction model comes at the
cost of constrained computation resources. The small form-factor of smart
glasses significantly limits their onboard computation capability due to size,
cooling, and battery life reasons. Second, placed at the center of computation
is the unstructured high-dimensional image and video data. Only these data types
can satisfy the need to extract rich semantic information to identify the
progress and mistakes a user makes. Furthermore, state-of-art computer vision
algorithms used to analyze image data are both compute-intensive and challenging
to develop. Third, many cognitive assistants give real-time feedback to users
and have stringent end-to-end latency requirements. An instruction that arrives
too late often provides no value and may even confuse or annoy users. This
latency-sensitivity further increases their high demands of system resource and
optimizations.

To meet the latency and the compute requirements, previous research leverages
edge computing and offloads computation to a cloudlet. A
cloudlet~\cite{satyanarayanan2009case} is a small data-center located at the
edge of the Internet, one wireless hop away from users. Researchers have
developed an application framework for wearable cognitive assistance, named
Gabriel, that leverages cloudlets, optimizes for end-to-end latency, and eases
application
development~\cite{chen2018application}~\cite{ha2014towards}~\cite{chen2017empirical}.
On top of Gabriel, several prototype applications have been built, such as
Ping-Pong Assistance, Lego Assistance, Sandwich Assistance, and Ikea Lamp
Assembly Assistance. Using these applications as benchmarks,
~\cite{chen2017empirical} presents empirical measurements detailing the latency
contributions of individual system components. Furthermore, a multi-algorithm
approach was proposed to reduce the latency of computer vision computation by
executing multiple algorithms in parallel and conditionally selecting a fast and
accurate algorithm for the near future.

While previous research has demonstrated the technical feasibility of wearable
cognitive assistants and meeting latency requirements, many practical concerns
have not been addressed. First, previous work operates the wireless networks and
cloudlets at low utilization in order to meet application latency. The economics
of practical deployment preclude operation at such low utilization. In contrast,
resources are often highly utilized and congested when serving many users. How
to efficiently scale Gabriel applications to a large number of users remains to
be answered. Second, previous work on the Gabriel framework reduces application
development efforts by managing client-server communication, network flow
control, and cognitive engine discovery. However, the framework does not address
the most time-consuming parts of creating a wearable cognitive assistance
application. Experience has shown that developing computer vision modules that
analyze video feeds is a time-consuming and painstaking process that requires
special expertise and involves rounds of trial and error. Developer tools that
alleviate the time and the expertise needed can greatly facilitate the creation
of these applications.

The core contribution of this thesis is to holistic improve the scalability of
wearable cognitive assistance. Scalability, in this thesis, is considered from
three facets. First, from a traditional distributed system perspective, a
scalable system is one that enables most associated clients with fixed amount of
infrastructure and has ways to serve more clients as resources increase. Second,
we also want to enable small software development team to quickly create these
applications. Third, devOps should be able to easily deploy and manage
applications on the fly on a variety of hardware.

% 1. tradition meaning of enabling most associated
% clients with fixed amount of infrastructure 
% 2. enabling small software dev team
% to develop large suite of applications 
% 3. enabling small admin team to deploy a
% large collection of applications

% \section{Motivation}
% \subsection{Wearable Cognitive Assistance}
% \subsection{Use of Cloudlets}
% \subsection{Characteristics of Gabriel Applications}
% \section{Related Work, mostly Zhuo's work}
% \section{Approach}
\section{Thesis Statement}

My thesis is that these efforts can help to scale wearable cognitive assistance.
Notably, we claim that:

\textbf{Two critical challenges to the widespread adoption of wearable cognitive
  assistance are 1) the need to operate cloudlets and wireless network at low
  utilization to achieve acceptable end-to-end latency 2) the level of specialized
  skills and the long development time needed to create new applications. These
  challenges can be effectively addressed through system optimizations,
  functional extensions, and the addition of new software development tools to
  the Gabriel platform.}

\section{Thesis Overview}

The thesis is organized as follows.



% This proposal lays out my plan to address these challenges. In order to meet
% latency requirements when utilization is high, restricting the freedom of using
% resources while taking account of workload characteristics is needed. The scarce
% resource can either be the wireless links or the cloudlets. First, upload
% bandwidth in cellular networks is limited compared to download bandwidth and has
% high variance. Existing wireless infrastructure cannot afford to continuously
% stream high-definition videos from many users. I plan to address this problem
% with application-level mechanisms that exploit the attributes of the workload to
% reduce bandwidth consumption. Second, accelerators, such as GPUs, on cloudlets
% are both limited and heterogeneous. Due to the high demands of accelerators from
% state-of-art computer vision algorithms, the intelligent discovery of
% accelerator resources and the usage coordination among applications are required to
% serve more users. I plan to work on these problems in an edge computing context
% to address how to discover appropriate cloudlets for offload and how to
% coordinate among applications with different latency requirements to share
% scarce accelerators.

% In order to address the difficulty of development, I plan to build tools to
% reduce the expertise and time needed when creating wearable cognitive
% assistants. First, state-of-art computer vision uses Deep Neural Networks (DNNs)
% for critical tasks, including image classification, object detection, and
% semantic segmentation. DNNs champion end-to-end learning instead of hand-crafted
% features. The absence of manually created features provides an opportunity to
% build developer tools that replace ad-hoc trial and error development process.
% On the other hand, DNNs requires a significant amount of labeled data for training. I
% plan to build tools that help label examples and automate the creation of
% DNN-based object detectors.

\chapter{Background}
\label{chapter: background}

\section{Edge Computing}
\label{sec: bg-edge}

\begin{figure*}[t]
\begin{minipage}[c]{2.5in}
\includegraphics[scale=0.45]{FIGS/fig-3tier-A.pdf}
\end{minipage}
\begin{minipage}[c]{2.7in}
\includegraphics[scale=0.25]{FIGS/fig-3tier-B-cropped.pdf}\\
\end{minipage}
\begin{minipage}[c]{1in}
\includegraphics[scale=0.45]{FIGS/fig-3tier-C.pdf}
\end{minipage}
\centering
\captiontext{(Source: Satya et al~\cite{satya2019computing})}
\caption{Tiered Model of Computing}
\label{fig:3tier}
\end{figure*}

{\em Edge computing} is a nascent computing paradigm that has gained
considerable traction over the past few years. It champions the idea of placing
substantial compute and storage resources at the edge of the Internet, in close
proximity to mobile devices or sensors.  Terms such as
``cloudlets''~\cite{Satya2009}, ``micro data centers (MDCs)''~\cite{Greene2012},
``fog''~\cite{Bonomi2012}, and ``mobile edge computing (MEC)''~\cite{Brown2013}
are used to refer to these small, edge-located computing nodes.  We use these
terms interchangably in the rest of this dissertation. Edge computing is motivated by
its potential to improve latency, bandwidth, and scalability over a cloud-only
model.  More practically, some efforts stem from the drive towards
software-defined networking (SDN) and network function virtualization (NFV), and
the fact that the same hardware can provide SDN, NFV, and edge computing
services. This suggests that infrastructure providing edge computing services
may soon become ubiquitous, and may be deployed at greater densities than
content delivery network (CDN) nodes today. 

Satya et al.~\cite{satya2019computing} best describes the modern computing
landscape with edge computing using a tiered model, shown in
Figure~\ref{fig:3tier}. Tiers are separated by distinct yet stable sets of
design constraints. From left to right, this tiered model represents a hierarchy
of increasing physical size, compute power, energy usage, and elasticity. Tier-1
represents today's large-scale and heavily consolidated data-centers. Compute
elasticity and storage permanence are two dominating themes here. Tier-3
represents IoT and mobile devices, which are constrained by their physical size,
weight, and heat dissipation. Sensing is the key functionality of Tier-3
devices. For example, today's smartphones are already rich in sensors, including
camera, microphone, accelerometers, gyroscopes and GPS. In addition, an
increasing amount of IoT devices with specific sensing modalities are getting
adopted, e.g. smart speakers, security cameras, and smart thermostats. 

With the large-scale deployment of Tier-3 devices, there exists a tension
between the gigantic amount of data collected and generated by them and their
limited capabilities to process these data on-board. For example, most
surveillance cameras are limited in computation to run state-of-art computer
vision algorithms to analyze the videos they capture. To overcome this tension,
a Tier-3 device could offload computation over network to Tier-1. This
capability was first demonstrated in 1997 by Noble et al.~\cite{Noble1997}, who
used it to implement speech recognition with acceptable performance on a
resource-limited mobile device. In 1999, Flinn et al.~\cite{Flinn1999} extended
this approach to improve battery life.  These concepts were generalized in a
2001 paper that introduced the term {\em cyber foraging} for the amplification
of a mobile device's data or compute capabilities by leveraging nearby
infrastructure~\cite{Satya2001}.  Thanks to these research efforts, computation
offloading is widely used by IoT devices today. For example, when a user asks an
Amazon Echo smart speaker ``Alexa, what is the weather today?'', the user's
audio stream is captured by the smart speaker and transmitted to the cloud for
speech recognition, text understanding, and question answering.

However, offloading computation to the cloud has its own downside. Because of
the consolidation needed to achieve the economy of scale, today's datacenters
are ``far'' from Tier-3 devices. The latency, throughput, and cost of wide-area
network (WAN) significantly limit the amount of applications that can benefit
from computation offloading. Even worse, it is the logical distance in the
network that matters rather than the physical distance. Routing decisions in
today's Internet are made locally and are based on business agreements,
resulting in suboptimal solutions. For example, using traceroute, we determine
that a LTE packet originating from a smartphone on the campus of Carnegie Mellon
University (CMU) in Pittsburgh to a nearby server actually traverses to
Philedelphia, a city several hundreds miles away. This is because Philidephia
has the nearest peering point of the particular commercial LTE network in use to
the public Internet. In 2010, Li et al.~\cite{li2010cloudcmp} report that the
average round trip time (RTT) from 260 global vantage points to their optimal
Amazon EC2 instances is 74 ms. In addition to long network delay, the high
network fan-in of datacenters means its aggregation network needs to carry
significant amount of traffic. As the number of Tier-3 devices is expected to
grow exponentially, these network links face significant challenges to handle
the ever-increasing volume of ingress traffic. 

\begin{figure}
\begin{minipage}[c]{3in}
\begin{center}
\includegraphics[height=1.5in]{FIGS/ping_cdf.pdf}
\captiontext{(Source: Hu et al~\cite{hu2016quantifying})}
\caption{CDF of pinging RTTs}
\label{fig:ping-CDF}
\end{center}
\end{minipage}
\begin{minipage}[c]{3.5in}
\begin{center}
    \hspace{0.2in}
    \includegraphics[width=0.8\textwidth,clip,trim=91pt 376pt 0 0]{FIGS/Legend-Wifi.pdf}
    \includegraphics[width=3in]{FIGS/Face-LTE.pdf}
\captiontext{(Source: Hu et al~\cite{hu2016quantifying})}
\caption{FACE Response Time over LTE}
\label{fig:response-time-lte-face}
\end{center}
\end{minipage}
\end{figure}

To counter these problems, edge computing, shown as the Tier-2 in
Figure~\ref{fig:3tier}, is proposed. Cloudlets at Tier-2 creates the illusion of
bringing Tier-1 ``closer''. They are featured by their network proximity to
Tier-3 devices and their significantly larger compute and storage compared to
Tier-3 devices. While Tier-3 devices typically run on battery and are optimized
for low energy consumption, saving energy is not a major concern for Tier-2 as
they are either plugged into the electric grid or powered by other sources of
energy (e.g. fuels in a vehicle). Cloudlets serve two purposes in the tiered
model. First, they provide infrastructure for compute-intensive and
latency-sensitive applications for Tier-3. Wearable cognitive assistance is an
examplar of these applications. Second, by processing data closer to the source
of the content, it reduces the excessive traffic going into Tier-1 datacenters.
Figure~\ref{fig:ping-CDF} shows the RTT comparison of PING to the cloud and the
cloudlet over WiFi and LTE. Cloudlet's RTT is on average 80 to 100ms shorter
than its counterpart to the cloud. Figure~\ref{fig:response-time-lte-face} shows
the impact of network latency on an application that recognizes faces. Three
offloading scenarios are considered: offloading to the cloud, offloading to the
cloudlet, and no offloading. The data transmitted are images captured by a
smartphone. As we can see, the limited bandwidth of the cellular network further
worsen the response time when offloading to the cloud. In fact, for this
particular application, even local execution outperforms offloading to the
nearest cloud due to network delay. The optimal computational offload location
is cloudlet, whose median response time is more than 200ms faster than local
execution and about 250 ms faster than the nearest cloud.

The low-latency and high-bandwidth compute infrastructure provided by cloudlets
is an indispensible foundation for latency-sensitive and compute-intensive
wearable cognitive assistance. Cloudlet also poses unique challenges for
scalability as resources are a lot more limited compared to datacenters. How to
scale WCAs to many users using cloudlets is the key question we set out to
investigate in this dissertation. 


\section{Gabriel Platform}
\label{sec: bg-gabriel}

\begin{figure}
\centering
\includegraphics[width=0.8\linewidth]{FIGS/fig-backend-structure-simple-crop.pdf}
\begin{captiontext}
{\rm (Source: Chen et al~\cite{chen2017empirical})}
\end{captiontext}
\caption{Gabriel Platform}
\label{fig:gabriel}
\end{figure}

The Gabriel platform~\cite{ha2014towards,chen2017empirical}, shown in
Figure~\ref{fig:gabriel}, is the first application framework for wearable
cognitive assistance. It consists of a front-end running on wearable devices and
a back-end running on cloudlets. The Gabriel front-end performs preprocessing of
sensor data (e.g., compression and encoding), which it then streams over a
wireless network to a cloudlet.  The Gabriel back-end on the cloudlet has a
modular structure. The {\em control module} is the focal point for all
interactions with the wearable device and can be thought as an agent for a
particular client on the cloudlet. A publish-subscribe (PubSub) mechanism
distributes the incoming sensor streams to multiple {\em cognitive modules}
(e.g., task-specific computer vision algorithms) for concurrent processing.
Cognitive module outputs are integrated by a task-specific {\em user guidance
module} that performs higher-level cognitive processing such as inferring task
state, detecting errors, and generating guidance in one or more modalities
(e.g., audio, video, text, etc.). The Gabriel platform automatically discovers
cognitive engines on the local network via a universal plug-and-play (UPnP)
protocol. The platform is designed to run on a small cluster of machines with
each modules capable of being separated or co-located with other modules via
process, container, or virtual machine virtualization. 

The original Gabriel platform was built with a single user in mind, and did not
have mechanisms to share cloudlet resources in a controlled manner.  It did,
however, have a token-based transmission mechanism.  This limited a client to
only a small number of outstanding operations, thereby offering a simple form of
rate adaptation to processing or network bottlenecks.  We have retained this
token mechanism in our system, described in the rest of this dissertation. In
addition, we have extended Gabriel with new mechanisms to handle multi-tenancy,
perform resource allocation, and support application-aware adaptation.  We refer
to the two versions of the platform as ``Original Gabriel'' and ``Scalable
Gabriel.''

\section{Example Gabriel Applications}
\label{sec:example-apps}

Many applications have been built on top of the Gabriel platform.
Figure~\ref{fig:bg-apps-table} provides a summary of applications built by Chen
et al~\cite{chen2018application}. These applications run on multiple wearable
devices such as Google Glass, Microsoft HoloLens, Vuzix Glass, and ODG R7. The
cloudlet processing of these applications consists of two major phases. The
first phase uses computer vision to extract a symbolic, idealized representation
of the state of the task, accounting for real-world variations in lighting,
viewpoint, etc.  The second phase operates on the symbolic representation,
implements the logic of the task at hand, and occasionally generates guidance
for the user.  In most WCA applications, the first phase is far more compute
intensive than the second phase. While visual data is the focus of the analysis,
other types of sensor data, e.g. audio, are also used to help infer user states.

Building on lessons learned by Chen et al~\cite{chen2018application}, we create
and implement several real-life WCA applications whose tasks are more complex.
They also pose more challenges to the computer vision processing as the parts
are not designed for machine recognition. In particular, we focus on assembly
tasks. Many assembly tasks, including manufacturing assembly and medical
procedures, are tedious and error-prone. WCAs can help reduce errors, provide
training, and assist human operators by continuously monitoring their actions
and offering feedback. We describe two of these applications below in detail.

\begin{figure*}
\begin{tabular}{|p{0.31in}|p{0.95in}|p{3.5in}|p{0.84in}|p{0.8in}|}
\hline
App Name & Example Input Video Frame & Description & Symbolic \phantom{000} Representation & Example Guidance \\
\hline
\phantom{000} \textbf{Pool}     & \raisebox{-0.9\totalheight}{\psfig{file=FIGS/bigtable2/example-pool.png, width=0.97in}}
&
Helps a novice pool player aim correctly. Gives continuous visual feedback (left arrow, right arrow, or thumbs up) as the user turns his cue stick. Correct shot angle is calculated based on fractional aiming system~\cite{FractionalAiming}. Color, line, contour, and shape detection are used. The symbolic representation describes the positions of the balls, target pocket, and the top and bottom of cue stick.
&
\phantom{000} $<$Pocket, object ball, cue ball, cue top, cue bottom$>$ & \phantom{000} \raisebox{-0.85\totalheight}{\psfig{file=FIGS/bigtable2/guidance-pool.png, width=0.8in}} \\
\hline
\phantom{000} \textbf{Ping-pong} & \raisebox{-0.9\totalheight}{\psfig{file=FIGS/bigtable2/example-pingpong.png, width=0.97in}}
&
Tells novice to hit ball to the left or right, depending on which is more likely to beat opponent. Uses color, line and optical-flow based motion detection to detect ball, table, and opponent. The symbolic representation is a 3-tuple: in rally or not, opponent position, ball position. Whispers ``left'' or ``right'' or offers spatial audio guidance using~\cite{tang2014assistive}.

    Video URL: {\em \href{https://youtu.be/\_lp32sowyUA}{https://youtu.be/\_lp32sowyUA}}
&
\phantom{000} $<$InRally, ball position, opponent position$>$ & \phantom{000} Whispers ``Left!'' \\
\hline
\phantom{000} \textbf{Work-out} & \raisebox{-0.9\totalheight}{\psfig{file=FIGS/bigtable2/example-workout.png, width=0.97in}}
&
Guides correct user form in exercise actions like sit-ups and push-ups, and counts out repetitions. Uses Volumetric Template Matching~\cite{ke2007event} on a 10-15 frame video segment to classify the exercise.
%A poorly-performed repetition is classified as a distinct type of exercise (e.g. ``good pushup'' versus ``bad pushup'').
Uses smart phone on the floor for third-person viewpoint.
&
\phantom{000} $<$Action, count$>$ & \phantom{000} Says ``8 '' \\
\hline
\phantom{000} \textbf{Face}     & \raisebox{-0.9\totalheight}{\psfig{file=FIGS/bigtable2/example-face.png, width=0.97in}}
&
Jogs your memory on a familiar face whose name you cannot recall. Detects and extracts a tightly-cropped image of each face, and then applies a state-of-art face recognizer using deep residual network~\cite{He2016}. Whispers the name of a person.
Can be used in combination with Expression~\cite{anam2014expression} to offer conversational hints.
&
\phantom{000} ASCII text of name & \phantom{000} Whispers ``Barack Obama'' \\
\hline
\phantom{000} \textbf{Lego} & \raisebox{-0.9\totalheight}{\psfig{file=FIGS/bigtable2/example-lego.png, width=0.97in}}
&
Guides a user in assembling 2D Lego models. Each video frame is analyzed in three steps: (i) finding the board using its distinctive color and black dot pattern; (ii) locating the Lego bricks on the board using edge and color detection; (iii) assigning brick color using weighted majority voting within each block. Color normalization is needed. The symbolic representation is a matrix representing color for each brick.

    Video URL: {\em \href{https://youtu.be/7L9U-n29abg}{https://youtu.be/7L9U-n29abg}}
&
\phantom{000} [[0, 2, 1, 1], \break [0, 2, 1, 6], \break [2, 2, 2, 2]] \break & \raisebox{-0.85\totalheight}{\psfig{file=FIGS/bigtable2/guidance-lego.png, width=0.8in}} Says ``Put a 1x3 green piece on top'' \\
\hline
\phantom{000} \textbf{Draw} & \raisebox{-0.9\totalheight}{\psfig{file=FIGS/bigtable2/example-draw.png, width=0.97in}}
&
Helps a user to sketch better. Builds on third-party app~\cite{Iarussi2013} that was originally designed to take input sketches from pen-tablets and to output feedback on a desktop screen. Our implementation preserves the back-end logic. A new Glass-based front-end allows a user to use any drawing surface and instrument. Displays the error alignment in sketch on Glass.

    Video URL: {\em \href{https://youtu.be/nuQpPtVJC6o}{https://youtu.be/nuQpPtVJC6o}}
&
\raisebox{-0.85\totalheight}{\psfig{file=FIGS/bigtable2/symbolic-draw.png, width=0.7in}} & \raisebox{-0.85\totalheight}{\psfig{file=FIGS/bigtable2/guidance-draw.png, width=0.8in}} \\
\hline
\phantom{000} \textbf{Sand-wich} & \raisebox{-0.9\totalheight}{\psfig{file=FIGS/bigtable2/example-sandwich.png, width=0.97in}}
&
Helps a cooking novice prepare sandwiches according to a recipe. Since real food is perishable, we use a food toy with plastic ingredients. Object detection follows the state-of-art faster-RCNN deep neural net approach~\cite{ren2015faster}. Implementation is on top of Caffe~\cite{jia2014caffe} and Dlib~\cite{dlib09}. Transfer learning~\cite{Pan2010} helped us save time in labeling data and in training.

    Video URL: {\em \href{https://youtu.be/USakPP45WvM}{https://youtu.be/USakPP45WvM}}
&
\phantom{000} Object: \break ``E.g. Lettuce on top of ham and bread'' & \raisebox{-0.9\totalheight}{\psfig{file=FIGS/bigtable2/guidance-sandwich.jpg, width=0.7in}} Says ``Put a piece of bread on the lettuce'' \\
\hline

\end{tabular}
\caption{Prototype Wearable Cognitive Assistance Applications}
\label{fig:bg-apps-table}
\end{figure*}

\begin{figure}
\centering
\includegraphics[width=0.5\linewidth]{FIGS/RibLoc.jpg}
\caption{RibLoc Kit}
\label{fig:ribloc}
\end{figure}

\begin{figure}
\centering
\includegraphics[width=\linewidth]{FIGS/ribloc-fsm-crop.pdf}
\caption{RibLoc Assistant Workflow}
\label{fig:ribloc-app}
\end{figure}

\subsection{RibLoc Application}

RibLoc Fracture Plating System~\cite{ribloc}, shown in Figure~\ref{fig:ribloc}
is used by surgeons to stabilize broken ribs for fracture repair. The surgery
consists of five major steps: measure ribs thickness, prepare the plate, drill
bone for screw insertion, insert screws, and tighten screws. To train doctors
how to use this kit, Acute Innovations, the manufacturer of RibLoc, currently
send sales personnel to doctors' office, often for extended period of time. To
reduce the cost of training, we develop a WCA for RibLoc that guides a surgeon
to learn RibLoc step-by-step on fake bones. Figure~\ref{fig:ribloc-app} shows
the task steps of the application as a finite state machine (FSM). Conditions of
state transitions are shown above the transition arrows and instructions given
to users are quoted in italics. Most of user state recognition is achieved by
verifying the existence and locations of key objects. One exception is rib
thickness measurement. The application asks the user to read out some text from
the gauge. Automatic speech recognition (ASR) is used to detect the user's
read-out. ASR is used because the text on the gauge has a low contrast with the
background that they are too hard to optical character recognition (OCR). A demo
video of RibLoc WCA is at \url{https://youtu.be/YRTXUty2P1U}.

\subsection{DiskTray Application}

\begin{figure}
\centering
\includegraphics[width=0.3\linewidth]{FIGS/disktray.jpg}
\caption{Assembled DiskTray}
\label{fig:disktray}
\end{figure}

\begin{figure}
\centering
\includegraphics[width=0.5\linewidth]{FIGS/disktray-challenge.jpg}
\caption{Small Parts in DiskTray WCA}
\label{fig:disktray-challenge}
\end{figure}

In collaboration with InWin~\cite{inwin}, a computer hardware manufacturer, we
created a cognitive assistance to train operators how to assemble their disk
tray product. Figure~\ref{fig:disktray} shows the assembled tray, which is used
to host hard drives that go into a server chassis. As with all the other WCAs,
we do not modify the original parts. 

We face two challenges when building this application. First, some parts are
tiny. In one step, the application needs to check if a pin is placed correctly
into a slot. As shown in Figure~\ref{fig:disktray-challenge}, both the slot and
the pin are tiny and hard to see. To facilitate the detection of these two
parts, we ask the user to bring the parts close to the camera in addition to
zooming the camera and turning on the camera flashlight. Second, there is a
plastic strip that is translucent. Transparent objects are extremely challenging
to detect using 2D RGB cameras, because their appearance depends on the
background and lighting~\cite{lysenkov2013recognition}. Instead of detecting the
placement of the transparent strip by computer vision, we leverage the human
operator to signal to the system when the operation is completed. InWin
showcased this application at 2018 Computex Technology Show~\cite{computex}. A
demo video of the Computex Demo can be viewed
at~\url{https://www.youtube.com/watch?v=AwWZcL9XGI0}.


\section{Application Latency Bounds}

\begin{figure}
\centering
\begin{tabular}{|l|c|c|c|c|c|c|c|}
\hline
                      & POOL & WORK OUT & PING PONG & FACE &
                      \multicolumn{3}{c|}{
                          \begin{tabular}{@{}c@{}}Assembly Tasks\\ (e.g. RibLoc)\end{tabular}} \\
\hline
\Xhline{2\arrayrulewidth}
\hline
    Bound Range (tight-loose) & 95-105 & 300-500 & 150-230 & 370-1000 & \multicolumn{3}{c|}{600-2700} \\
\hline
\end{tabular}
\caption{Application Latency Bounds (in milliseconds)}
\begin{captiontext}
{\rm (Adapted from Chen et al~\cite{chen2017empirical})}
\end{captiontext}
\label{fig:bg-bounds}
\end{figure}

Not only are the accuracy of the instructions important to WCAs, but the speed
at which these instructions are delivered is also critical. With a human in the
loop, the latency requirements of WCAs are closely related to the task-specific
human speed. For example, for assembly tasks, an instruction delivered tens of
seconds after a user has finished a step can cause user annoyance and
frustration. For PING PONG assistant, a task that is even more fast-paced, an
instruction on where to hit the ball becomes useless if it is delivered after
the user has made a hit. 

Previous work~\cite{chen2017empirical} quantifies task-dependent application
latency bounds by answering the question \textit{How fast does an instruction
need to be delivered?} Three different approaches are used. For tasks that have
published human speed, numbers from the literature are used as the upper bound
of the end-to-end system response time. For example, it takes about 1000 ms for
a human to recognize the identity of a face. Therefore, a face recognition
assistant should deliver a person's name faster than 1000ms to exceed human
speed. For tasks in which users interact with physical systems, the latency
bounds can be derived directly from first principles of physics. For instance,
the latency bounds for PINGPONG assistant are calculated by subtracting audio
perception time, motion initiation time, and swinging time from the average time
between opponents hitting the ball. In addition, an user study of LEGO assembly
assistant is also conducted to deduct latency bounds for assembly tasks.

Figure~\ref{fig:bg-bounds} shows a summary of latency bounds calculated from the
previous work~\cite{chen2017empirical}. Each application is assigned both a
tight and a loose bound from the application perspective. The tight bound
represents an ideal target, below which the user is insensitive to improvements.
Above the loose bound, the user becomes aware of slowness, and user experience
and performance is significantly impacted. Latency improvements between the two
limits may be useful in reducing user fatigue.

These application latency bounds can be considered as application quality of
service (QoS) metrics. Similar to bitrate and startup time in video streaming,
these metrics serve as measurable proxies to user experience. When the system
delay increases, we can compare the delay with these latency bounds to estimate
how much a user is suffering. In this dissertation, we use these bounds to
formulate application utility functions, which quantify user experience when a
system response is delayed due to contention. These application utility
functions are the foundation for our adaptation-centered approach to
scalability.

\chapter{Application-Agnostic Techniques to Reduce Network Transmission}
\label{chapter: bandwidth}

WCAs continuously stream sensor data to a cloudlet. The richer a sensing
modality is, the more information can be extracted. The core sensing modality of
WCAs is visual data, e.g. egocentric images and videos from wearable cameras.
Compared to other sensors, e.g. microphones and inertial measurement units
(IMUs), cameras capture visual information with rich semantics. As commercial
camera hardware become more affordable, they have become increasingly pervasive.
In 2013, it is estimated that there is one security camera for every eleven
citizens in the UK~\cite{Barrett2013}. In the meantime, deep neural networks
(DNNs) have driven significant advancement in computer vision and have achieved
human-level accuracy in several previously intractable perception problems (e.g.
face recognition, image classification)~\cite{learned2016labeled,
    schroff2015facenet}. The richness and the open-endness of visual data makes
camera the ideal sensor for WCAs.

However, continuous video transmission from many Tier-3 devices places severe
stress on the wireless spectrum.  Hulu estimates that its video streams require
13 Mbps for 4K resolution and 6 Mbps for HD resolution using highly optimized
offline encoding~\cite{Hulu2017}. Live streaming is less bandwidth-efficient, as
confirmed by our measured bandwidth of 10 Mbps for HD feed at 25 FPS. Just 50
users transmitting HD video streams continuously can saturate the theoretical
uplink capacity of 500 Mbps in a 4G LTE cell that covers a large rural
area~\cite{LteWorld2009}.  This is clearly not scalable.

In this chapter, we show how per-user bandwidth demand in WCA-like live video
analytics can be significantly reduced using an application-agnostic approach.
We aim to reduce bandwidth demand without compromising the timeliness or
accuracy of results. In contrast to previous
works~\cite{Wang2017networked,zhang2015design,Wang2016skyeyes}, we leverage
state-of-the-art deep neural networks (DNNs) to selectively transmit interesting
data from a video stream and explore environment-specific optimizations. The
accuracy of the data selection is important, as fewer false positives result in
lower network bandwidth and cloudlet computing cycle consumption.

This chapter is organized as follows. We first discuss the challenges of running
DNNs for visual perception solely on Tier-3 devices in
Section~\ref{bw:challenges}. Next, we propose and compare two
application-agnostic techniques to reduce network transmission. We present our
results first in the context of live video analytics for small autonomous
drones. Both as emerging Tier-3 devices, drones and wearable devices face
similar challenges in live video analysis. Finally, we showcase how these
techniques can be applied to WCAs in Section~\ref{bw:wca}.

\section{Video Processing on Mobile Devices}
\label{bw:challenges}

In the context of real-time video analytics, Tier-3 devices represent
fundamental mobile computing challenges that were identified two decades
ago~\cite{Satya1996}.  Two challenges have specific relevance here. First,
mobile elements are resource-poor relative to static elements.  Second, mobile
connectivity is highly variable in performance and reliability.  We discuss
their implications below.


\subsection{Computation Power on Tier-3 Devices}
\label{bw:payload}

Unfortunately, the hardware needed for deep video stream processing in real time
is larger and heavier than can fit on a typical Tier-3 device. State-of-art
techniques in image processing use DNNs that are compute- and memory-intensive.
Figure~\ref{fig:onboard-dnn-speed} presents experimental results on two
fundamental computer vision tasks, image classification and object detection, on
four different devices. In the figure, MobileNet V1 and ResNet101 V1 are image
classification DNNs. Others are object detection DNNs. Both tasks used publicly
available pretrained DNN models. We carefully choose hardware platforms to
represent a range of computation capabilities of Tier-3 devices. To anticipate
future improvements in smartphone technology, our experiments also consider more
powerful devices such as the Intel$^{\tiny\textregistered}$
Joule~\cite{Hardawar2016} and the NVIDIA Jetson~\cite{NVIDIA2017} that are
physically compact and light enough to be credible as wearable device platforms
in a few years.

Fig.~\ref{fig:onboard-dnn-speed}, we present the best results we could obtain on
each platform. This is not intended to directly compare frameworks and platforms
(as others have been doing~\cite{Zhang2018pcamp}), but rather to illustrate the
differences between wearable platforms and fixed infrastructure servers. 

\begin{figure}
\centering
\begin{flushleft}
M: MobileNet V1; R: ResNet101 V1;
S-M: SSD MobileNet V1; S-I: SSD Inception V2;\\F-I: Faster R-CNN Inception V2;
F-R: Faster R-CNN ResNet101 V1
\end{flushleft}
\sisetup{
    table-format=3,
    table-number-alignment=left
}
\hspace{-0.8in}
\begin{tabular}{|p{1.3cm}|S[table-column-width=1.3cm]|p{3.0cm}|p{1.8cm}|S[table-column-width=1.2cm]|S[table-column-width=1.4cm]|S[table-column-width=1.3cm]|S[table-column-width=1.3cm]|S[table-format=4, table-number-alignment=left, table-column-width=1.5cm]|S[table-format=5, table-number-alignment=left, table-column-width=1.9cm]|}
\hline
{\multirow{2}{*}{}} & {\multirow{2}{*}{}} & {\multirow{2}{*}{}}                                        & {\multirow{2}{*}{}}                                                                        & \multicolumn{2}{c|}{\parbox[t]{1.8cm}{\centering Image\\Classification\\~}}                                & \multicolumn{4}{c|}{Object Detection} \\ \cline{5-10}
                  &  {\parbox[t]{0.9cm}{\centering Weight\\(g)}}
                  & \centering CPU
                  & \centering GPU
                  & {\parbox[t]{0.9cm}{\centering M\\(ms)}}
                  & {\parbox[t]{1.1cm}{\centering R\\(ms)}}
                  & {\parbox[t]{1.0cm}{\centering S-M\\(ms)}}
                  & {\parbox[t]{1.0cm}{\centering S-I\\(ms)}}
                  & {\parbox[t]{1.3cm}{\centering F-I\\(ms)}}
                  & {\parbox[t]{1.6cm}{\centering F-R\\(ms)}} \\ \hline
Nexus~6    & 184                                                      & 4-core 2.7 GHZ Krait 450, 3GB Mem                           & Adreno 420                                                                              & 353 {\scriptsize \ (67)}                                                 & 983 {\footnotesize \ (141)}                                                   & 441 {\footnotesize \ (60)           }                               & 794 {\footnotesize \ (44)            }                                       & {\small ENOMEM}                                                                 & {\small ENOMEM}                                                               \\ \hline
Intel$^{\tiny \textregistered}$ Joule 570x     & 25                                                       & 4-core 1.7 GHz Intel Atom$^{\tiny\textregistered}$ T5700, 4GB Mem                          & Intel$^{\tiny\textregistered}$ HD Graphics (gen 9)                                                               & 37 {\scriptsize \ (1)$\ddagger$ }                                       & 183 {\footnotesize \ (2)$\dagger$$\ddagger$ }                                        & 73  {\footnotesize \ (2)$\ddagger$ }                               & 442 {\footnotesize \ (29)            }                                       & 5125 {\footnotesize \ (750)}                                                             & 9810 {\footnotesize \ (1100)}                                                          \\ \hline
NVIDIA Jetson TX2     & 85                                                       & 2-Core 2.0 GHz Denver2 + 4-Core 2.0 GHz Cortex-A57, 8GB Mem & 256 cuda core 1.3 GHz NVIDIA Pascal                                                          & 13 {\scriptsize \ (0)$\dagger$  }                                       & 92 {\footnotesize \ (2)$\dagger$ }                                          & 192 {\footnotesize \ (18)           }                               & 285 {\footnotesize \ (7)$\dagger$   }                                       & {\small ENOMEM}                                                                 & {\small ENOMEM}                                                               \\ \hline
Rack-mounted Server     &                                                          & 2x 36-core 2.3 GHz Intel$^{\tiny\textregistered}$ Xeon$^{\tiny\textregistered}$ E5-2699v3 Processors, 128GB Mem     & 2880 cuda core 875MHz NVIDIA Tesla K40, 12GB GPU Mem                             & 4 {\scriptsize \ (0)$\ddagger$  }                                       & 33 {\footnotesize \ (0)$\dagger$  }                                          & 12  {\footnotesize \ (2)$\ddagger$ }                               & 70  {\footnotesize \ (6)             }                                       & 229 {\footnotesize \ (4)$\dagger$}                                                               & 438 {\footnotesize \ (5)$\dagger$}                                                               \\ \hline
\end{tabular}
\begin{captiontext}
\vspace{0.1in}
Figures above are means of 3 runs across 100 random images. The time shown
includes only the forward pass time using batch size of 1. ENOMEM indicates
failure due to insufficient memory. Figures in parentheses are standard
deviations. The weight figures for Joule and Jetson include only the modules
without breakout boards. Weight for Nexus~6 includes the complete phone with
battery and screen. Numbers are obtained with TensorFlow (TensorFlow Lite for
Nexus 6) unless indicated otherwise. \\
$\dagger$ indicates GPU is used. $\ddagger$ indicates
Intel$^{\tiny\textregistered}$ Computer Vision SDK beta 3 is used.
\end{captiontext}
\caption{Deep Neural Network Inference Speed}
\label{fig:onboard-dnn-speed}
\end{figure}

Image classification maps an image into categories, with each category
indicating whether one or many particular objects (e.g., a human survivor, a
specific animal, or a car) exist in the image.  The prediction speed using two
different DNNs are shown. MobileNet V1~\cite{Howard2017} is a DNN
designed for mobile devices from the ground-up by reducing the number of
parameters and simplifying the computation using
depth-wise separable convolution. ResNet101 V1~\cite{He2016} is a more accurate
but also more resource-hungry DNN that won the ImageNet
classification challenge in 2015~\cite{Russakovsky15}. 

Object detection is a harder task than image classification, because it requires
bounding boxes to be predicted around the specific areas of an image that
contains a particular class of object. Object detection DNNs are built on top of
image classification DNNs by using image classification DNNs as low-level
feature extractors. Since feature extractors in object detection DNNs can be
changed, the DNN structures excluding feature detectors are referred as object
detection meta-architectures. We benchmarked two object detection DNN
meta-architectures: Single Shot Multibox Detector (SSD)~\cite{Liu2016} and
Faster R-CNN~\cite{Ren2015}. We used multiple feature extractors for each
meta-architecture. The meta-architecture SSD uses simpler methods to identify
potential regions for objects and therefore requires less computation and runs
faster. On the other hand, Faster R-CNN~\cite{Ren2015} uses a separate region
proposal neural network to predict regions of interest and has been shown to
achieve higher accuracy~\cite{Huang2017}. Figure~\ref{fig:onboard-dnn-speed}
presents results in four columns: SSD combined with MobileNet V1 or Inception
V2, and Faster R-CNN combined with Inception V2 or ResNet101 V1~\cite{He2016}.
The combination of Faster R-CNN and ResNet101 V1 is one of the most accurate
object detectors available today~\cite{Russakovsky15}. The entries marked ``{\sc
ENOMEM}'' correspond to experiments that were aborted because of insufficient
memory.

These results demonstrates the computation gap between mobile and static
elements. While the most accurate object detection model Faster R-CNN Resnet101 V1
can achieve more than two FPS on a server GPU, it either takes several seconds
on  mobile platforms or fails to execute due to insufficient memory. In
addition, the figure also confirms that sustaining open-ended real-time video
analytics on smartphone form factor computing devices is well beyond the state
of the art today and may remain so in the near future.  This constrains what is
achievable with Tier-3 devices.

\subsection{Result Latency, Offloading \& Scalability}
\label{bw:offloading}

{\em Result latency} is the delay between first capture of a video frame in
which a particular result (e.g., image of a survivor) is present, and report of
its discovery or feedback based on the discovery after video processing.
Operating totally disconnected, a Tier-3 device can capture and store video, but
defer its processing until the mission is complete.  At that point, the data can
be uploaded from the device to the cloud and processed there.  This approach
completely eliminates the need for real-time video processing, obviating the
challenges of Tier-3 computation power mentioned previously. Unfortunately, this
approach delays the discovery and use of knowledge in the captured data by a
substantial amount (e.g., many tens of minutes to a few hours).  Such delay may
be unacceptable in use cases such as search-and-rescue using drones, or
step-by-step instruction feedback from wearable devices. In this chapter, we focus
on approaches that aim for much smaller result latency: ideally, close to
real-time.

A different approach is to offload video processing during flight over a
wireless link to an edge computing node (cloudlet). With this approach, even a
weak Tier-3 device can leverage the substantial processing capability of a
ground-located cloudlet, without concern for its weight, size, heat dissipation,
or energy usage.  Much lower result latency is now possible.  However, even if
cloudlet resources are viewed as ``free'' from the viewpoint of mobile
computing, the Tier-3 device consumes wireless bandwidth in transmitting video.

Today, 4G LTE offers the most plausible wide-area connectivity from a Tier-3
device to its associated cloudlet.  The much higher bandwidths of 5G are still
many years away, especially at global scale.  More specialized wireless
technologies, such as Lightbridge 2~\cite{LightBridge2} for drones, can also be
used.  Regardless of specific wireless technology, the principles and techniques
described in this chapter apply.

{\em Scalability,} in terms of maximum number of concurrently operating Tier-3
devices within a 4G LTE cell becomes an important metric.  In this chapter we
explore how the limited processing capability on a Tier-3 device can be used to
greatly decrease the volume of data transmitted, thus improving scalability
while minimally impacting result accuracy and result latency.

Note that the uplink capacity of 500 Mbps per 4G LTE cell assumes standard
cellular infrastructure that is undamaged.  In natural disasters and military
combat, this infrastructure may be destroyed. Emergency substitute
infrastructure, such as Google and AT\&T's partnership on balloon-based 4G LTE
infrastructure for Puerto Rico after hurricane Maria~\cite{Morse2017}, can only
sustain much lower uplink bandwidth per cell, e.g. 10Mbps for the balloon-based
LTE~\cite{Sankaran2018}.  Conserving wireless bandwidth from Tier-3 video
transmission then becomes even more important, and the techniques described here
will be even more valuable.

% \emph{Result accuracy} influences a second dimension of scalability, namely
% the ability of one individual to supervise the result streams from
% many drones.  The output of each video processing pipeline should only
% demand occasional human attention.  The accuracy, sophistication, and
% speed of this pipeline determines the cognitive load on mission
% personnel for a given video stream.  For example, a pipeline that has
% virtually no false positives or false negatives in detecting survivors
% will consume less supervisory human attention than a mediocre
% pipeline.  That will allow one person to confidently supervise a large
% swarm that rapidly covers a large search area.

\section{Baseline: {\xc Dumb}  S\lc{trategy}}
\label{sec:dumbdrone}

\subsection{Description}

We first establish and evaluate the baseline case of no image processing
performed at the Tier-3 device.  Instead, all captured video is immediately
transmitted to the cloudlet.  Result latency is very low, merely the sum of
transmission delay and cloudlet processing delay. We use drones as the example
of Tier-3 devices and drone video search as the scenario of video analytics
first. We later demonstrate how to apply the techniques developed for drone
search to WCAs.

\begin{figure}
\centering
\begin{tabular}{|p{1cm}|p{2.5cm}|p{2.5cm}|p{2.5cm}|p{2.5cm}|p{2.5cm}|}
\hline
   & Detection & Data & Data & Training & Testing \\ 
Task& Goal & Source & Attributes & Subset & Subset\\ 
\hline
T1 & {\small People in scenes of daily life}&{\small Okutama Action Dataset~\cite{Barekatain2017}}&\makecell[tl]{\small 33 videos \\\small 59842 fr\\\small 4K@30~fps}&\makecell[tl]{\small 9 videos\\\small 17763 fr}&\makecell[tl]{\small 6 videos\\\small 20751 fr}\\ 
\hline
T2 &{\small Moving cars}&{\small Stanford Drone Dataset~\cite{Robicquet2016}}&\makecell[tl]{\small 60 videos \\\small 522497 fr\\\small 1080p@30~fps}&\makecell[tl]{\small 16 videos\\\small 179992 fr} & \multirow{4}{*}{\parbox{1.5cm}{\centering\small 14 videos 92378 fr\\ Combination of test videos from each dataset.}} \\ \cline{1-5}
  %% \makecell[tl]{\small 14 videos\\\small 92378 fr\\\small Combination of \\\small human, car, raft \\\small and elephant videos\\\small from each datasets}} \\ \cline{1-5}
%% \makecell[tl]{\small 14 videos\\\small 92378 fr}\\ 
%% \hline
T3 &{\small Raft in flooding scene}&{\small YouTube collection~\cite{YouTube1}}&\makecell[tl]{\small 11 videos \\\small 54395 fr\\\small 720p@25~fps}&\makecell[tl]{\small 8 videos\\\small 43017 fr} & \\ \cline{1-5}
%% \makecell[tl]{\small 14 videos\\\small 92378 fr}\\
%% \hline
T4 &{\small Elephants in natural habitat}&{\small YouTube collection~\cite{YouTube2}}&\makecell[tl]{\small 11 videos \\\small 54203 fr\\\small 720p@25~fps}&\makecell[tl]{\small 8 videos\\\small 39466 fr} & \\ \cline{1-5}
%% \makecell[tl]{\small 14 videos\\\small 92378 fr}\\
%% \hline
% T5 &{\small Pushing or pulling Suitcases}&\small Okutama Action Dataset &\small Same as T1 &\small Same as T1 & \\
%% \small Same as T1\\
\hline
\end{tabular}
\vspace{0.1in}
\begin{captiontext}
fr = ``frames''\\
fps = ``frames per second''\\
No overlap between training and testing subsets of data
\end{captiontext}
\caption{Benchmark Suite of Drone Video Traces}
\label{fig:benchmarksuite}
\end{figure}

\begin{figure}
\centering
\begin{tabular}{|c|c|c|c|c|}
\hline
     & Total & Avg & & \\ 
     & Bytes & BW & & \\ 
Task & (MB) & (Mbps) & Recall & Precision \\ 

\hline
T1 & \phantom{0}924 & 10.7 & 74\% & 92\%\\ 
\hline
T2 & 2704 & \phantom{0}7.0 & 66\% & 90\%\\ 
\hline
\end{tabular}
\vspace{0.2in}
\begin{captiontext}
Peak bandwidth demand is same as average since video is transmitted
continuously. Precision and recall are at the maximum F1 score.
\end{captiontext}
\caption{Baseline Object Detection Metrics}
\label{fig:baseline}
\end{figure}

\subsection{Experimental Setup}
\label{sec:dumbdrone-setup}

To ensure experimental reproducibility, our evaluation is based on
replay of a benchmark suite of pre-captured videos rather than on
measurements from live drone flights.  In practice, live results may
diverge slightly from trace replay because of non-reproducible
phenomena.  These can arise, for example, from wireless propagation
effects caused by varying weather conditions, or by seasonal changes
in the environment such as the presence or absence of leaves on trees.
In addition, variability can arise in a  drone's pre-programmed flight
path due to collision avoidance with moving obstacles such as birds,
other drones, or aircraft.

All of the pre-captured videos in the benchmark suite are publicly accessible,
and have been captured from aerial viewpoints. They characterize drone-relevant
scenarios such as surveillance, search-and-rescue, and wildlife conservation.
Figure~\ref{fig:benchmarksuite} presents this benchmark suite of videos,
organized into four tasks. All the tasks involve detection of tiny objects on
individual frames. Although T2 is also nominally about action detection (moving
cars), it is implemented using object detection on individual frames and then
comparing the pixel coordinates of vehicles in successive frames.
% Task T5 additionally involves action detection, which operates on short video segments rather than individual frames. 

\subsection{Evaluation}
\label{sec:dumbdrone-results}

Figure~\ref{fig:baseline} presents the key performance indicators on the object
detection tasks T1 and T2. We use the well-labeled dataset to train and evaluate
Faster-RCNN with ResNet 101. We report the precision and recall at maximum F1
score.  Peak bandwidth is not shown since it is identical to average bandwidth
demand for continuous video transmission.  As shown earlier in
Figure~\ref{fig:onboard-dnn-speed}, the accuracy of this algorithm comes at the
price of very high resource demand.  This can only be met today by server-class
hardware that is available in a cloudlet.  Even on a cloudlet, the figure of 438
milliseconds of processing time per frame indicates that only a rate of two
frames per second is achievable.  Sustaining a higher frame rate will require
striping the frames across cloudlet resources, thereby increasing resource
demand considerably.  Note that the results in
Figure~\ref{fig:onboard-dnn-speed} were based on 1080p frames, while tasks T1
uses the higher resolution of 4K. This will further increase demand on cloudlet
resources.

Clearly, the strategy of blindly shipping all video to the cloudlet
and processing every frame is resource-intensive to the point of being
impractical today.  It may be acceptable as an offline processing
approach in the cloud, but is unrealistic for real-time processing on
cloudlets.  We therefore explore an approach in which a modest amount
of computation on the Tier-3 is able, with high confidence, to avoid
transmitting many video frames and thereby saving wireless bandwidth
as well as cloudlet processing resources.  This leads us to the {\xc
  EarlyDiscard} strategy of the next section.





\section{EarlyDiscard Strategy}
\label{sec:earlydiscard}

\begin{figure}
    \includegraphics[trim={0cm 13cm 14cm 0cm},clip,width=\linewidth]{FIGS/fig-early-discard.pdf}
    \caption{Early Discard on Tier-3 Devices}
    \label{fig:ondrone}
\end{figure}


\subsection{Description}
EarlyDiscard is based on the idea of using on-board processing to filter and
transmit only interesting frames in order to save bandwidth when offloading
computation. Frames are considered to be interesting if they capture objects or
events valuable for processing, for instance, survivors for a search task.
Previous work~\cite{Hu2015,Naderiparizi2017} leveraged pixel-level
features and multiple sensing modalities to select interesting frames from
hand-held or body-worn cameras. In this section, we explore the use of DNNs to
filter frames from aerial views. The benefits of using DNNs are as follows.
First, DNNs, even shallow ones, are capable of understanding some semantically
meaningful visual information. Their decisions of what to send are based on the
reasoning of image content in addition to pixel-level characteristics. Next,
DNNs are trained and specialized for each task, resulting in their high accuracy
and robustness for that particular task. Finally, compared to a sensor fusing
approach that requires other sensing modalities to be present on Tier-3 devices,
no additional hardware is added to the existing platforms.

Although smartphone-class hardware is incapable of supporting the most accurate
object detection algorithms at full frame rate today, it is typically powerful
enough to support less accurate algorithms.  These {\em weak detectors}, for
instance, MobileNet in Table~\ref{fig:onboard-dnn-speed}, are typically
designed for mobile platforms or were the state of the art just a few years ago.
In addition, they can be biased towards high recall with only modest loss of
precision. In other words, many clearly irrelevant frames can be discarded by a
weak detector, without unacceptably increasing the number of relevant frames
that are erroneously discarded.  This asymmetry is the basis of the early
discard strategy.

As shown in Figure~\ref{fig:ondrone}, we envision a choice of weak detectors
being available as early discard filters on Tier-3 devices with the specific
choice of filter being task-specific.  Based on the measurements presented in
Table~\ref{fig:onboard-dnn-speed}, we choose cheap DNNs that can run in
real-time as EarlyDiscard filters on Tier-3 devices. Note that both object
detection and image classification algorithms can yield meaningful early discard
results, as it is not necessary to know exactly where in the frame relevant
objects occur --- just an estimate of key object presence is good enough. This
suggests that MobileNet would be a good choice as a weak detector. For a given
image or partial of an image, it can predict whether the input contains objects
of interests. More importantly, MobileNet's speed of 13 ms per frame on the
Tier-3 platform Jetson yields more than 75 fps. We therefore use MobileNet for
early discard in our experiments.

Pre-trained classifiers for MobileNet are available today for generic objects
such as cars, animals, human faces, human bodies, watercraft, and so on.
However, these DNN classifiers have typically been trained on images that were
captured from a human perspective --- often by a camera held or worn by a
person. These images typically have the objects at the center of the image and
occupy the majority of the image. Many Tier-3 devices, however, capture images
from different viewpoints (e.g. aerial views) and need to recognize rare
task-specific objects different from generic categories. To improve the
classification accuracy for custom objects from different viewpoints, we used
    {\em transfer learning}~\cite{Yosinski2014} to finetune the pre-trained
classifiers on small training sets of images that were captured from correct
viewpoint. The process of fine-tuning involves initial re-training of the last
DNN layer, followed by re-training of the entire network until convergence.
Transfer learning enables accuracy to be improved significantly for custom
objects without incurring the full cost of creating a large training set.

% captured
% from an aerial viewpoint.

% Finetuning not only allows us to adapt pre-trained classifiers to
% drone views, but also makes it possible to target custom objects of interests
% that are not in the original dataset, for instance, survivors in orange life
% jacket. 


\begin{figure}
    \centering
    \includegraphics[width=0.8\linewidth]{FIGS/fig-training.pdf}
    \caption{Tiling and DNN Fine Tuning}
    \label{fig:tiling}
\end{figure}

For live drone video analytics, images are typically captured from a significant
height, and hence objects in such an image are small.  This interacts negatively
with the design of many DNNs, which first transform an input image to a fixed
low resolution --- for example, 224x224 pixels in MobileNet. Many important but
small objects in the original image become less recognizable.  It has been shown
that small object size correlates with poor accuracy in DNNs~\cite{Huang2017}.
To address this problem, we {\em tile} high resolution frames into multiple
sub-frames and then perform recognition on the sub-frames as a batch.  This is
done offline for training, as shown in Figure~\ref{fig:tiling}, and also for
online inference on the drone and on the cloudlet.  The lowering of resolution
of a sub-frame by a DNN is less harmful, since the scaling factor is smaller.
Objects are represented by many more pixels in a transformed sub-frame than if
the entire frame had been transformed. The price paid for tiling is increased
computational demand.  For example, tiling a frame into four sub-frames results
in four times the classification workload. Note that this increase in workload
typically does not translates into the same increase in inference time, as
workloads can be batched together to leverage hardware parallelism for a reduced
total inference time.

\begin{figure}
    \centering
    \includegraphics[width=.8\linewidth]{FIGS/fig-tile-resolution-speed-accuracy.pdf}
    \caption{Speed-Accuracy Trade-off of Tiling}
    \label{fig:earlydiscard-tile-accuracy-speed}
\end{figure}


\subsection{Experimental Setup}

Our experiments on the EarlyDiscard strategy used the same benchmark suite
described in Section~\ref{sec:dumbdrone-setup}. We used Jetson TX2 as the Tier-3
device platform. We run MobileNet filters to get predictions on whether
sub-frames contain objects of interests. We compare the predictions with ground
truths (e.g. whether a sub-frame is indeed interesting) to evaluate the
effectiveness of EarlyDiscard. Both frame-based and event-based metrics are used
in the evaluation.

\begin{figure}
    \centering
    \includegraphics[width=\linewidth]{FIGS/fig-event-recall-frame-percentage-legend.pdf}\\
    \vspace{.5in}
    \begin{minipage}[]{0.45\linewidth}
        \centering
        \includegraphics[width=\linewidth]{FIGS/fig-event-recall-frame-percentage-vs-threshold-okutama.pdf}\\
        {(a) T1}
    \end{minipage}
    \begin{minipage}[]{0.45\linewidth}
        \centering
        \includegraphics[width=\linewidth]{FIGS/fig-event-recall-frame-percentage-vs-threshold-stanford.pdf}\\
        {(b) T2}
    \end{minipage}

    \vspace{.5in}

    \begin{minipage}[]{0.45\linewidth}
        \centering
        \includegraphics[width=\linewidth]{FIGS/fig-event-recall-frame-percentage-vs-threshold-raft.pdf}\\
        {(c) T3}
    \end{minipage}
    \begin{minipage}[]{0.45\linewidth}
        \centering
        \includegraphics[width=\linewidth]{FIGS/fig-event-recall-frame-percentage-vs-threshold-elephant.pdf}\\
        {(c) T4}
    \end{minipage}

    \vspace{.5in}
    \caption{Bandwidth Breakdown}
    \label{fig:earlydiscard-frame-percent-breakdown}
\end{figure}

\subsection{Evaluation}
\label{sec:earlydiscard-result}

EarlyDiscard is able to significantly reduce the bandwidth consumed while
maintaining high result accuracy and low average delay. For three out of four
tasks, the average bandwidth is reduced by a factor of ten. Below we present
our results in detail.

\subsubsection{Effects of Tiling}
% \noindent{\textbf{Effects of Tiling}}: 
Tiling is used to improve the accuracy
for high resolution aerial images. We used the Okutama Action Dataset, whose
attributes are shown in row T1 of Table~\ref{fig:benchmarksuite}, to explore
the effects of tiling.  For this dataset,
Figure~\ref{fig:earlydiscard-tile-accuracy-speed} shows how speed and accuracy
change with tile size.  Accuracy improves as tiles become smaller, but the
sustainable frame rate drops.  We group all tiles from the same frame in a
single batch to leverage parallelism, so the processing does not change linearly
with the number of tiles. The choice of an operating point will need to strike a
balance between the speed and accuracy.  In the rest of the chapter, we use two
tiles per frame by default.

\subsubsection{EarlyDiscard Filter Accuracy}

The output of a Tier-3 filter is the probability of the current tile being
``interesting.''  A tunable {\em cutoff threshold} parameter specifies the
threshold for transmission to the cloudlet. All tiles, whether deemed
interesting or not, are still stored in the Tier-3 storage for offline processing.

Since objects have temporal locality in videos, we define an event (of an
object) in a video to be consecutive frames containing the same object of
interests. For example, the appearance of the same red raft in T3 in consecutive
45 frames constitutes a single event. A correct detection of an event is defined
as at least one of the consecutive frames being transmitted to the cloudlet.

Figure~\ref{fig:earlydiscard-frame-percent-breakdown} shows our results on all
four tasks. Blue lines show how the event recalls of EarlyDiscard filters for different
tasks change as a function of cutoff threshold. The MobileNet DNN filter we used
is able to detect all the events for T1 and T4 even at a high cutoff threshold.
For T2 and T3, the majority of the events are detected. Achieving high recall on
T2 and T3 (on the order of 0.95 or better) requires setting a low cutoff
threshold.  This leads to the possibility that many of the transmitted frames
are actually uninteresting (i.e., false positives).

\subsubsection{False negatives}
As discussed earlier, false negatives are
a source of concern with early discard.  Once the Tier-3 device drops a frame
containing an important event, improved cloudlet processing cannot help. The
results in the third column of Table~\ref{fig:early-discard-results} confirm
that there are no false negatives for T1 and T4 at a cutoff threshold of 0.5.
For T2 and T3, lower cutoff thresholds are needed to achieve perfect recalls.

\subsubsection{Result latency}
The contribution of early discard processing to total result latency
is calculated as the average time difference between the first frame
in which an object occurs (i.e., first occurrence in ground truth) and
the first frame containing the object that is transmitted to the
backend (i.e., first detection).  The results in the fourth column of
Table~\ref{fig:early-discard-results} confirm that early discard
contributes little to result latency.  The amounts range from 0.1~s
for T1 to 12.7~s for T3.

% At the timescale of human actions
% such as dispatching of a rescue team, these are negligible delays.

\begin{table}
    \centering
    \begin{tabular}{|c|c|c|c|c|c|c|}
        \hline
           & Task Total Events & Detected Events  & Avg Delay      & Total Data     & Avg B/W & Peak B/W       \\
           &                   &                  & (s)            & (MB)           & (Mbps)  & (Mbps)         \\

        \hline
        T1 & \phantom{0}62     & 100~\%           & \phantom{0}0.1 & \phantom{0}441 & 5.10    & 10.7           \\
        \hline
        T2 & \phantom{0}11     & \phantom{0}73~\% & \phantom{0}4.9 & \phantom{00}13 & 0.03    & \phantom{0}7.0 \\ % 100% recall at 47% of frames, 82% recall at 21% of frames
        \hline
        T3 & \phantom{0}31     & \phantom{0}90~\% & 12.7           & \phantom{00}93 & 0.24    & \phantom{0}7.0 \\ % 100% recall at 9% frames
        \hline
        T4 & \phantom{0}25     & 100~\%           & \phantom{0}0.3 & \phantom{0}167 & 0.43    & \phantom{0}7.0 \\
        \hline
    \end{tabular}\\
    \caption{Recall, Event Latency and Bandwidth at Cutoff Threshold 0.5}
    \label{fig:early-discard-results}
\end{table}


\subsubsection{Bandwidth}
Columns 5--7 of Table~\ref{fig:early-discard-results} pertain to wireless
bandwidth demand for the benchmark suite with early discard.  The figures shown
are based on H.264 encoding of each individual frames in the video transmission.
Average bandwidth is calculated as the total data transmitted divided by mission
duration.  Comparing column 5 of Table~\ref{fig:early-discard-results} with
column 2 of Table~\ref{fig:baseline}, we see that all videos in the benchmark
suite are benefited by early discard (Note T3 and T4 have the same test dataset
as T2). For T2, T3, and T4, the bandwidth is reduced by more than 10x. The
amount of benefit is greatest for rare events (T2 and T3).  When events are
rare, the Tier-3 device can drop many frames.

Figure~\ref{fig:earlydiscard-frame-percent-breakdown} provides deeper insight
into the effectiveness of cutoff-threshold on event recall. It also shows how
many true positives (violet) and false positives (aqua) are
transmitted. Ideally, the aqua section should be zero.  However for T2, most
frames transmitted are false positives, indicating the early discard filter has
low precision.  The other tasks exhibit far fewer false positives.  This
suggests that the opportunity exists for significant bandwidth savings if
precision could be further improved, without hurting recall.

\subsection{Use of Sampling}

\begin{figure}
    \centering
    \includegraphics[width=.7\linewidth]{FIGS/fig-random-select-interval-recall-hatch.pdf}
    \caption{Event Recall at Different Sampling Intervals}
    \label{fig:sampling-only}
\end{figure}


\begin{figure}
    \centering
    \includegraphics[width=0.7\linewidth]{FIGS/fig-recall-frame-aggregated-legend.pdf}

    \begin{minipage}[]{0.47\linewidth}
        \centering
        \includegraphics[trim={0.5cm 0.5cm 0 0},clip,width=\linewidth]{FIGS/fig-random-select-and-filter-recall-frame-okutama-aggregated.pdf}\\
        {(a) T1}
    \end{minipage}
    \begin{minipage}[]{0.47\linewidth}
        \centering
        \includegraphics[trim={0.5cm 0.5cm 0 0},clip,width=\linewidth]{FIGS/fig-random-select-and-filter-recall-frame-stanford-aggregated.pdf}
        {(b) T2}
    \end{minipage}
    \begin{minipage}[]{0.47\linewidth}
        \centering
        \includegraphics[trim={0.5cm 0.5cm 0 0},clip,width=\linewidth]{FIGS/fig-random-select-and-filter-recall-frame-raft-aggregated.pdf}
        {(c) T3}
    \end{minipage}
    \begin{minipage}[]{0.47\linewidth}
        \centering
        \includegraphics[trim={0.5cm 0.5cm 0 0},clip,width=\linewidth]{FIGS/fig-random-select-and-filter-recall-frame-elephant-aggregated.pdf}
        {(c) T4}
    \end{minipage}
    \caption{Sample with Early Discard. Note the log scale on y-axis.}
    \label{fig:sampling-discard}
\end{figure}

\begin{table}
    \centering
    \begin{tabular}{|c|c|c|c|c|}
        \hline
        JPEG Frame Sequence & H264 High Quality & H264 Medium Quality & H264 Low Quality \\
        (MB)                & (MB)              & (MB)                & (MB)             \\
        \hline
        5823                & 3549              & 1833                & 147              \\
        \hline
    \end{tabular}\\
    \vspace{0.1in}
    \begin{captiontext}
        H264 high quality uses Constant Rate Factor (CRF) 23. Medium
        uses CRF 28 and low uses CRF 40~\cite{Merritt2007}.
    \end{captiontext}
    \caption{Test Dataset Size With Different Encoding Settings}
    \label{fig:video-vs-images}
\end{table}

Given the relatively low precision of the weak detectors, a significant number
of false positives are transmitted.  Furthermore, the occurrence of an object will
likely last through many frames, so true positives are also often redundant for
simple detection tasks.  Both of these result in excessive
consumption of precious bandwidth.
This suggests that simply restricting the number of transmitted
frames by sampling may help reduce bandwidth consumption.

Figure~\ref{fig:sampling-only} shows the effects of
sending a sample of frames from Tier-3, without any
content-based filtering.  Based on these results, we can reduce
the frames sent as little as one per second and still get
adequate recall at the cloudlet.  Note that this result is very
sensitive to the actual duration of the events in the videos.
For the detection tasks outlined here, most of the events (e.g.,
presences of a particular elephant) last for many seconds (100's
of frames), so such sparse sampling does not hurt recall.
However, if the events were of short duration, e.g., just a few
frames long, then this method would be less effective, as
sampling may lead to many missed events (false negatives).

Can we use content-based filtering along with sampling to further
reduce bandwidth consumption?  Figure~\ref{fig:sampling-discard}
shows results when running early discard on a sample of the
frames. This shows that for the same recall, we can reduce the
bandwidth consumed by another factor of 5 on average over sampling alone.
This effective combination can reduce the average bandwidth
consumed for our test videos to just a few hundred kilobits
per second.  Furthermore, more processing time is available per
processed frame, allowing more sophisticated algorithms to be
employed, or to allow smaller tiles to be used, improving
accuracy of early discard.

One case where sampling is not an effective solution is when all frames
containing an object need to be sent to the cloudlet for some form of activity
or behavior analysis from a complete video sequence.  In this case, bandwidth
will not reduce much, as all frames in the event sequence must be sent.
However, the processing time benefits of sampling may still be exploited,
provided all frames in a sample interval are transmitted on a match.


\subsection{Effects of Video Encoding}

One advantage of the Dumb strategy is that since all
frames are transmitted, one can use a modern video encoding to
reduce transmission bandwidth.  With early discard, only a subset
of disparate frames are sent.  These will likely need to be
individually compressed images, rather than a video stream.  How
much does the switch from video to individual frames affect
bandwidth?

In theory, this can be a significant impact. Video encoders leverage the
similarity between consecutive frames, and model motion to efficiently encode
the information across a set of frames. Image compression can only exploit
similarity within a frame, and cannot efficiently reduce number of bits needed
to encode redundant content across frames. To evaluate this difference, we start
with extracted JPEG frame sequences of our video data set. We encode the frame
sequence with different H.264 settings. Table~\ref{fig:video-vs-images}
compares the size of frame sequences in JPEG and the encoded video file sizes.
We see only about 3x difference in the data size for the medium quality. We can
increase the compression (at the expense of quality) very easily, and are able
to reduce the video data rate by another order of magnitude before quality
degrades catastrophically.

However, this compression does affect analytics. Even at medium quality level,
visible compression artifacts, blurring, and motion distortions begin to appear.
Initial experiments analyzing compressed videos show that these distortions do
have a negative impact on accuracy of analytics. Using average precision
analysis, a standard method to evaluate accuracy, we estimate that the
most accurate model (Faster-RCNN ResNet101) on low quality videos performs similarly
to the less accurate model (Faster-RCNN InceptionV2) on high quality
JPEG images. This negates the benefits of using the state-of-art models.

In our EarlyDiscard design, we pay a penalty of sending frames instead of a
compressed low quality video stream. This overhead (approximately 30x) is
compensated by the 100x reduction in frames transmitted due to sampling with
early discard. In addition, the selective frame transmission preserves the
accuracy of the state-of-art detection techniques.

Finally, one other option is to treat the set of disparate frames as a sequence
and employ video encoding at high quality. This can ultimately eliminate the per
frame overhead while maintaining accuracy. However, this will require a complex setup with
both low-latency encoders and decoders, which can generate output data
corresponding to a frame as soon as input data is ingested, with no buffering,
and can wait arbitrarily long for additional frame data to arrive.

For the experiments in the rest of this chapter, we only account for the
fraction of frames transmitted, rather than the choice of specific encoding
methods used for those frames.
\section{{\xc Just-in-time-Learning} Strategy To Improve Early Discard}
\label{sec:jitl}

\begin{figure}
    \centering
    \includegraphics[trim={0 1.8cm 0 0},clip,width=0.7\linewidth]{FIGS/fig-jitl-legend.pdf}\\
    \vspace{.5in}
    \begin{subfigure}[b]{.48\linewidth}
    \centering
    \includegraphics[width=\linewidth]{FIGS/fig-jitl-okutama-eventrecall-step.pdf}
    \caption{T1}
    \end{subfigure}
    \begin{subfigure}[b]{.48\linewidth}
    \centering
    \includegraphics[width=\linewidth]{FIGS/fig-jitl-stanford-eventrecall-step.pdf}
    \caption{T2}
    \end{subfigure}

    \vspace{.5in}

    \begin{subfigure}[b]{.48\linewidth}
    \centering
        \includegraphics[width=\linewidth]{FIGS/fig-jitl-raft-eventrecall-step.pdf}
    \caption{T3}
    \end{subfigure}
    \begin{subfigure}[b]{.48\linewidth}
    \centering
        \includegraphics[width=\linewidth]{FIGS/fig-jitl-elephant-eventrecall-step.pdf}
    \caption{T4}
    \end{subfigure}

    \vspace{.5in}
\caption{JITL Fraction of Frames under Different Event Recall}
\label{fig:jitl-eventrecall}
\end{figure}

While EarlyDiscard filters are customized and optimized for specific tasks (e.g.
detecting a human with red life jacket), we observe that EarlyDiscard filters do
not leverage context information within a specific video stream. Opportunities
exist if we could further specialize the computer vision processing to the
characteristics of video streams.

We propose Just-in-time-learning  (JITL), which tunes the Tier-3 processing
pipeline to the characteristics of the current mission in order to reduce
transmitted false positives from the Tier-3 device, and therefore reduce wasted
bandwidth.  Intuitively, JITL leverages temporal locality in video streams to
quickly adapt processing outcomes based on recent feedback. 

It is inspired by the ideas of cascade architecture from the computer vision
community~\cite{Viola2001}, but is different in construction. A JITL filter is a
cheap cascade filter that distinguishes between the EarlyDiscard DNN's
\emph{true positives} (frames that are actually interesting) and \emph{false
positives} (frames that are wrongly considered interesting). Specifically, when
a frame is reported as positive by EarlyDiscard, it is then passed through a
JITL filter. If the JITL filter reports negative, the frame is regarded as a
false positive and will not be sent. Ideally, all \emph{true positives} from
EarlyDiscard are marked \emph{positive} by the JITL filter, and all \emph{false
positives} from EarlyDiscard are marked \emph{negative}.  Frames dropped by
EarlyDiscard are not processed by the JITL filter, so this approach can only
serve to improve precision, but not recall.


Periodically during a drone mission, a JITL filter is trained on the cloudlet
using the frames transmitted from the drone.  The frames received on the
cloudlet are predicted positive by the EarlyDiscard filter. The cloudlet, with
more processing power, is able to run more accurate DNNs to identify true
positives and false positives. Using this information as a feedback on how well
current Tier-3 processing pipeline is doing, a small and lightweight JITL filter
is trained to distinguish true positives and false positives of EarlyDiscard
filters. These JITL filters are then pushed to the drone to run as a cascade
filter after the EarlyDiscard DNN.

True/false positive frames have high temporal locality throughout a drone
mission. The JITL filter is expected to pick up the features that confused the
EarlyDiscard DNN in the immediate past and improve the pipeline's accuracy in
the near future. These features are usually specific to the current flight, and
may be affected by terrain, shades, object colors, and particular shapes or
background textures.

JITL can be used with EarlyDiscard DNNs of different cutoff probabilities to
strike different trade-offs. In a bandwidth-favored setting, JITL can work with
an aggressively selective EarlyDiscard DNN to further reduce wasted bandwidth. In
a recall-favored setting, JITL can be used with a lower-cutoff DNN to preserve
recall.

In our implementation, we use a linear support vector machine
(SVM)~\cite{Friedman2001} as the JITL filter. Linear SVM has several advantages:
1) short training time in the order of seconds; 2) fast inference; 3) only
requires a few training examples; 3) small in size to transmit, usually on the
order of 50KB in our experiments. The input features to the JITL SVM filter are
the image features extracted by the EarlyDiscard DNN filter. In our case, since
we are using MobileNet as our EarlyDiscard filter, they are the 1024-dimensional
vector elements from the second last layer of MobileNet. This vector, also
called ``bottleneck values'' or ``transfer values'' captures high-level features
that represents the content of an image. Note that the availability of such
image feature vector is not tied to a particular image classification DNN nor
unique to MobileNet. Most image classification DNNs can be used as a feature
extractor in this way.

\subsection{JITL Experimental Setup}
We used Jetson TX2 as our Tier-3 device platform and evaluated the JITL strategy
on four tasks, T1 to T4. For the test videos in each task, we began with the
EarlyDiscard filter alone and gradually trained and deployed JITL filters.
Specifically, every ten seconds, we trained an SVM using the frames transmitted
from the drone and the ground-truth labels for these frames. In a real
deployment, the frames would be marked as true positives or false positives by
an accurate DNN running on the cloudlet since ground-truth labels are not
available. In our experiments, we used ground-truth labels to control variables
and remove the effect of imperfect prediction of DNN models running on the
cloudlet. 

In addition, we used the true and false positives from all previous intervals,
not just the last ten seconds when training the SVM. The SVM, once trained, is
used as a cascade filter running after the EarlyDiscard filter on the drone to
predict whether the output of the EarlyDiscard filter is correct or not. For a
frame, if the EarlyDiscard filter predicts it to be interesting, but the JITL
filter predicts the EarlyDiscard filter is wrong, it would not be transmitted to
the cloudlet. In other words, following two criteria need to be satisfied for a
frame to be transmitted to the cloudlet: 1) EarlyDiscard filter predicts it to
be interesting 2) JITL filter predicts the EarlyDiscard filter is correct on
this frame.

\subsection{JITL Results}

From our experiments, JITL is able to filter out more than 15\% of remaining
frames after EarlyDiscard without loss of event recall for three of four tasks.
Figure~\ref{fig:jitl-eventrecall} details the fraction of frames saved by JITL.
The x-axis presents event recall. Y-axis represents the fraction of total
frames. The blue region presents the achievable fraction of frames by
EarlyDiscard. The orange region shows the additional savings using JITL. For T1,
T3, and T4, at the highest event recall, JITL filters out more than 15\% of
remaining frames. This shows that JITL is effective at reducing the false
positives thus improving the precision of the drone filter. However,
occasionally, JITL predicts wrongly and removes true positives. For example, for
T2, JITL does not achieve a perfect event recall. This is due to shorter event
duration in T2, which results in fewer positive training examples to learn
from. Depending on tasks, getting enough positive training examples for JITL
could be difficult, especially when events are short or occurrences are few. To
overcome this problem in practice, techniques such as synthetic data
generation~\cite{Dwibedi2017} could be explored to synthesize true positives
from the background of the current flight.

\section{Applying EarlyDiscard and JITL to Wearable Cognitive Assistants}
\label{bw:wca}

While the experiments in previous sections
(~\ref{sec:earlydiscard}~\ref{sec:jitl}) are performed in a drone video
analytics context, EarlyDiscard and JITL approaches can be applied more
generally to live video analytics offloading from Tier-3 devices to Tier-2 edge
data-centers. In this section, we use the LEGO
application~\cite{chen2018application} to showcase how to apply these bandwidth
saving approaches to WCAs.

\begin{figure}
    \centering
    \begin{minipage}[]{0.45\linewidth}
        \centering
        \includegraphics[width=\linewidth]{FIGS/lego-search}\\
        {(a) Searching for Lego Blocks}
    \end{minipage}
    \begin{minipage}[]{0.45\linewidth}
        \centering
        \includegraphics[width=\linewidth]{FIGS/lego-assembled}\\
        {(b) Assembling Lego Pieces}
    \end{minipage}
    \caption{Example Images from a Lego Assembly Video}
    \label{fig:wca-lego-example-images}
\end{figure}

The LEGO wearable cognitive assistant helps a user put together a specific Lego
pattern by providing step-by-step audiovisual instructions. The application
works as follows. The assistant first prompts a user an animated image showing
the Lego block to use and asks the user to put it on the Lego board or assemble
it with previous pieces. Following the guidance, the user searches for the
particular Lego block, assemble it, and put the assembled piece on the Lego
board for the next instruction. Figure~\ref{fig:wca-lego-example-images} shows
the first-person view images captured from the wearable device during this
process. The assistant analyzes the assembled Lego piece on the Lego board by
identifying its shape and color using computer vision and provides the suitable
instruction.

\begin{figure}
    \centering
    \begin{minipage}[]{0.31\linewidth}
        \centering
        \includegraphics[width=\linewidth]{FIGS/lego-dataset-1}\\
    \end{minipage}
    \begin{minipage}[]{0.31\linewidth}
        \centering
        \includegraphics[width=\linewidth]{FIGS/lego-dataset-2}\\
    \end{minipage}
    \begin{minipage}[]{0.31\linewidth}
        \centering
        \includegraphics[width=\linewidth]{FIGS/lego-dataset-3}\\
    \end{minipage}
    \caption{Example Images from LEGO Dataset}
    \label{fig:wca-lego-dataset}
\end{figure}

Intuitively, to the assistant, frames capturing the assembled piece on the Lego
board, (for example Figure~\ref{fig:wca-lego-example-images} (b)) are the
crucial frames to process, as they reflect the user's working progress.
Figure~\ref{fig:wca-lego-example-images} (a), on the other hand, is less
interesting as it does not contain information on user progress. If some cheap
processing on the wearable device could distinguish
(a) from (b), bandwidth consumption can be
reduced as we can discard Figure~\ref{fig:wca-lego-example-images} (a) early on
the wearable device without transmitting the frame to the cloudlet for processing.
This provides opportunities to apply EarlyDiscard and JITL.

We collect a LEGO dataset of twelve videos, in which users assemble Lego pieces
in three environments with different background, lighting, and viewpoints.
Figure~\ref{fig:wca-lego-dataset} shows example images from the dataset. We run
the LEGO WCA on these videos to get pseudo ground truth labels. Specifically,
for each frame, based on the outputs of the LEGO WCA vision processing, we
categorize the frame to be either ``interesting'' or ``not interesting''. A
frame is considered to be interesting if a LEGO board is found in the frame,
otherwise considered not interesting.

\begin{figure}
    \centering
    \includegraphics[width=.6\linewidth]{FIGS/earlydiscard-cm}\\
    \caption{EarlyDiscard Filter Confusion Matrix}
    \label{fig:wca-early-discard}
\end{figure}

We use this dataset to finetune a MobileNet DNN in order to automatically
distinguish interesting frames from the boring ones for EarlyDiscard. For each
of the three environments, we randomly select two videos for training, one video
for validation, and one video for testing. We randomly sample 2000 interesting
images and 2000 boring images from the six training videos as the training
data. Similarly, we random sample 200 interesting images and 200 boring images
from the three validation videos as the validation data. We implement
MobileNet transfer learning using the PyTorch
framework~\cite{paszke2019pytorch}. We train the model for 20 epochs and
select the model weights that give the highest accuracy on the validation set as
the model for inference.

Our test sets have in total 14725 frames. Figure~\ref{fig:wca-early-discard}
shows the confusion matrix of our trained EarlyDiscard classifier. X-axis
represents the predicted results: ``Transmit'' means the frame is predicted to
be interesting and should be transmitted to cloudlet for processing while
``Discard'' means the frame is predicted to be boring and should not be
transmitted. Similarly, Y-axis represents the ground truth results. As we can
see, the classifier correctly predicts 2260 out of 14725 frames to be
interesting and correctly suppresses 11971 frames. With EarlyDiscard in place,
only 19\% of all the frames are transmitted. Meanwhile, the false negative is 0
frame, meaning no ``interesting'' frame is wrongly discarded. This is the result
of biasing the classifier towards recall instead of precision.

\begin{figure}
    \centering
    \begin{minipage}[b]{.45\linewidth}
        \centering
        \includegraphics[width=\linewidth]{FIGS/jitl-earlydiscard-cm}\\
        {(a) EarlyDiscard}
    \end{minipage}
    \begin{minipage}[b]{.45\linewidth}
        \centering
        \includegraphics[width=\linewidth]{FIGS/jitl-combined-cm}\\
        {(b) EarlyDiscard + JITL}
    \end{minipage}
    \caption{JITL Confusion Matrix}
    \label{fig:wca-jitl}
\end{figure}

Among all the frames that are transmitted, 18\% of them are false positives.
These 494 false positives suggest that there are room to improve using JITL. For
each of the test videos, we use the first half of the video as training examples
for JITL to train a SVM that produces a confidence score for EarlyDiscard
prediction. Figure~\ref{fig:wca-jitl} compares the confusion matrix of using
EarlyDiscard alone with EarlyDiscard + JITL. As we can see, JITL reduces 13\% of
the false positives at the cost of 2 false negatives. Note that these 2 false
negative frames do not result in missing instructions as adjacent interesting
frames are still transmitted.

% To reduce bandwidth consumption with EarlyDiscard and JITL, we first need to
% identify what frames should be considered interesting.

% the cheap computer vision processing that can identify ``interesting'' frames on
% Tier-3 devices. Figure~\ref{fig:wca-lego-example-images} provides intuitions on how to
% apply EarlyDiscard and JITL to the Lego application
\section{Related Work}
\label{bw:relatedwork}

In the context of drone video analytics, Wang et al.~\cite{Wang2017networked}
shares our concern for wireless bandwidth, but focuses on coordinating a network
of drones to capture and broadcast live sport event. In addition, Wang et
al~\cite{Wang2016skyeyes} explored adaptive video streaming with drones using
content-based compression and video rate adaptation. While we share their
inspiration, our work leverages characteristics of DNNs to enable
mission-specific optimization strategies.

Much previous work on static camera networks explored efficient use of compute
and network resources at scale. Zhang et al.~\cite{zhang2017live} studied
resource-quality trade-off under result latency constraints in video analytics
systems. Kang et al.~\cite{kang2017noscope} worked on optimizing DNN queries
over videos at scale. While they focus on supporting a large number of computer
vision workload, our work optimizes for the first hop wireless bandwidth. In
addition, Zhang et al.~\cite{zhang2015design} designed a wireless distributed
surveillance system that supports a large geographical area through frame
selection and content-aware traffic scheduling. In contrast, our work does not
assume static cameras. We explore techniques that tolerate changing scenes in
video feeds and strategies that work for moving cameras.

Some previous work on computer vision in mobile settings has relevance to
aspects of our system design.  Chen et al.~\cite{chen2015glimpse} explore how
continuous real-time object recognition can be done on mobile devices. They meet
their design goals by combining expensive object detection with computationally
cheap object tracking.  Although we do not use object tracking in our work, we
share the resource concerns that motivate that work.  Naderiparizi et
al.~\cite{naderiparizi2017glimpse} describe a programmable early-discard camera
architecture for continuous mobile vision.  Our work shares their emphasis on
early discard, but differs in all other aspects.  In fact, our work can be
viewed as complementing that work: their programmable early-discard camera would
be an excellent choice for Tier-3 devices. Lastly, Hu et al~\cite{Hu2015} have
investigated the approach of using lightweight computation on a mobile device to
improve the overall bandwidth efficiency of a computer vision pipeline that
offloads computation to the edge.  We share their concern for wireless
bandwidth, and their use of early discard using inexpensive algorithms on the
mobile device.

\section{Chapter Summary and Discussion}
\label{bw:discussion}

In this chapter, we address the bandwidth challenge of running many WCAs at
scale. We propose two application-agnostic methods to reduce bandwidth
consumption when offloading computation to edge servers.

The EarlyDiscard technique employs on-board filters to select interesting frames
and suppress the transmission of mundane frames to save bandwidth. In
particular, cheap yet effective DNN filters are trained offline to fully
leverage the large quantity of training data and the high learning capacities of
DNNs. Building on top of EarlyDiscard, JITL adapts an EarlyDiscard filter to a
specific environment online. While a WCA is running, JITL continuously evaluates
the EarlyDiscard filter and reduces the number of false positives by predicting
whether an EaryDiscard decision is made correctly. These two techniques together
reduce the total number of unnecessary frames transmitted.

We evaluate these two strategies first in the drone live video analytics context
for search tasks in domains such as search-and-rescue, surveillance, and
wildlife conservation, and then for WCAs. Our experimental results show that
this judicious combination of Tier-3 processing and edge-based processing can
save substantial wireless bandwidth and thus improve scalability, without
compromising result accuracy or result latency.
\chapter{Application-Aware Techniques to Reduce Offered Load}
\label{chapter: load}

{\em Elasticity} is a key attribute of cloud computing.  When load
rises, new servers can be rapidly spun up.  When load subsides, idle
servers can be quiesced to save energy.  Elasticity is vital to
scalability, because it ensures acceptable response times under a wide
range of operating conditions.  To benefit, cloud services need to be
architected to easily scale out to more servers.  Such a design is
said to be ``cloud-native.''

In contrast, edge computing has limited elasticity.  As its name implies, a
cloudlet is designed for much smaller physical space and electrical power than a
cloud data center.  Hence, the sudden arrival of an unexpected flash crowd can
overwhelm a cloudlet.  Since low end-to-end latency is a prime reason for edge
computing, shifting load elsewhere (e.g., the cloud) is not an attractive
solution.  {\em How do we build multi-user edge computing systems that preserve
low latency even as load increases?}  That is the focus of the next two chapters.

Our approach to scalability is driven by the following observation. Since
compute resources at the edge cannot be increased on demand, the only paths to
scalability are (a) to reduce offered load, as discussed in this chapter,
or (b) to reduce queueing delays through improved end-to-end scheduling,
as discussed in Chapter~\ref{chapter: cloudlet}.  Otherwise, the mismatch between
resource availability and offered load will lead to increased queueing delays
and hence increased end-to-end latency.  Both paths require the average burden
placed by each user on the cloudlet to fall as the number of users increases.
This, in turn, implies {\em adaptive application behavior} based on guidance
received from the cloudlet or inferred by the user's mobile device.  In the
context of Figure~\ref{fig:3tier}, scalability at the left is achieved very
differently from scalability at the right.  The relationship between Tier-3 and
Tier-2 is {\em non-workload-conserving}, while that between Tier-1 and other
tiers is workload-conserving.

While we demonstrated application-agnostic techniques to reduce network
transmission between Tier-3 and Tier-2 in Chapter~\ref{chapter: bandwidth},
offered load can be further reduced with application assistance. We claim that
scalability at the edge can be better achieved for applications that have been
designed with this goal in mind.  We refer to applications that are specifically
written to leverage edge infrastructure as {\em edge-native applications.} These
applications are deeply dependent on the services that are only available at the
edge (such as low-latency offloading of compute, or real-time access to video
streams from edge-located cameras), and are written to adapt to
scalability-relevant guidance.  For example, an application at Tier-3 may be
written to offload object recognition in a video frame to Tier-2, but it may
also be prepared for the return code to indicate that a less accurate (and hence
less compute-intensive) algorithm than normal was used because Tier-2 is heavily
loaded.  Alternatively, Tier-2 or Tier-3 may determine that the wireless channel
is congested; based on this guidance, Tier-3 may reduce offered load by
preprocessing a video frame and using the result to decide whether it is
worthwhile to offload further processing of that frame to the cloudlet.  Several
earlier work~\cite{Hu2015}~\cite{christensen2019towards} have shown that even
modest computation, such as color filtering and shallow DNN processing, at Tier-3
can make surprisingly good predictions about whether a specific use of Tier-2 is
likely to be worthwhile.

Edge-native applications may also use {\em cross-layer adaptation
  strategies,} by which knowledge from Tier-3 or Tier-2 is used in the
management of the wireless channel between them.  For example, an
assistive augmented reality (AR) application that verbally guides a
visually-impaired person may be competing for the wireless channel and
cloudlet resources with a group of AR gamers.  In an overload
situation, one may wish to favor the assistive application over the
gamers.  This knowledge can be used by the cloudlet operating system
to preferentially schedule the more important workload.  It can also
be used for prioritizing network traffic by using {\em fine-grain
  network slicing,} as envisioned in 5G~\cite{Contreras2018}.

Wearable cognitive assistance, perceived to be ``killer apps'' for edge
computing, are perfect exemplars of edge-native applications. In the rest of
this chapter, we showcase how we can leverage unique application characteristics
of WCAs to adapt application behavior and reduce offered load. Our work is built
on the Gabriel platform~\cite{ha2014towards,chen2017empirical}, shown in
Figure~\ref{fig:gabriel}. The Gabriel front-end on a wearable device performs
preprocessing of sensor data (e.g., compression and encoding), which it streams
over a wireless network to a cloudlet. We refer to the Gabriel platform with new
mechanisms that handle multitenancy, perform resource allocation, and support
application-aware adaptation as ``Scalable Gabriel'' and the single-user
baseline platform as ``Original Gabriel''.

% The Gabriel back-end on the cloudlet has
% a modular structure. The {\em control module} is the focal point for all
% interactions with the wearable device.  A publish-subscribe (PubSub) mechanism
% decodes and distributes the incoming sensor streams to multiple {\em cognitive
% modules} (e.g., task-specific computer vision algorithms) for concurrent
% processing. Cognitive module outputs are integrated by a task-specific {\em user
% guidance module} that performs higher-level cognitive processing such as
% inferring task state, detecting errors, and generating guidance in one or more
% modalities (e.g., audio, video, text, etc.).

% It did,
% however, have a token-based transmission mechanism.  This limited a client to
% only a small number of outstanding operations, thereby offering a simple form of
% rate adaptation to processing or network bottlenecks.  We have retained this
% token mechanism in our system. In addition, 


% Since the techniques for reducing offered workload are application-specific, we
% focus on a specific class of edge-native applications to validate our ideas.
% Our choice is a class of applications called {\em Wearable Cognitive Assistance}
% (WCA) applications~\cite{ha2014towards}.  They are perceived to be ``killer apps'' for
% edge computing because (a) they transmit large volumes of video data to the
% cloudlet; (b) they have stringent end-to-end latency requirements; and (c) they
% make substantial compute demands of the cloudlet, often requiring high-end GPUs.

% We leverage unique characteristics of WCA applications to reduce
% offered load through graceful degradation and improved resource
% allocation.  Our contributions are as follows:
% \begin{smitemize}
%   \item{An architectural framework for WCA that enables graceful degradation under heavy load.}
%   \item{An adaptation taxonomy of WCA applications, and techniques for workload reduction.}
%   \item{A cloudlet resource allocation scheme based on degradation heuristics and external policies.}
%   \item{A prototype implementation of the above.}
%   \item{Experimental results showing up to 40\% reduction in offered load and graceful degradation in oversubscribed edge systems.}
% \end{smitemize}

\section{Adaptation Architecture and Strategy}

\begin{figure}[h]
\centering
\includegraphics[width=\textwidth,trim=0 3em 11cm 0, clip]{FIGS/arch-vertical.pdf}
\vspace{-0.3in}
\caption{System Architecture}
\label{fig:arch}
\end{figure}

The original Gabriel platform has been validated in meeting the latency bounds
of WCA applications in single-user settings~\cite{chen2017empirical}.  Scalable
Gabriel aims to meet these latency bounds in multi-user settings, and to ensure
performant multitenancy even in the face of overload.  We take two complementary
approaches to scalability.  The first is for applications to reduce their
offered load to the wireless network and the cloudlet through adaptation.  The
second uses end-to-end scheduling of cloudlet resources to minimize queueing and
impacts of overload (See Chapter~\ref{chapter: cloudlet} for more details).  We
both approaches, and combine them using the system architecture shown in
Figure~\ref{fig:arch}.  We assume benevolent and collaborative clients in the
system.

\section{System Architecture}

Computer vision processing is at the core of wearable cognitive assistance. We
consider scenarios in which multiple Tier-3 devices concurrently offload their
vision processing to a single cloudlet over a shared wireless network.  The
devices and cloudlet work together to adapt workloads to ensure good performance
across all of the applications vying for the limited Tier-2 resources and
wireless bandwidth.  This is reflected in the system architecture shown in
Figure~\ref{fig:arch}.

Monitoring of resources is done at both Tier-3 and Tier-2.  Certain
resources, such as battery level, are device-specific and can only be
monitored at Tier-3.  Other shared resources can only be monitored at
Tier-2: these include processing cores, memory, and GPU.  Wireless
bandwidth and latency are measured independently at Tier-3 and Tier-2,
and aggregated to achieve better estimates of network conditions.

This information is combined with additional high-level predictive knowledge and
factored into scheduling and adaptation decisions.  The predictive knowledge
could arise at the cloudlet (e.g., arrival of a new device, or imminent change
in resource allocations), or at the Tier-3 device (e.g., application-specific,
short-term prediction of resource demand).  All of this information is fed to a
{\em policy module} running on the cloudlet.  This module is guided by an
external policy specification and determines how cloudlet resources should be
allocated across competing Tier-3 applications.  Such policies can factor in
latency needs and fairness, or simple priorities (e.g., a blind person
navigation assistant may get priority over an AR game).

A {\em planner module} on the Tier-3 device uses current resource
utilization and predicted short-term processing demand to determine
which workload reduction techniques (described in
Section~\ref{sec:workload-reduction}) should be applied to achieve
best performance for the particular application given the resource
allocations.

\begin{table*}[]
    \begin{tabular}{|r|p{25ex}|p{30ex}|p{25ex}|}
    \hline&&&\\[0.1in]
    &{\normalsize\bf Question}  & {\normalsize\bf Example} & {\normalsize\bf Load-reduction Technique} \\
    & & & \\ 
    \hline
    1&How often are instructions given, compared to task duration? 
        & Instructions for each step in IKEA lamp assembly are 
            rare compared to the total task time, e.g., 6 instructions over 
            a 10 minute task.
        & Enable adaptive sampling based on active and passive phases. \\ \hline
    2&Is intermittent processing of input frames sufficient for giving instructions?
        & Recognizing a face in any one frame is sufficient for
whispering the person's name.
        & Select and process key frames.  \\ \hline
    3&Will a user wait for system responses before proceeding?  
        & A first-time user of a medical device will pause until  an instruction is received.
        & Select and process key frames. \\ \hline
    4&Does the user have a pre-defined workspace in the scene?
        & Lego pieces are assembled on the lego board. Information outside the board can
             be safely ignored.
        & Focus processing attention on the region of interest. \\ \hline
    5&Does the vision processing involve identifying and locating objects?
        & Identifying a toy lettuce for a toy sandwich.
        & Use tracking as cheap approximation for detection. \\ \hline
    6&Are the vision processing algorithms insensitive to image resolution?
        & Many image classification DNNs limit resolutions to  
            the size of their input layers.
        & Downscale sampled frames on device before transmission.    \\ \hline
    7&Can the vision processing algorithm trade off accuracy and computation? 
        & Image classification DNN MobileNet is computationally cheaper than ResNet, but less accurate. 
        & Change computation fidelity based on resource utilization. \\ \hline
    % 8&Is there inherent human-level latency for a state change?
    %     & It takes at least a few seconds for a human to drill a hole.
    %     & Frame sampling and processing suppression based on heuristics   \\ \hline
    8&Can IMUs be used to identify the start and end of user activities?
        & User's head movement is significantly larger when searching for a Lego block.
        & Enable IMU-based frame suppression. \\ \hline
    9&Is the Tier-3 device powerful enough to run parts of the processing pipeline?
        & A Jetson TX2 can run MobileNet-based image recognition in real-time.
        & Partition the vision pipeline between Tier-3 and Tier-2.   \\ \hline
    \end{tabular}
\vspace{0.1in}
    \caption{Application characteristics and corresponding applicable techniques to reduce load}
    \label{tab:questions-techniques}
\end{table*}

\section{Adaptation Goals}
 
For WCAs, the dominant class of offloaded computations are computer vision
operations, e.g., object detection with deep neural networks (DNNs), or activity
recognition on video segments. The interactive nature of these applications
precludes the use of deep pipelining that is commonly used to improve the
efficiency of streaming analytics.  Here, end-to-end latency of an individual
operation is more important than throughput. Further, it is not just the mean or
median of latency, but also the tail of the distribution that matters.  There is
also significant evidence that user experience is negatively affected by
unpredictable variability in response times. Hence, a small mean with short tail
is the desired ideal. Finally, different applications have varying degrees of
benefit or utility at different levels of latency.  Thus, our adaptation
strategy incorporates application-specific utility as a function of latency as
well as policies maximizing the total utility of the system.


\section{Leveraging Application Characteristics}
\label{sec:workload-reduction}

WCA applications exhibit certain properties that distinguish them from
other video analytics applications studied in the past.  Adaptation
based on these attributes provides a unique opportunity to improve
scalability.

\textbf{Human-Centric Timing:} The frequency and speed with which guidance must
be provided in a WCA application often depends on the speed at which the human
performs a task step.  Generally, additional guidance is not needed until the
instructed action has been completed. For example, in the RibLoc assistant
(Chapter~\ref{chapter: background}), drilling a hole in bone can take several
minutes to complete.  During the drilling, no further guidance is provided after
the initial instruction to drill.  Inherently, these applications contain {\em
active phases,} during which an application needs to sample and process video
frames as fast as possible to provide timely guidance, and {\em passive phases,}
during which the human user is busy performing the instructed step.  During a
passive phase, the application can be limited to sampling video frames at a low
rate to determine when the user has completed or nearly completed the step, and
may need guidance soon.  Although durations of human operations need to be
considered random variables, many have empirical lower bounds.  Adapting
sampling and processing rates to match these active and passive phases can
greatly reduce offered load.  Further, the offered load across users is likely
to be uncorrelated because they are working on different tasks or different
steps of the same task. If inadvertent synchronization occurs, it can be broken
by introducing small randomized delays in the task guidance to different users.
These observations suggest that proper end-to-end scheduling can enable
effective use of cloudlet resources even with multiple concurrent applications.

\textbf{Event-Centric Redundancy}: In many WCA applications, guidance
is given when a user event causes visible state change. For example,
placing a lamp base on a table triggers the IKEA Lamp application to
deliver the next assembly instruction.  Typically, the application
needs to process video at high frame rate to ensure that such state
change is detected promptly, leading to further guidance.  However,
all subsequent frames will continue to reflect this change, and are
essentially redundant, wasting wireless and computing resources.
Early detection of redundant frames through careful semantic
deduplication and frame selection at Tier-3 can reduce the use of
wireless bandwidth and cloudlet cycles on frames that show no
task-relevant change.

\textbf{Inherent Multi-Fidelity}: Many vision processing algorithms
can tradeoff fidelity and computation.  For example, frame resolution
can be lowered, or a less sophisticated DNN used for inference, in
order to reduce processing at the cost of lower accuracy.  In many
applications, a lower frame rate can be used, saving computation and
bandwidth at the expense of response latency.  Thus, when a cloudlet
is burdened with multiple concurrent applications, there is scope to
select operating parameters to keep computational load manageable.
Exactly how to do so may be application-dependent.  In some cases,
user experience benefits from a trade-off that preserves fast response
times even with occasional glitches in functionality.  For others,
e.g., safety-critical applications, it may not be possible to
sacrifice latency or accuracy.  This in turn, translates to lowered
scalability of the latter class of application, and hence the need for
a more powerful cloudlet and possibly different wireless technology to
service multiple users.

\begin{figure}[h]
    \centering
    \includegraphics[width=\linewidth, trim=10em 2em 6em 2em, clip]{FIGS/fig-design-space.pdf}
    \caption{Design Space of WCA Applications}
    \label{figs:design-space}
\end{figure}
\subsection{Adaptation-Relevant Taxonomy}

\label{sec:taxonomy}

The characteristics described in the previous section largely hold for
a broad range of WCA applications.  However, the degree to which
particular aspects are appropriate to use for effective adaptation is
very application dependent, and requires a more detailed
characterization of each application.  To this end, our system
requests a manifest describing an application from the developers.
This manifest is a set of yes/no or short numerical responses to the
questions in Table~\ref{tab:questions-techniques}.  Using these, we
construct a taxonomy of WCA applications (shown in
Figure~\ref{figs:design-space}), based on clusters of applications along
dimensions induced from the checklist of questions.  In this case, we
consider two dimensions -- the fraction of time spent in "active"
phase, and the significance of the provided guidance (from merely
advisory, to critical instructions).  Our system varies the adaptation
techniques employed to the different clusters of applications.  We
note that as more applications and more adaptation techniques are
created, the list of questions can be extended, and the taxonomy can be
expanded.

\section{Adaptive Sampling}

The processing demands and latency bounds of a WCA application can
vary considerably during task execution because of human speed
limitations.  When the user is awaiting guidance, it is desirable to
sample input at the highest rate to rapidly determine task state and
thus minimize guidance latency.  However, while the user is performing
a task step, the application can stay in a passive state and sample at
a lower rate.  For a short period of time immediately after guidance
is given, the sampling rate can be very low because it is not humanly
possible to be done with the step.  As more time elapses, the
sampling rate has to increase because the user may be nearing
completion of the step.  Although this active-passive phase
distinction is most characteristic of WCA applications that provide
step-by-step task guidance (the blue cluster in
the lower right of Figure~\ref{figs:design-space}), most WCA
applications exhibit this behavior to some degree.  As shown in the
rest of this section, adaptive sampling rates can reduce processing
load without impacting application latency or accuracy.

We use task-specific heuristics to define application active and
passive phases.  In an active application phase, a user is likely to
be waiting for instructions or comes close to needing instructions,
therefore application needs to be ``active`` by sampling and
processing at high frequencies. On the other hand, applications can
run at low frequency during passive phases when an instruction is
unlikely to occur.

\begin{figure}
\centering
\includegraphics[width=\textwidth,trim=0em 3em 20em 15em, clip]{FIGS/fig-lego-sampling-model.pdf}
\caption{Dynamic Sampling Rate for LEGO}
\label{fig:lego-sampling-model}
\end{figure}

We use the LEGO application from Section~\ref{sec:example-apps} to show the
effectiveness of adaptive sampling. By default, the LEGO application runs at
active phase. The application enters passive phases immediately following the
delivery of an instruction, since the user is going to take a few seconds
searching and assembling LEGO blocks. The length and sampling rate of a passive
phase is provided by the application to the framework. We provide the following
system model as an example of what can be provided. We collect five LEGO traces
with 13739 frames as our evaluation dataset.

\textbf{Length of a Passive Phase: }
We model the time it takes to finish each step as a Gaussian distribution. We
use maximum likelihood estimation to calculate the parameters of the Guassian
model.

 
\begin{figure}
\centering
\begin{subfigure}{.45\linewidth}
  \centering
  \includegraphics[width=\linewidth, trim=0em 0em 0em 0em, clip]{FIGS/fig-lego-adaptive-sr.pdf}
  {(a) Passive Sampling Rate}
\end{subfigure}
\begin{subfigure}{.45\linewidth}
    \centering
    \includegraphics[width=0.92\linewidth, trim=0em 0em 0em 0em, clip]{FIGS/fig-lego-example-sr.pdf}
   {(b) Trace Sampling Rate}
\end{subfigure}
\caption{Adaptive Sampling Rate}
\label{fig:adaptive-sampling-example}
\end{figure}

\textbf{Lowest Sampling Rate in Passive Phase: }
The lowest sampling rate in passive phase still needs to meet application's
latency requirement. Figure~\ref{fig:lego-sampling-model} shows the system model
to calculate the largest sampling period S that still meets the latency bound.
In particular,
$$(k-1)S + processing\_delay \leq latency\_bound $$ $k$ represents the
cumulative number of frames an event needs to be detected in order to be
certain an event actually occurred. The LEGO application empirically sets this
value to be 5. 

\textbf{Adaptation Algorithm: }
At the start of a passive phase, we set the sampling rate to the
minimum calculated above.  As time progresses, we gradually increase
the sampling rate.  The idea behind this is that the initial low
sampling rates do not provide good latency, but this is acceptable, as
the likelihood of an event is low.  As the likelihood increases (based
on the Gaussian distribution described earlier), we increase sampling
rate to decrease latency when events are likely.
Figure~\ref{fig:adaptive-sampling-example}(a) shows the sampling rate
adaptation our system employs during a passive phase.
The sampling rate is calculated as $$sr = \\
%
 min\_sr + \alpha * (max\_sr - min\_sr) * cdf\_Gaussian(t)$$ 
%
$sr$ is the sampling rate. $t$ is the time after an instruction has been given. $\alpha$ is
a recovery factor which determines how quickly the sampling rate
rebounds to active phase rate. 


Figure~\ref{fig:adaptive-sampling-example}(b) shows the sampling rate
for a trace as the application runs. The video captures a user doing 7
steps of a LEGO assembly task. Each drop in sampling rate happens
after an instruction has been delivered to the user.
Table~\ref{tab:adaptive-sample-eval} shows the percentage of frames
sampled and guidance latency comparing adaptive sampling with naive
sampling at half frequency. Our adaptive sampling scheme requires
processing fewer frames while achieving a lower guidance latency.

\begin{table}[]
\centering
\begin{tabular}{|c|c|c|}
\hline
Trace  & \begin{tabular}[c]{@{}c@{}}Sample\\ Half Freq\end{tabular} & \begin{tabular}[c]{@{}c@{}}Adaptive\\ Sampling\end{tabular} \\ \hline
1 & 50\%          & 25\%              \\ \hline
2 & 50\%          & 28\%              \\ \hline
3 & 50\%          & 30\%              \\ \hline
4 & 50\%          & 30\%              \\ \hline
5 & 50\%          & 43\%              \\ \hline
\end{tabular}\\[0.1in]
{(a) Percentage of Frames Sampled}\\[0.2in]

\begin{tabular}{|c|c|}
\hline
                                                            & \begin{tabular}[c]{@{}c@{}}Guidance Delay \\ (frames$\pm$stddev)\end{tabular}
                                                            \\ \hline
\begin{tabular}[c]{@{}c@{}}Sample Half Freq\end{tabular}      & 7.6 $\pm$ 6.9                                                                  \\ \hline
\begin{tabular}[c]{@{}c@{}}Adaptive Sampling\end{tabular} &  5.9 $\pm$ 8.2                                                                  \\ \hline
% \begin{tabular}[c]{@{}c@{}}Theoretical Minimum\end{tabular} &  ?? $$                                                                  \\ \hline
\end{tabular}\\[0.1in]
{(b) Guidance Latency}\\[0.1in]
\caption{Adaptive Sampling Results}
\label{tab:adaptive-sample-eval}
\end{table}
\section{IMU-based Approaches: Passive Phase Suppression + Evaluation of Image Quality}

In many applications, passive phases can often be associated with the
user's head movement. We illustrate with two applications here. In
LEGO, during the passive phase, which begins after the user receives
the next instruction, a user typically turns away from the LEGO board
and starts searching for the next brick to use in a parts box. During
this period, the computer vision algorithm would detect no meaningful
task states if the frames are transmitted.  In PING PONG, an active
phase lasts throughout a rally.  Passive phases are in between actual
game play, when the user takes a drink, switches sides, or, most
commonly, tracks down and picks up a wayward ball from the floor.
These are associated with much large range of head movements than
during a rally when the player generally looks toward the opposing
player.  Again, the frames can be suppressed on the client to reduce
wireless transmission and load on the cloudlet.  In both scenarios,
significant body movement can be detected through Inertial Measurement
Unit (IMU) readings on the wearable device, and used to predict those
passive phases.

For each frame, we get a six-dimensional reading from the IMU:
rotation in three axes, and acceleration in three axes.  We train an
application-specific SVM to predict active/passive phases based on IMU
readings, and suppress predicted passive frames on the client.
Figure~\ref{fig:imu-trace-example}(a) and (b) show an example trace
from LEGO and PING PONG, respectively.  Human-labeled ground truth
indicating passive and active phases is shown in blue.  The red dots
indicate predictions of passive phase frames based on the IMU
readings; these frames are suppressed at the client and not
transmitted.  Note that in both traces, the suppressed frames also
form streaks. In other words, a number of frames in a row can be
suppressed. As a result, the saving we gain from IMU is orthogonal to
that from adaptive sampling.

Although the IMU approach does not capture all of the passive frames
(e.g., in LEGO, the user may hold his head steady while looking for
the next part), when a passive frame is predicted, this is likely
correct (i.e., high precision, moderate recall).  Thus, we expect
little impact on event detection accuracy or latency, as few if any
active phase frames are affected.  This is confirmed in
Table~\ref{tab:imu-result}, which summarizes results for five traces
from each application.  We are able to suppress up to 49.9\% of
passive frames for LEGO and up to 38.4\% of passive frames in case of
PING PONG on the client, while having minimal impact on application
quality --- incurring no delay in state change detection in LEGO, and
less than 2\% loss of active frames in PING PONG.



\section{Related Work}
\label{sec: workload-related}

Although edge computing is new, the techniques for reducing offered load to
adapt application behaviors examined in this chapter bear some resemblance to
work that was done in the early days of mobile computing.

Several different approaches to adapting application fidelity have been studied.
Dynamic sampling rate with various heuristics for adaptation have been tried
primarily in the context of individual mobile devices for energy
efficiency~\cite{lorincz2009mercury, lorincz2008resource, vallina2012energy,
lane2010survey}. Semantic deduplication to reduce redundant processing of frames
have been suggested by ~\cite{Hu2015, kang2017noscope, hsieh2018focus,
zhang2015design}. Similarly, previous works have looked at suppression based on
motion either from video content~\cite{naderiparizi2017glimpse,
lebeckcollaborative} or IMUs~\cite{jain2015overlay}. Others have investigated
exploiting multiple deep models with accuracy and resource
tradeoffs~\cite{han2016mcdnn,jiang2018chameleon}. In addition, using tracking as
approximate result of detection and recognition has been explored to leverage
the temporal locality of video data and reduce computational
demand~\cite{wang2017scalable, chen2015glimpse, you1999hybrid}. While most of
these efforts were in mobile-only, cloud-only, or mobile-cloud context, we
explore similar techniques in an edge-native context.


Partitioning workloads between mobile devices and the cloud have been
studied in sensor networks~\cite{newton2009wishbone},
throughput-oriented systems~\cite{cuervo2010maui, yi2017lavea}, for
interactive applications~\cite{ra2011odessa, chen2015glimpse}, and
from programming model perspectives~\cite{balan2003tactics}. We believe
that these approaches will become important techniques to scale
applications on heavily loaded cloudlets.

\section{Chapter Summary and Discussion}
\label{sec:workload-reduction-summary}

In this chapter, we demonstrate that scalability can be increased by leveraging
application characteristics to reduce offered load. Our approach to increased
scalability is through adaptation. Specifically, we first present an
adaptation-centric architecture that monitors and coordinates Tier-3 devices and
the edge server. When contention arises, enabled are application-specific
techniques to reduce offered load to the edge server. In addition, we highlight
two of adaptation techniques, selective sampling and IMU-based suppression. Our
experiment show that they can significantly reduce the offered load. Finally, we
provide a taxonomy to help developers reason about characteristics of their
applications, and identify and specify reduction techniques applicable to their
needs.

\chapter{Cloudlet Resource Management for Graceful Degradation of Service}
\label{chapter: cloudlet}

In addition to workload reduction at the client as discussed in
Chapter~\ref{chapter: load}, another key aspect of adaptation lies in the
resource management of cloudlet resources. We argue that naive resource
management schemes using statistical multiplexing cannot satisfy the needs of
edge-native applications in oversubscribed edge scenarios. Needed are
intelligent mechanisms that takes account of application degradation behaviors.
In this chapter, we introduce and evaluate such an adaptation-centric cloudlet
resource management mechanism.

In Original Gabriel which relies on operating system level statistical
multiplexing for resource sharing, an increasing number of clients means less
resources for each client. This results in all clients and all applications
suffering from long response time. For wearable cognitive assistance, long
response time results in feedback delivered too late, which has significantly
decreased value. In a data-center setting, a cloud-native application, can
quickly scale-up or scale-out by acquiring additional resources (e.g.
instantiating more virtual machines). However, at the edge, since the total
amount of hardware resources is constrained, acquiring more resources in face of
a flash crowd is not possible. Instead, we make the observation that
applications behave differently when the amount of resources allocated to them
changes. We can leverage application adaptation characteristics to create a
judicious and intelligent resource allocation plan that prioritizes some
applications. In particular, we focus on mechanisms rather than policies. Our
mechanism in this chapter enables external allocation policies to divide
cloudlet resources taking account of adaptation characteristics, so that quality
of service can be maintained for some clients.



\section{System Model and Application Profiles}

% A complementary method to improve scalability is through judicious
% allocation of cloudlet resources among concurrent application
% services. 

Resource allocation has been well explored in many contexts
of computer systems, including operating system, networks, real-time
systems, and cloud data centers.  While these prior efforts can
provide design blueprints for cloudlet resource allocation, the
characteristics of edge-native applications emphasize unique design
challenges.

The ultra-low application latency requirements of edge-native
applications are at odds with large queues often used to maintain high
resource utilization of scarce resources.  Even buffering a small
number of requests may result in end-to-end latencies that are several
multiples of processing delays, hence exceeding acceptable latency
thresholds.  On the other hand, when using short queues, accurate
estimations of throughput, processing, and networking delay are vital
to enable efficient use of cloudlet resources.  However, sophisticated
computer vision processing represents a highly variable computational
workload, even on a single stream. For example, as shown in
Figure~\ref{fig:lego-dag}, the processing pipeline for LEGO has many
exits, resulting in highly variable execution times.

\begin{figure}
\centering
\includegraphics[width=\linewidth, trim=0em 30em 20em 0em,clip]{FIGS/fig-lego-dag.pdf}
\caption{LEGO Processing DAG}
\label{fig:lego-dag}
\end{figure}

To adequately provision resources for an application, one approach is
to leave the burden to developers, asking them to specify and reserve
a static amount of cores and memories needed for the service. However,
this method is known to be highly inaccurate and typically leads to
over-reservation in data-centers. For cloudlets, which are more
resource constrained, such over-reservation will lead to even worse
under-utilization or inequitable sharing of the available resources.
Instead, we seek to create an automated resource allocation system
that leverages knowledge of the application requirements and minimizes
developer effort.  To this end, we ask developers to provide target
Quality of Service (QoS) metrics or a utility function that relates a
single, easily-quantified metric (such as latency) to the quality of
user experience.  Building on this information, we construct offline
application profiles that map multidimensional resource allocations to
application QoS metrics.  At runtime, we calculate a resource
allocation plan to maximize a system-wide metric (e.g., total utility,
fairness) specified by cloudlet owner. We choose to consider the
allocation problem per application rather than per client in order to
leverage statistical multiplexing among clients and multi-user
optimizations (e.g., cache sharing) in an application.

\begin{figure}
\centering
\includegraphics[width=0.5\linewidth]{FIGS/fig-allocation-system-model-cropped.pdf}
\begin{captiontext}Only request flow is shown.\end{captiontext}
\caption{Resource Allocation System Model}
\label{fig:allocation-system-model}
\vspace{-0.2in}
\end{figure}

\subsection{System Model}
Figure~\ref{fig:allocation-system-model} shows the system model we consider.
Each application is given a separate input queue. Each queue can feed one or
more application instances, which are the units of application logic that can be
replicated (e.g. a single process or several collaborative processes). Each
application instance is encapsulated in a container with controlled resources.
In this model, with adequate computational resources, clients of different
applications have minimal sharing and mainly contend for the wireless network.

We use a utility-based approach to measure and compare system-wide
performance under different allocation schemes. For WCA, the utility
of a cloudlet response depends on both the quality of response and its
QoS characteristics (e.g., end-to-end latency). The total utility of a
system is the sum of all individual utilities. A common limitation of
a utility-based approach is the difficulty of creating these
functions. One way to ease such burden is to position an application
in the taxonomy described in Section~\ref{sec:taxonomy} and borrow
from similar applications. Another way is to calculate or measure
application latency bounds, such as through literature review or
physics-based calculation as done in~\cite{chen2017empirical}.

The system-wide performance is a function of the following independent
variables. 
\begin{enumerate}[label=(\alph*)]
    \item the number of applications and the number of clients of
each application;
    \item the number of instances of each application;
    \item the resource allocation for each instance;
\end{enumerate}

Although (a) is not under our control, Gabriel is free to adapt (b) and (c).
Furthermore, to optimize system performance, it may sacrifice the performance of
certain applications in favor of others. Alternatively, it may choose not to run
certain applications.


\begin{figure}
\centering
\begin{minipage}[b]{.35\linewidth}
\centering
\includegraphics[width=\linewidth,trim=0em 0em 0em 0em, clip]{FIGS/fig-lat-util-face.pdf}
{(a) Utility For FACE}
\end{minipage}
\begin{minipage}[b]{.6\linewidth}
\centering
\includegraphics[width=\linewidth, trim=0em 0em 0em 0em, clip]{FIGS/fig-app-profile-face.pdf}
{(b) Profile for FACE}
\end{minipage}
\caption{FACE Application Utility and Profile}
\label{fig:face-utility-and-profile}
\end{figure}

\begin{figure}
\centering
\begin{minipage}[b]{.35\linewidth}
\centering
\includegraphics[width=\linewidth,trim=0em 0em 0em 0em, clip]{FIGS/fig-lat-util-pool.pdf}
{(a) Utility For POOL}
\end{minipage}
\begin{minipage}[b]{.6\linewidth}
\centering
\includegraphics[width=\linewidth, trim=0em 0em 0em 0em, clip]{FIGS/fig-app-profile-pool.pdf}
{(b) Profile for POOL}
\end{minipage}
\caption{POOL Application Utility and Profile}
\label{fig:pool-utility-and-profile}
\end{figure}

\subsection{Application Utility and Profiles}

We build application profiles offline in order to estimate latency and
throughput at runtime. First, we ask developers to provide a utility
function that maps QoS metrics to application experience.
Figure~\ref{fig:face-utility-and-profile}(a) and
Figure~\ref{fig:pool-utility-and-profile}(a) show utility functions
for two applications based on latency bounds identified
by~\cite{chen2017empirical} for each request. Next, we profile an application
instance by running it under a discrete set of cpu and memory
limitations, with a large number of input requests. We record the
processing latency and throughput, and calculate the system-wide
utility per unit time. We interpolate between acquired data points of
(system utility, resources) to produce continuous functions.  Hence,
we effectively generate a multidimensional resource to utility profile
for each application.

We make a few simplifying assumptions to ensure profile generation and
allocation of resources by utility are tractable.  First, we assume utility
values across different applications are comparable. Furthermore, we assume
utility is received on a per-frame basis, with values that are normalized
between 0 and 1.  Each frame that is sent, accurately processed, and replied
within its latency bound receives 1, so a client running at 30 FPS under ideal
conditions can receive a maximum utility of 30 per second.  This clearly ignores
variable utility of processing particular frames (e.g., differences between
active and passive phases), but simplifies construction of profiles and modeling
for resource allocation; we leave the complexities of variable utility to future
work. Figure~\ref{fig:face-utility-and-profile}(b) and
Figure~\ref{fig:pool-utility-and-profile}(b) show the generated application
profiles for FACE and POOL. We see that POOL is more efficient than FACE in
using per unit resource to produce utility. If an application needs to deliver
higher utility than a single instance can, our framework will automatically
launch more instances of it on the cloudlet.

\section{Profiling-based Resource Allocation}
\label{sec: resource-allocation}

Given a workload of concurrent applications running on a cloudlet, and the
number of clients requesting service from each application, our resource
allocator determines how many instances to launch and how much resource (CPU
cores, memory, etc.) to allocate for each application instance.  We assume
queueing delays are limited by the token mechanism used in Original Gabriel,
which limits the number of outstanding requests on a per-client basis.


\subsection{Maximizing Overall System Utility}

As described earlier, for each application $a \in $ \{FACE, LEGO, PING PONG, POOL, \ldots \}, 
we construct a resource to utility mapping
$u_a: \mathbf{r} \rightarrow \mathbb{R}$ for one instance of the application on cloudlet, 
where $\mathbf{r}$ is a resource vector of allocated CPU, memory, etc. We formulate the 
following optimization problem which maximizes the system-wide total utility,
subject to a tunable maximum per-client limit:

\begin{equation}
  \begin{aligned}
  \max_{\{k_a, \mathbf{r}_a\}} \quad & \sum_a{k_a \cdot u_a(\mathbf{r}_a)} \\
  \textrm{s.t.} \quad & \sum_a k_a \cdot \mathbf{r}_a \preccurlyeq \hat{\mathbf{r}} \\
      & 0 \preccurlyeq \mathbf{r}_a  \quad \forall a \\
      & k_a \cdot u_a(\mathbf{r}_a) \le \gamma \cdot c_a \quad \forall a \\
      & k_a \in \mathbb{Z}
  \end{aligned}
  \end{equation}

In above, $c_a$ is the number of mobile clients requesting service from application $a$.
The total resource vector of the cloudlet is  $\hat{\mathbf{r}}$. 
 For each application $a$, we determine how many instances to launch --- $k_a$, and 
allocate resource vector $\mathbf{r}_a$ to each of them.
A tunable knob $\gamma$ regulates the maximum utility allotted 
per application, and serves to enforce a form of partial fairness (no application
can be given excessive utility, though some may still receive none). 
The larger $\gamma$ is, the more aggressive our scheduling algorithm
will be in maximizing global utility and
suppressing low-utility applications. 
%In our system model, the maximum $\gamma$ is 30 for
%clients at 30 FPS. 
By default, we set $\gamma=10$, which, based on our definition of
utility, roughly means resources will be allocated so 
no more than one third of frames (from a 30FPS source) 
will be processed within satisfactory latency bounds for a given
client.

Solving the above optimization problem is computationally difficult. We thus use an
iterative greedy allocation algorithm as follows.

\begin{algorithm}[H]
\SetAlgoLined
% \KwResult{Write here the result }
 Profile applications under varying resources\;
 $u_a(\mathbf{r})$: resource $\mathbf{r}$ to utility mapping for application
 $a$\;
\For{each application}{
  find the highest \emph{utility-to-resource} ratio $\frac{u_a(\mathbf{r})}{|\mathbf{r}|}$\;
}
 \While{leftover system resource}{
 Find the application with the largest \emph{utility-to-resource}
 $\frac{u_a(\mathbf{r}^*_a)}{|\mathbf{r}^*_a|}$, which has not been allocated
 resources\;
Allocate $k_a$ application instances,
each with resource $\mathbf{r}^*_a$, such that $k_a$ is the largest integer with
$k_a \cdot u_a(\mathbf{r}^*_a) \le \gamma \cdot c_a$\;
 }
 \caption{Iterative Allocation Algorithm to Maximize Overall System Utility}
\end{algorithm}

% For each application profile $u_a(\mathbf{r})$, we find the resource point that gives
% the highest $\frac{u_a(\mathbf{r})}{|\mathbf{r}|}$, i.e., \emph{utility-to-resource} ratio. 
% Denote this point as $\mathbf{r}^*_a$. We start with the application with the largest 
% $\frac{u_a(\mathbf{r}^*_a)}{|\mathbf{r}^*_a|}$. We allocate $k_a$ application instances,
% each with resource $\mathbf{r}^*_a$, such that $k_a$ is the largest integer with
% $k_a \cdot u_a(\mathbf{r}^*_a) \le \gamma \cdot c_a$. 
% If there is leftover resource, we move to the application with the next highest 
% utility-to-resource ratio and repeat the process.

In our implementation, we exploit the \texttt{cpu-shares} and
\texttt{memory-reservation} control options of Linux Docker containers. It puts
a soft limit on containers' resource utilization only when they are in
contention, but allows them to use as much left-over resource as needed.
\section{Evaluation}

We use five WCA applications, including FACE, PING PONG, LEGO, POOL, and IKEA
for evaluation~\cite{chen2017empirical,chen2018application}. These
applications are selected based on their distinct requirements and
characteristics to represent the variety of WCA apps. IKEA and LEGO assist users
step by step to assemble an IKEA lamp or a LEGO model. While their 2.7-second
loose latency bound is less stringent than other apps, the significance of their
instructions is high, as a user could not proceed without the instruction. On
the other hand, users could still continue their tasks without the instructions
from FACE, POOL, and PING PONG assistants. For POOL and PING PONG, the speed of
an instruction is paramount to its usefulness. For example, any instruction that
comes 105ms after a user action for POOL is no longer of value, because it is
too late to guide the next action.

\begin{figure}
  \centering
  \includegraphics[width=\linewidth, trim=0em 3em 0em 3em, clip]{FIGS/fig-sec6-reduction-legend.pdf}
  \begin{minipage}[b]{0.38\linewidth}
    \centering
    \includegraphics[height=2.5in, trim=0em 1em 0em 0em, clip]{FIGS/fig-sec6-reduction-Pingpong.pdf}\\
    {(a) PING PONG}
  \end{minipage}
  \begin{minipage}[b]{0.3\linewidth}
    \centering
    \includegraphics[height=2.5in, trim=0em 1em 0em 0em, clip]{FIGS/fig-sec6-reduction-Lego.pdf}\\
    {(b) LEGO}
  \end{minipage}
  \begin{minipage}[b]{0.3\linewidth}
    \centering
    \includegraphics[height=2.5in, trim=0em 1em 0em 0em, clip]{FIGS/fig-sec6-reduction-Pool.pdf}\\
    {(c) POOL}
  \end{minipage}
  \caption{Effects of Workload Reduction}
  \label{figs:workload-reduction}
\end{figure}


\subsection{Effectiveness of Workload Reduction}

We first evaluate the effectiveness of all of the workload reduction techniques
explored in Chapter~\ref{chapter: load}. For this set of experiments, we do not
use multiple concurrent applications. Adaptation-centric cloudlet resource
allocation is not enabled for a controlled setup. We use four Nexus 6 mobile
phones as clients. They offload computation to a cloudlet over Wi-Fi links. We
run PING PONG, LEGO, and POOL applications one at a time with 2, 4, 6, and 8
cores on the edge server. We constrain the number of cores available using Linux
cgroup. Figure~\ref{figs:workload-reduction} shows the total number of frames
processed with and without workload reduction. The yellow lines for Original
Gabriel do not have workload reduction while the blue lines for Scalable Gabriel
do. The solid lines represent the total number of frames offloaded. The dashed
lines represent the number of active frames, those frames that actually contain
user state information. Note that although the offered work is greatly reduced,
the processed frames for active phases of the application have not been
affected. Thus, we confirm that we can significantly reduce cloudlet load
without affecting the critical processing needed by these applications.

\begin{table}[]
  \centering
  \begin{tabular}{|c|c||c|c|c|c|c|}
    \hline
    Exp & \multicolumn{6}{|c|}{Number of Clients}                                    \\
    \cline{2-7}
    \#  & Total                                   & FACE & LEGO & POOL & PING & IKEA \\
        &                                         &      &      &      & PONG &      \\ \hline
    1   & 15                                      & 3    & 3    & 3    & 3    & 3    \\ \hline
    2   & 20                                      & 4    & 4    & 4    & 4    & 4    \\ \hline
    3   & 23                                      & 5    & 5    & 4    & 4    & 5    \\ \hline
    4   & 25                                      & 5    & 5    & 5    & 5    & 5    \\ \hline
    5   & 27                                      & 5    & 6    & 6    & 5    & 5    \\ \hline
    6   & 30                                      & 5    & 7    & 6    & 6    & 6    \\ \hline
    7   & 32                                      & 5    & 7    & 7    & 7    & 6    \\ \hline
    8   & 40                                      & 8    & 8    & 8    & 8    & 8    \\ \hline
  \end{tabular}
  \vspace{0.1in}
  \caption{Resource Allocation Experiment Setup}
  \label{tab:alloc-exps}
\end{table}

\subsection{Effectiveness of Resource Allocation}

We next evaluate our adaptation-centric resource allocation mechanism on a
server machine with 2 Intel{\textregistered} Xeon{\textregistered} E5-2699 v3
processors, totaling 36 physical cores running at 2.3 Ghz (turbo boost disabled)
and 128 GB memory. We dedicate 8 physical cores (16 Intel{\textregistered} hyper
threads) and 16 GB memory as cloudlet resources using cgroup. We run 8
experiments with increasing numbers of clients across four concurrent
applications. The total number of clients gradually increases from 15 to 40.
Table~\ref{tab:alloc-exps} shows the breakdown of the number of clients used for
each experiment. Note that these clients are running simultaneously, resulting
in heavier and heavier contention. We generate application adapation profiles
offline using the method discussed in Section~\ref{sec: resource-allocation}. We
leverage these profiles to optimize for maximizing the total system utility.
Figure~\ref{fig:alloc-max-util} shows how the total system utility changes as we
add more clients and hence more workload. The yellow line represents the
Original Gabriel which relies on the operating system alone to divide system
resources. The blue line shows our Scalable Gabriel approach. In the beginning,
while the system is under-utilized, we see that the Original Gabriel yields
slightly higher total utility. However, as contention increases, Original
Gabriel's total utility quickly drops, eventually more than 40\%, since every
client contends for resources in an uncontrolled fashion.  All applications
suffer, but the effects of increasing latencies are vastly different among
different applications. In contrast, scalable Gabriel maintains a high level of
system-wide utility by differentially allocating resources to different
applications based on their sensitivity captured in the adaptation profiles.

\begin{figure}[h]
  \centering
  \includegraphics[width=.9\linewidth]{FIGS/fig-alloc-max-util.pdf}
  \caption{Total Utility with Increasing Contention}
  \label{fig:alloc-max-util}
\end{figure}


Figure~\ref{fig:alloc-latency} and Figure~\ref{fig:alloc-fps} provide insights
into how scalable Gabriel strikes the balance. We present both application
throughput in terms of average frames per second and latency in terms of
90\%-tile response delay. Latencies are better controlled as resources are
dedicated to applications with high utility, and more clients are kept within
their latency bounds. Of course, with higher contention, fewer frames per second
can be processed for each client. Original Gabriel degrades applications in an
undifferentiated fashion. Scalable Gabriel, in contrast, tries to maintain
higher throughput for some applications at the expense of the others, e.g. LEGO
up to 27 clients. The accuracies of application profiles influence how well
Scalable Gabriel can manage latency. Run-time resource demand could deviate
from profiles due to the differences in the request content (e.g. image
content). Profile inaccuracies result in the overshoot of POOL and IKEA
90\%-tile latencies in Figure~\ref{fig:alloc-latency}, as the profiles
underestimate their resource demand and overestimate LEGO resource demand when
the number of clients is low. When the system becomes more crowded, the
throttling of LEGO throughput reduces such an effect.

\begin{figure}[]
  \begin{center}
    \includegraphics[width=\linewidth]{FIGS/fig-alloc-latency-legend.pdf}
    \includegraphics[width=\linewidth]{FIGS/fig-alloc-latency-baseline.pdf}
    {(a) Original Gabriel}
    \includegraphics[width=\linewidth]{FIGS/fig-alloc-latency-cpushares.pdf}
    {(b) Scalable Gabriel}
  \end{center}
  \begin{captiontext}
    \centering
    The normalization is  by per-application tight and loose
    bounds~\cite{chen2017empirical}.

    The allocation policy is to maximize the overall system utility.
  \end{captiontext}
  \caption{Normalized 90\%-tile Response Latency}
  \label{fig:alloc-latency}
\end{figure}

\begin{figure}[]
  \begin{center}
    \includegraphics[width=.9\linewidth]{FIGS/fig-alloc-latency-legend.pdf}
    \includegraphics[width=\linewidth]{FIGS/fig-alloc-fps-baseline.pdf}
    {(a) Original Gabriel}
    \includegraphics[width=\linewidth]{FIGS/fig-alloc-fps-cpushares.pdf}
    {(b) Scalable Gabriel}
  \end{center}
  \caption{Average Processed Frames Per Second Per Client}
  \label{fig:alloc-fps}
\end{figure}


\subsection{Effects on Guidance Latency}

We next evaluate the combined effects of workload reduction and
resource allocation in our system. We emulate many users running
multiple applications simultaneously. All users share the same
cloudlet with 8 physical cores and 16 GB memory. We conduct three experiments,
with 20 (4 clients per app), 30 (6 clients per app), and 40 (8 clients
per app) clients. Each client loops through pre-recorded video traces
with random starting points.  Figure~\ref{fig:frame-latency} and
Fig~\ref{fig:frame-fps} show per client frame latency and FPS
achieved. The first thing to notice is that concurrently utilizing
both sets of techniques does not cause conflicts. In fact, they appear
to be complementary and latencies remain in better control than using
resource allocation alone.

\begin{figure}[]
  \begin{center}
    \includegraphics[width=\linewidth]{FIGS/fig-alloc-latency-legend.pdf}

    \begin{tabular}{c@{}c}
      \includegraphics[width=.5\linewidth]{FIGS/fig-eval-latency-baseline.pdf}
                              & \includegraphics[width=.5\linewidth]{FIGS/fig-eval-latency-cpushares.pdf} \\
      {(a) Original  Gabriel} & {(b) Scalable Gabriel}
    \end{tabular}
  \end{center}

  \begin{captiontext}
    \centering
    The normalization is by per-application tight and loose
    bounds~\cite{chen2017empirical}.
  \end{captiontext}
  \vspace{-0.1in}
  \caption{Normalized 90\%-tile Response Latency}
  \label{fig:frame-latency}
\end{figure}

\begin{figure}[]
  \begin{center}
    % \includegraphics[width=\linewidth]{FIGS/fig-alloc-latency-legend.pdf}
    \begin{tabular}{c@{}c}
      \includegraphics[width=.5\linewidth]{FIGS/fig-eval-fps-baseline.pdf}
                             & \includegraphics[width=.5\linewidth]{FIGS/fig-eval-fps-cpushares.pdf} \\
      {(a) Original Gabriel} & {(b) Scalable Gabriel}
    \end{tabular}
  \end{center}
  \caption{Processed Frames Per Second Per Application}
  \label{fig:frame-fps}
\end{figure}

\begin{figure}[]
  \begin{center}
    \includegraphics[width=.7\linewidth]{FIGS/fig-alloc-latency-legend.pdf}
    \includegraphics[width=.7\linewidth]{FIGS/fig-sec6-latency-allocation.pdf}
  \end{center}
  \vspace{-0.1in}
  \caption{Fraction of Cloudlet Processing Allocated}
  \label{figs:resource-allocated}
\end{figure}

\begin{figure}[]
  \centering
  \includegraphics[width=0.5\linewidth, trim=0em 0em 0em 0em, clip]{FIGS/fig-sec6-latency-legend.pdf} \\
  \centering
  \begin{turn}{90}{\hspace{0.6in}\small (a) FACE}\end{turn}\hspace{0.2in}\includegraphics[width=2.5in, trim=0em 0em 0em 0em, clip]{FIGS/fig-sec6-latency-face.pdf}
  \begin{turn}{90}{\hspace{0.6in}\small (b)
      LEGO}\end{turn}\hspace{0.2in}\includegraphics[width=2.5in, trim=0em 0em 0em 0em,
    clip]{FIGS/fig-sec6-latency-lego.pdf}\\[0.08in]
  \vspace{0in}
  \begin{turn}{90}{\hspace{0.6in}\small (c) PING PONG}\end{turn}\hspace{0.2in}\includegraphics[width=2.5in, trim=0em 0em 0em 0em, clip]{FIGS/fig-sec6-latency-pingpong.pdf}
  \begin{turn}{90}{\hspace{0.6in}\small (d)
      POOL}\end{turn}\hspace{0.2in}\includegraphics[width=2.5in, trim=0em 0em 0em 0em,
    clip]{FIGS/fig-sec6-latency-pool.pdf}\\[0.08in]
  \vspace{0in}
  \begin{turn}{90}{\hspace{0.6in}\small (e) IKEA}\end{turn}\hspace{0.2in}\includegraphics[width=2.5in, trim=0em 0em 0em 0em, clip]{FIGS/fig-sec6-latency-ikea.pdf}
  \caption{Guidance Latency Compared to Loose Latency Bound}
  \label{figs:inst-delay}
\end{figure}

The previous plots consider per request latencies. The ultimate goal of our work
is to maintain user experience as much as possible and degrade it gracefully
when overloaded. For WCA applications, the key measure of user experience is
guidance latency, the time between the occurrence of an event and the delivery
of corresponding guidance. Note that guidance latency is different than per
request latency, as a guidance may need not one but several frames to recognize
a user state. Figure~\ref{figs:inst-delay} shows boxplots of per-application
guidance latency for the concurrent application experiments above. The red
dotted line denotes the application-required loose bound. It is clear that our
methods control latency significantly better than the baseline. Scalable Gabriel
is able to serve at least 3x number of clients when moderately loaded while
continuing to serve more than half of the clients when severely loaded. In these
experiments, the utility is maximized at the expense of the FACE application,
which provides the least utility per resource consumed. At the highest number of
clients, scalable Gabriel sacrifices the LEGO application to maintain the
quality of service for PINGPONG and POOL. This differentiated allocation is
reflected in Figure~\ref{figs:resource-allocated}. In contrast, with original
Gabriel, none of the applications are able to regularly meet deadlines.
\section{R\lc{elated} W\lc{ork}}
\label{sec: resource-management-related}

Although edge computing is new, the techniques for scalability
examined in this paper bear some resemblance to work that was done in
the early days of mobile computing, and more recent cloud management
work.

Odyssey~\cite{Noble1997} and extensions~\cite{Flinn1999} proposed upcall-based
collaboration between a mobile's operating system and its applications to adapt
to variable wireless connectivity and limited battery. Such adaption was purely
reactive by the mobile device; in our context, adaptation for a collection of
devices can be centrally managed by their cloudlet, with failover to reactive
methods as needed. Exploration of tradeoffs between application fidelity and
resource demand led to the concept of {\em multi-fidelity
applications}~\cite{Satya1999}; such concepts are relevant to our work, but the
critical computing resources in our setting are those of the cloudlet rather
than the mobile device.

Several different approaches to adapting application fidelity have been studied.
Dynamic sampling rate with various heuristics for adaptation have been tried
primarily in the contexts of individual mobile devices for energy
efficiency~\cite{lorincz2009mercury, lorincz2008resource, vallina2012energy,
lane2010survey}. Semantic deduplication to reduce redundant processing of frames
have been suggested by ~\cite{Hu2015, kang2017noscope, hsieh2018focus,
zhang2015design}. Similarly, previous works have looked at suppression based on
motion either from video content~\cite{naderiparizi2017glimpse,
lebeckcollaborative} or IMUs~\cite{jain2015overlay}. Others have investigated
exploiting multiple deep models with accuracy and resource
tradeoff~\cite{han2016mcdnn,jiang2018chameleon}. While most of these efforts
were in mobile-only, cloud-only, or mobile-cloud context, we explore similar
techniques in an edge-native context.


Partitioning workloads between mobile devices and the cloud have been
studied in sensor networks~\cite{newton2009wishbone},
throughput-oriented systems~\cite{cuervo2010maui, yi2017lavea}, for
interactive applications~\cite{ra2011odessa, chen2015glimpse}, and
from programming model perspective~\cite{balan2003tactics}. We believe
that these approaches will become important techniques to scale
applications on heavily loaded cloudlets.

Dynamic resource allocation schemes on the cloud for video processing have been
well explored~\cite{sembiring2013dynamic, fu2015drs, kaseb2015cloud}.  More
recently, profile-based adaptation of video analytics~\cite{zhang2017live,
hung2018videoedge, jiang2018chameleon} focused on throughput-oriented analytics
application on large clusters or cloud. In contrast, our goals focus on
interactive performance on relatively small edge deployments.




\section{Chapter Summary and Discussion}
\label{cloudlet: summary}

More than a decade ago, the emergence of cloud computing led to the
realization that applications had to be written in a certain way to
take full advantage of elasticity of the cloud.  This led to the
concept of ``cloud-native applications'' whose scale-out capabilities
are well matched to the cloud, as well as tools and techniques to
easily create such applications.

The emergence of edge computing leads to another inflection point in application
design.  As last two chapters have shown, edge-native applications have to be
written in a way that is very different from cloud-native applications if they
are to be scalable. We explore client workload reduction and server resource
allocation to manage application quality of service in the face of contention
for cloudlet resources. We demonstrate that our system is able to ensure that in
overloaded situations, a subset of users are still served with good quality of
service rather than equally sharing resources and missing latency requirements
for all.

% In particular, it leads to ``edge-native applications'' (e.g Wearable
% Cognitive Assistance) that are deeply dependent on attributes such as low
% latency or bandwidth scalability that can only be obtained at the edge. However,
% as last two chapters have shown, edge-native applications have to be written in
% a way that is very different from cloud-native applications if they are to be
% scalable.

% We show that cloud-native implementation strategies that focus primarily on
% dynamic scale-out are unlikely to be effective for scalability in edge
% computing.  Instead, wearable cognitive assistance need to adapt their network
% and cloudlet resource demand to system load.  As the total number of Tier-3
% devices associated with a cloudlet increases, the per-device network and
% cloudlet load has to decrease.  This is a fundamental difference between
% cloud-native and edge-native approaches to scalability. 

\chapter{Simplifying Application Development}
\section{Tools For Painless Object Detection (TPOD)}
\section{Finite State Machine Authoring Tools}
\section{Discussion}

% \chapter{Simplifying Application Deployment}
\section{Cloudlet Gateway}
\section{Enabling GPU Usage for Cloudlets}
\label{enable-gpu-for-cloudlets}

Many edge workloads require a GPU to meet latency requirements. However,
sharing and virtualizing GPU devices for computation remain a challenge.
In below, I will describe our group's setup to allow multi-tenancy on
GPUs.

\subsection{Architecture}\label{architecture}

The figure below represents the architecture. We adopt a containers on
top of virtual machines approach to allow GPU sharing. The GPU device is
dedicated to a particular VM through GPU-passthrough on the host.

\begin{figure}[htbp]
\centering
\includegraphics{GPU-Support-in-Cloudlet.png}
\caption{Architecture}
\end{figure}

\subsection{Setup}\label{setup}

\subsubsection{GPU-Passthrough to a libvirt
VM}\label{gpu-passthrough-to-a-libvirt-vm}

The setup process is automated through Ansible. Please see
\href{https://github.com/junjuew/ansible-dotfiles/}{repo}. Use following
command to set up GPU passthrough.

\begin{Shaded}
\begin{Highlighting}[]
\KeywordTok{ansible-playbook} \NormalTok{-i hosts-gpu-passthrough gpu-passthrough-playbook.yml}
\end{Highlighting}
\end{Shaded}

\subsubsection{Container Access to GPU}\label{container-access-to-gpu}

nvidia-docker enables containers to access GPU easily. See
\href{https://github.com/junjuew/ansible-dotfiles/}{repo} for
installation.

\subsection{Performance Overhead}\label{performance-overhead}

\textbf{In all of our benchmark measurements, the overheads introduced
by virtualization are between 0\% and 2.3\%.}

We used two benchmarks to measure the overhead introduced by VM and
container virtuation. The first benchmark
\href{https://github.com/baidu-research/DeepBench}{DeepBench} is
compute-intensive. In particular, we focused on convolution operation
--- the core workload of convolutional neural networks. The second
benchmark
\href{https://github.com/parallel-forall/code-samples/blob/master/series/cuda-cpp/optimize-data-transfers/bandwidthtest.cu}{BandwidthTest}
evaluates the data transfer bandwidth between the host and the GPU.

\subsubsection{HW \& SW Setup}\label{hw-sw-setup}

The GPU in test is NVIDIA Tesla GTX 1080 Ti GPU.

\begin{itemize}
\tightlist
\item
  Max GPU Clock: 1911 MHz(Graphics), 5505 MHz(Memory)
\item
  Default Computing mode
\end{itemize}

The software in bare-metal, VM, and container-inside-VM is kept the
same.

\begin{itemize}
\tightlist
\item
  Ubuntu 16.04
\item
  linux kernel 4.4.0-130
\item
  396.37 NVIDIA driver + cuda 9.0 + cudnn 7.1.4.18
\end{itemize}

The VM is created using qemu-kvm 2.6.2 and libvirt. GPU passthrough is
achieved through vfio. The container-inside-VM is created using
nvidia-docker 2.0.3 and docker 18.03.1.

\subsubsection{Convolution Kernels ---
Compute}\label{convolution-kernels-compute}

We used floating point general matrix multiplication from
\href{https://github.com/baidu-research/DeepBench}{this benchmark}. The
benchmark is invoked with

\begin{Shaded}
\begin{Highlighting}[]
\KeywordTok{./conv_bench} \NormalTok{inferenct float}
\end{Highlighting}
\end{Shaded}

The benchmark uses \emph{cudnnFindConvolutionForwardAlgorithm} in cudnn
to determine the convolution algorithm to use at runtime. In our
experiments, such dynamic algorithm selection results in large variance
of execution time as different algorithms are used across different
runs. It is unclear why cudnn would select different algorithms even
when convolution parameters are kept the same. To obtain reproducible
results, we manuallly fixed the convolutional algorithm to be
CUDNN\_CONVOLUTION\_FWD\_ALGO\_IMPLICIT\_PRECOMP\_GEMM in the code. See
\href{https://docs.nvidia.com/deeplearning/sdk/cudnn-developer-guide/index.html\#api-introduction}{here}
for more on what the algorithm does.

We used the same convolutional kernels as described in
\href{https://github.com/baidu-research/DeepBench}{DeepBench} ``Server
Inference Setup'' for comparison.

\paragraph{Convolution Experiment
Parameters}\label{convolution-experiment-parameters}

\begin{longtable}[c]{@{}llllll@{}}
\toprule
Experiment & Input Size & Filter Size & \# of Filters & Padding (h, w) &
Stride (h, w)\tabularnewline
\midrule
\endhead
1 & W = 341, H = 79, C = 32, N = 4 & R = 5, S = 10 & 32 & 0,0 &
2,2\tabularnewline
2 & W = 224, H = 224, C = 3, N = 1 & R = 7, S = 7 & 64 & 3, 3 & 2,
2\tabularnewline
3 & W = 56, H = 56, C = 256, N = 1 & R = 1, S = 1 & 128 & 0, 0 & 2,
2\tabularnewline
4 & W = 7, H = 7, C = 512, N = 2 & R = 1, S = 1 & 2048 & 0, 0 & 1,
1\tabularnewline
\bottomrule
\end{longtable}

\paragraph{Convolution Speed}\label{convolution-speed}

\begin{longtable}[c]{@{}ccccc@{}}
\toprule
Virtualization & Exp 1 (us) & Exp 2 (us) & Exp 3 (us) & Exp 4
(us)\tabularnewline
\midrule
\endhead
bare-metal & 381 +- 9 & 44 +- 1 & 39 +- 1 & 68 +- 1\tabularnewline
VM & 384 +- 11 & 45 +- 1 & 39 +- 1 & 67 +- 1\tabularnewline
Container inside VM & 386 +- 9 & 45 +- 1 & 39 +- 1 & 68 +-
2\tabularnewline
\bottomrule
\end{longtable}

\subsubsection{Bandwidth Test ---
Bandwidth}\label{bandwidth-test-bandwidth}

We benchmarked memory bandwidth between the host and the GPU device
using
\href{https://github.com/parallel-forall/code-samples/blob/master/series/cuda-cpp/optimize-data-transfers/bandwidthtest.cu}{bandwidthtest.cu}.
You can learn more about pinned transfer bandwdith
\href{https://devblogs.nvidia.com/how-optimize-data-transfers-cuda-cc/}{here}.

\paragraph{Pinned Transfer Bandwidth}\label{pinned-transfer-bandwidth}

\begin{longtable}[c]{@{}ccc@{}}
\toprule
\begin{minipage}[b]{0.16\columnwidth}\centering\strut
Virtualization
\strut\end{minipage} &
\begin{minipage}[b]{0.39\columnwidth}\centering\strut
Host To Device Bandwidth (GB/s)
\strut\end{minipage} &
\begin{minipage}[b]{0.36\columnwidth}\centering\strut
Device to Host Bandwidth (GB/s)
\strut\end{minipage}\tabularnewline
\midrule
\endhead
\begin{minipage}[t]{0.16\columnwidth}\centering\strut
bare-metal
\strut\end{minipage} &
\begin{minipage}[t]{0.39\columnwidth}\centering\strut
6.17 +- 0.00
\strut\end{minipage} &
\begin{minipage}[t]{0.36\columnwidth}\centering\strut
6.67 +- 0.01
\strut\end{minipage}\tabularnewline
\begin{minipage}[t]{0.16\columnwidth}\centering\strut
VM
\strut\end{minipage} &
\begin{minipage}[t]{0.39\columnwidth}\centering\strut
6.13 +- 0.00
\strut\end{minipage} &
\begin{minipage}[t]{0.36\columnwidth}\centering\strut
6.66 +- 0.00
\strut\end{minipage}\tabularnewline
\begin{minipage}[t]{0.16\columnwidth}\centering\strut
Container inside VM
\strut\end{minipage} &
\begin{minipage}[t]{0.39\columnwidth}\centering\strut
6.12 +- 0.02
\strut\end{minipage} &
\begin{minipage}[t]{0.36\columnwidth}\centering\strut
6.58 +- 0.02
\strut\end{minipage}\tabularnewline
\bottomrule
\end{longtable}

\subsubsection{Experiments Data}\label{experiments-data}

Complete experiment results are in \url{data} directory.

\begin{itemize}
\tightlist
\item
  baremetal-conv, vm-conv, container-conv: Convolution kernel benchmark
  results for bare-metal, vm, and container-inside-VM.
\item
  run.sh: Convolution kernel result summary script
\item
  bandwidth-test: BandwithTest benchmark results for bare-metal, vm, and
  container-inside-VM.
\item
  conv-dynamic-algorithm: Unmodified convolution kernel benchmark
  results, which select convolution algorithms at runtime.
\item
  gemm-test: GEMM kernel benchmark results using DeepBench. Note that
  the data has high variance since not enough runs are executed.
\item
  vm-ssd-*: Object detection benchmark results on VM using
  \href{www.github.com/junjuew/cvutils}{cvutils}. Note that this
  benchmark includes extra processing time on CPU as well. It should not
  be used for measuring virtualization overhead.
\end{itemize}

\section{Gabriel Deployment}
\section{Discussion}
\chapter{Conclusion and Future Work}

This dissertation addresses the problem of \textit{scaling wearable cognitive
assistance} for widespread deployment. We propose system optimizations that
reduce network consumption, leverage application characteristics to adapt client
behaviors, and provides an adaptation-centric cloudlet resource management
mechanism to better serve many clients in an over-subscribed cloudlet. In
addition, we design and develop a suite of development tools that lower the
barrier of entry and improve developer productivity. In this chapter, we
conclude the dissertation with a summary of contributions, and discuss future
research directions and challenges in this area.

\section{Contributions}

This dissertation claims the following thesis.

\noindent\fbox{%
    \parbox{\textwidth}{%
\textbf{Two critical challenges to the widespread adoption of wearable cognitive
  assistance are 1) the need to operate cloudlets and wireless network at low
  utilization to achieve acceptable end-to-end latency 2) the level of specialized
  skills and the long development time needed to create new applications. These
  challenges can be effectively addressed through system optimizations,
  functional extensions, and the addition of new software development tools to
  the Gabriel platform.}
    }%
}

To validate this thesis, we first provide example wearable cognitive assistance
and present measurement studies on how they would saturate existing wireless
network bandwidth. We propose two application agnostic techniques to reduce
network transmission. Then, leveraging WCA application characteristics, we
proposed an adaptation taxonomy and demonstrated techniques to reduce offered
load on the client device. With these adaptation mechanism, we design and
evaluate a cloudlet resource allocation mechanism that take advantages of
application degradation profiles. In the end, we propose a new application
development methodology and provide a suite of tools to reduce development
difficulty and speed up application development process.

\subsection{Application Agnostic Methods to Reduce Network Transmission}

Wearable cognitive assistance, as a latency-sensitive analytics applications
poses several difficult mobile computing challenges that arise in performing
real-time video analytics on small wearable devices. These challenges lie at the
intersection of wireless bandwidth, result accuracy, and timeliness of results.

To address these challenges, we have developed an adaptive computer vision
pipeline for reducing network transmission without application assistance.  We
explore an early discard strategy to selectively send the most interesting
frames and reduce precious bandwidth between the drone and a ground-based
cloudlet. We propose just-in-time learning to further improve bandwidth
efficiency. Our experimental results show that this judicious combination of
client-side processing and edge-based processing can save substantial wireless
bandwidth and thus improve scalability, without compromising result accuracy or
result latency. 

\subsection{Application-Aware Techniques to Reduce Offered 
Load and Adaptation Centric Cloudlet Resource Management}

More than a decade ago, the emergence of cloud computing led to the
realization that applications had to be written in a certain way to
take full advantage of elasticity of the cloud.  This led to the
concept of ``cloud-native applications'' whose scale-out capabilities
are well matched to the cloud, as well as tools and techniques to
easily create such applications.

The emergence of edge computing leads to another inflection point in
application design.  In particular, it leads to ``edge-native
applications'' that are deeply dependent on attributes such as low
latency or bandwidth scalability that can only be obtained at the edge.
However, as this paper has shown, edge-native applications have to be 
written in a way that is very different from cloud-native applications
if they are to be scalable.

This is the first work to show that cloud-native implementation
strategies that focus primarily on dynamic scale-out are unlikely to
be effective for scalability in edge computing.  Instead, edge-native
applications need to adapt their network and cloudlet resource demand
to system load.  As the total number of Tier-3 devices associated with
a cloudlet increases, the per-device network and cloudlet load has to
decrease.  This is a fundamental difference between cloud-native and
edge-native approaches to scalability. 

We explore client workload reduction and server resource allocation to manage
application quality of service in the face of contention for cloudlet resources.
We demonstrate that our system is able to ensure that in overloaded situations,
a subset of users are still served with good quality of service rather than
equally sharing resources and missing latency requirements for all.

% \subsection{Cloudlet Resource Management for Graceful Degradation of Service}
\subsection{Wearable Cognitive Assistance Development Tools}

Wearable cognitive assistance used to be difficult to develop. We propose a
unifying development methodology to streamline the development process. Together
with the methodology, we have built and provided a suite of development tools
that lower the barrier of application development and speed up the
implementation process. With these tools, we have shown the productivity
improvement can be 10x-20x for application developers.

\section{Future Work}

\subsection{More Sophisticated Computer Vision For Cognitive Assistance}
Recent development in computer vision has shown potential to use more
sophisticated computer vision algorithms for cognitive assistance, such as
activity recognition. Leveraging these algorithms can be useful for creating
cognitive assistants that are more sophisticated.

\subsection{Fine-grained Resource Management}
This work serves as an initial step towards practical resource management for
edge-native applications. There are many potential directions to explore further
in this space. We have alluded to some of these earlier in the paper. One
example we briefly mentioned is dynamic partitioning of work between Tier-3 and
Tier-2 to further reduce offered load on cloudlets.  In addition, other resource
allocation policies, especially fairness-centered policies, such as max-min
fairness and static priority can be explored when optimizing overall system
performance. These fairness-focused policies could also be used to address
aggressive users, which are not considered in this paper.  While we have shown
offline profiling is effective for predicting demand and utility for WCA
applications, for a broader range of edge-native applications, with ever more
aggressive and variable offload management, online estimation may prove to be
necessary. Another area worth exploring is the particular set of control and
coordination mechanisms to allow cloudlets to manage client-offered load
directly. Finally, the implementation to date only controls allocation of
resources but allows the cloudlet operating system to arbitrarily schedule
application processes.  Whether fine-grained control of application scheduling
on cloudlets can help scale services remains an open question.



\backmatter

%\renewcommand{\baselinestretch}{1.0}\normalsize

% By default \bibsection is \chapter*, but we really want this to show
% up in the table of contents and pdf bookmarks.
\renewcommand{\bibsection}{\chapter{\bibname}}
%\newcommand{\bibpreamble}{This text goes between the ``Bibliography''
%  header and the actual list of references}
\bibliographystyle{plainnat}
\bibliography{thesis} %your bib file

\end{document}
