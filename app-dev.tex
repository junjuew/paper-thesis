\chapter{Wearable Cognitive Assistance Development Tools}

\label{chapter: dev}

While previous chapters addresses the system challenges of scaling wearable
cognitive assistance at the edge, another key problem to the widespread adoption
of WCAs is the level of specialized skills and the long development time needed
to create new applications. The expertise needed to build a WCA application
include task domain knowledge, software development, and computer vision.
Researchers~\cite{chen2018application} report that it typically takes multiple
person-months of effort to create a single application. The majority of
development time is spent on learning the computer vision techniques, selecting
robust algorithms to use through trial and error, and implementing the
application. This high barrier of entry significantly limit the number of
wearable cognitive assistants available today. This is clearly not scalable. 

In this chapter, we reflect on WCA development procedures, propose a new
development methodology centered around DNN-based object detection, and
introduce implement development tools to lower the barrier of WCA development.
Our goal is to simplify the process so that a small team (1-2 people) of a task
expert and a developer, without computer vision expertise, can create an initial
version of a Gabriel application within a few days. This is a productivity
improvement of at least one order of magnitude. Refinement and tuning may take
longer, but can be guided by early use of the application. 

Simplification is difficult. The application needs a precise definition of the
end point of each step (e.g., a particular screw mounted flush into a
workpiece), yet needs to be tolerant of alternative paths to reaching that end
point (e.g., hand-tightening versus using a screwdriver). We assume the small
development team has knowledge of the task specifics and general programming
skills, but not necessarily experiences with wearables and machine learning.

\section{Ad Hoc WCA Development Process}

\begin{figure*}
  \centering
  \includegraphics[trim={0 10cm 0 0},width=\linewidth]{FIGS/ad-hoc-workflow}
  \hspace{-0.50in}
	\caption{Development Workflow}
    \vspace{-0.0in}
    \label{fig:workflow}
\end{figure*}

The overall development workflow of wearable cognitive assistance is shown in
figure~\ref{fig:workflow}. After a use case is identified, developers would need
to identify meaningful visual states that can be detected using computer vision.
In the meantime, a task is divided into steps based on the use case and
detectable visual states. Task procedures could be changed to reduce the
difficulties of CV checks. In fact, since there is a human in the loop, relying
on humans to do what they are good at is the main reason that wearable cognitive
assistance can be implemented without solving perception and planning problems
intrinsic to robotics. Task procedures together with error states form a task
model. Developers implement the application according to the task model. After
 initial implementation, test runs and measurements are conducted to evaluate
the robustness of computer vision checks and end-to-end application latency.
This process is iterated until developers are satisfied.

Among all the development procedures, creating the computer vision checks to
detect user states consumes the most of development time and requires computer
vision expertise and experience. With the adoption of DNNs, developers no longer
need to spend days to select and tweak handcrafted features. Instead, the entire
model is trained end-to-end using labeled data. However, DNNs, with millions of
parameters to train, requires a significant amount of training data. Collecting
and labeling data are time-consuming and painstaking. Besides, to craft and
implement a DNN by hand is not trivial. Significant machine learning background
is needed to tweak network architectures and parameters. Therefore, developer
tools are needed to both help label the data and create deep neural networks.

In summary, implementing the workflow of cognitive assistance takes time and
efforts. Ad-hoc implementation requires a team of domain experts, developers and
computer vision experts. Such development model cannot scale to thousands of
applications. Therefore, Gabriel needs to be extended with tools to reduce the
effort of creating wearable cognitive assistants.

Existing ad-hoc approach to develop wearable cognitive assistance not only takes
a long time, but also requires computer vision expertise. A developer new to
wearable cognitive assistance would need to spend months learning computer
vision basics and acquire intuitions to determine what is achievable before
developing an application. For instance, a researcher mentions the first
application developed to help a user assemble LEGO pieces took him more than
four months.

Figure~\ref{fig:workflow} shows the ad-hoc development process. The most
critical step in building wearable cognitive assistance is to identify task
steps and the visual states of task steps. For example, for the Lego wearable
cognitive assistance ~\cite{chen2017empirical}, the task steps are the sequence
of shapes needed to achieve the final assembled lego shape. The visual states to
recognize are the current shapes on the lego board. Identifying visual states
and task steps takes significant time and expertise due to several reasons.
First, a developer needs to be familiar with the state-of-art of computer vision
algorithms to determine what visual states can be recognized reliably. Second,
identifying task steps requires domain knowledge. Third, when visual states
become too hard for CV, developers need to adjust the task steps to use other
methods for confirmation. Often a redesign of task steps is required to
compensate computer vision. For instance, when designing the RibLoc application,
a redesign of the task steps involves asking the user to read out a word on the
gauge instead of performing optical character recognition on the lightly-colored
characters.

\section{Object-Detection Centered Development Process}

Object detection is at the core of computer vision tasks used by Gabriel
application. In Ping-Pong assistance, the Ping-Pong table, the ball, and the
opponent need to be recognized and localized. In Ikea Lamp assistance, the lamp
base, the shade, and the bulb need to be detected. Being able to create reliable
object detectors quickly can substantially facilitate application development.

Recent advances in
DNNs~\cite{girshick2014rich},~\cite{ren2015faster},~\cite{he2016deep} have not
only drastically improved the accuracy of object detection, but also provide an
opportunity to automate the creation of them. Unlike traditional CV algorithms,
DNNs adopt a end-to-end learning approach, in which features are no longer
hand-crafted but learned. The replacement of custom CV code with machine-learned
models gives automation opportunities. Nevertheless, creating a DNN-based object
detector is still both time-consuming and painstaking due to other reasons. DNNs
have a lot of parameters and requires millions of labeled examples to train from
scratch. Collecting and labeling these large amount of training data becomes a
bottleneck.

\section{Tools For Painless Object Detection (TPOD)}
\label{sec: app-dev-tpod}

TPOD (Tool for Painless Object Detection) is a web-based tool I developed to
help quickly create DNN-based object detectors. It provides a tracking-assisted
labeling interface for speedy labeling and transfer learning-based DNN training
and evaluation backends that abstract the nuances of DNNs. Using TPOD to create
object detectors is straight-forward. A user would first upload short videos of
the object collected from varying lighting conditions and perspectives. Then,
the user would label these objects using TPOD's labeling interface. TPOD assists
labeling by tracking the labeled object across frames. Augmenting training data
with synthetically generated data is also supported. A user then can start training
from the web interface. TPOD backend uses transfer learning to finetune an
object detector DNN from publicly available networks that have been trained with
millions of images. When the training is done, a user can download the object
detector as a container image to run the trained models for inference. TPOD also
provides interfaces for evaluating and testing trained DNNs.

For a wide range of real-world objects, deep convolutional neural networks
have been shown to be effective at accurate detection and identification of
different objects in natural photographs.  They have been demonstrated to
differentiate between very subtly different classes, such as breeds of
dogs, and have been shown to approach human-level accuracies in such tasks.
Therefore, we believe a DNN-based approach should suffice to clearly
identify the goal states of each step in the task workflow.

Training a DNN for such image detection tasks is not a trivial process.  In
particular, it involves constructing a correctly-labeled training data set
with millions of positive and negative examples.  The training process
itself may take days to complete, and involves a set of arcane procedures
to ensure both convergence and efficacy of the model.  Fortunately, one
does not typically need to train a DNN from scratch.  Rather, one can start
with a pretrained model based on a public image data set such as ImageNet,
and then adapt it to detect custom classes for a new application, though a
process called \emph{transfer learning}.  The key idea is that much of the
training teaches the model to discover low-level features, such as edges,
textures, shapes, and patterns that are useful in distinguishing objects,
and such features can largely be reused in other detection tasks on similar
images.  Thus, adapting a pretrained model for new object classes requires
only thousands of examples and hours of training time.

However, even with transfer learning, collecting a labeled training set of a
thousand examples per object class can be a daunting and painful task. TPOD
helps to greatly reduce the labeling effort in construction of the dataset, and
automates and takes some of the guesswork out of training a DNN model.

With TPOD, the developer first captures videos of the desired object from
all different viewing angles, and against multiple backgrounds.  These can
simply be taken using a cellphone camera.  A few minutes of video will
contain several thousand frames containing the object of interest.

Next, the videos are imported into the web-based TPOD tool.  TPOD allows the
user to quickly switch between videos and label objects.  As illustrated in
Figure~\ref{fig:tpod_gui} A, for a given video, the user labels the first
occurrence of the object by drawing a bounding box around it. The key advantage
of TPOD is that the user does not need to perform this labeling action in every
frame.  Rather, TPOD leverages the fact that the frames are part of a continuous
video shot, and automatically places the bounding box in subsequent frames using
a correlation tracking algorithm~\cite{danelljan2014accurate}. Thus, the user
does not need to do anything for these frames, and can skip through them
quickly.  Of course, tracking is not perfect, and the bounding box may drift off
the object over time. In this case, the user can scroll forward or back in the
video, and adjust the bounding box to better match the object.  This
reinitializes the tracking for subsequent frames. Overall, this approach of
labeling followed by tracking can reduce the number of frames that the user
needs to manually label by a factor of 10--20x.


TPOD then performs a data cleaning and augmentation pass.  Because of interframe
correlations, many of the object examples may be close to identical.  TPOD will
eliminate the near duplicate examples, as these will not help in the training
process.  Optionally, data augmentation can be employed.  This adds synthetic
images, created by pasting the object samples on varying backgrounds, at
different scales and rotations.  Such augmentation has been shown to help
produce more robust object detectors.

Finally, TPOD can automate the transfer learning of a DNN model using the
collected dataset with the GUI shown in Figure~\ref{fig:tpod_gui} B.  By
default, TPOD uses a state-of-the-art FasterRCNN-VGG network pre-trained on the
Pascal VOC dataset~\cite{Everingham15}, though other network architectures can
be used as well. Negative examples are mined  from the video background; these
are parts of the frames not included in the object bounding boxes.  The training
is started as a batch process that uses a standard, scripted learning schedule,
and generates both the final Tensorflow model, as well as a Docker container
with the executable detector.  The TPOD web interface provides notification and
download links for the generated model files and containers once training is
complete.

Overall, TPOD can greatly reduce both the labeling effort and in-depth machine
learning knowledge needed to effectively train and deploy a DNN-based detector.
We note that TPOD is largely a stand-alone tool, and can be used separately from
the rest of the system outlined in this paper.  As described here, TPOD is used
after workflow extraction to train detectors to find the ends of the steps.
Alternatively, if the developer already has a good sense of the needed object
detectors, TPOD can be used first, and the resulting detectors used in the
automatic extraction stage to produce more accurate task workflows.


The initial prototype of TPOD has been used by researchers and students to build
wearable cognitive assistance. For example, a group of master students in CMU
mobile and pervasive computing class successfully used TPOD to build an
assistant for using AED machines.

\section{Finite State Machine (FSM) Authoring Tools}
\label{sec: app-dev-fsm}

In addition to creating DNN-based object detectors, developers need to write
custom logic to implement the WCA task model running on the cloudlet. In this
section, we introduce a FSM authoring tool that provides libraries and a GUI to
enable fast implementation to allow for quick development iteration.

As discussed in Section~\ref{sec: app-dev-fsm-representation}, the WCA cloudlet
logic can be represented as a finite state machine. The FSM representation
allows us to impose structure and provide tools for task model implementation.
Our FSM authoring tool consists of a web GUI that allows users to visualize and
edit a WCA FSM within a browser, a python library that supports the creation and
execution of a FSM, and a binary file format that efficiently stores the FSM.
The FSM authoring tool video demo can be found at \url{https://youtu.be/TgHOVF0ktVs}.

\subsection{FSM Editor}

\begin{figure}
    \centering
    \fbox{\includegraphics[trim={0 0 0 0},width=\linewidth]{FIGS/fsm-web-gui}}
	\caption{FSM Web GUI}
    \label{figs:fsm-web-gui}
\end{figure}

Figure~\ref{figs:fsm-web-gui} shows our FSM Web Editor. Users can create a WCA
FSM from the GUI by editing states and transitions. State processors, e.g. the
computer vision processing logic to run in a given state, can be specified by a
container url. User guidance can be added through adding text, video urls or by
uploading images. The Web GUI also supports import and export functionalities to
interface with other WCA tools. The exported FSM is in a custom binary format
and can be executed by the FSM python library.

The GUI is implemented as a pure browser-based user interface, using
React~\cite{staff2016react}. No web backend is needed. This makes the GUI easy
to set up and deploy. The user only needs to open an HTML file in a browser to 
use the tool.

\subsection{FSM Python Library}

Another way to programmatically create a FSM is through the FSM Python library.
The library provides python APIs to create and modify FSMs. The python APIs
provides additional interfaces to add custom computer vision processing as
functions and ad hoc transition predicates for customization. 

In addition, the Python library provides a state machine executor that takes a
WCA FSM (e.g. made with the Web GUI) and launches a WCA program using the
Gabriel platform. The program is then ready to be connected by Gabriel Android
Client. The WCA program follows the logic defined in the state machine. A
Jupyter Notebook~\cite{kluyver2016jupyter} is also provided to make it possible to launch the program from
a browser. This library has been made available on The Python Package Index for
easy installation.

\subsection{FSM Binary Format}

We define a custom FSM binary format for WCAs that can be read and wrote using
multiple programming languages. We use the serialization library Protocol
Buffers to generate language-specific stub code.
Figure~\ref{figs:fsm-binary-format} shows a summary of the serialization format.


\begin{figure}
\small
\begin{lstlisting}
// represents the trigger condition
message TransitionPredicate {
  string name = 1;
  string callable_name = 2; 
  map<string, bytes> callable_kwargs = 3; // arguments
  string callable_args = 4; // arguments
}

message Instruction {
  string name = 1;
  string audio = 2; // audio in text format.
  bytes image = 3;
  bytes video = 4;
}

message Transition {
  string name = 1;
  // function name of the trigger condition
  repeated TransitionPredicate predicates = 2;
  Instruction instruction = 3;
  string next_state = 4;
}

// represent feature extraction modules
message Processor {
  // input are images
  // outputs are key/value pairs that represents application state
  string name = 1;
  string callable_name = 2; 
  map<string, bytes> callable_kwargs = 3; // arguments
  string callable_args = 4; // arguments
}

message State {
  string name = 1;
  repeated Processor processors = 2; // extract features
  repeated Transition transitions = 3;
}

message StateMachine {
  string name = 1;
  repeated State states = 2; // all states
  map<string, bytes> assets = 3; // shared assets
  string start_state = 4;
}
\end{lstlisting}
\caption{FSM Binary Format}
\label{figs:fsm-binary-format}
\end{figure}

% We observed that the
% implementation of a cognitive assistant mainly consists of components to
% identify user states using computer vision models, specify instructions to users
% based on the current state, and keep track of progress on the task. We created a
% a workflow modeling tool, SME, which can transform the process of completing a
% task into a specific model, substantially reducing the expert modeling effort
% required. By using a framework to reason about step-by-step application, along
% with workflow information from WE and object recognition models from OpenTPOD, SME
% also automatically generates a Gabriel executable application based on the
% modeled workflow.

% The applications SME generates are defined by a set of steps. Each state
% corresponds to the subset of steps that a user has completed. We use a FSM to
% keep track of the current state of a task and the future states that we should
% transition to when a user takes a certain action. Transitions among states are
% triggered by visual changes resulting from user actions. For example, the user
% might put a piece of ham on top of a piece of bread. In addition, each state has
% its own processing functions, which convert the current visual state into
% information that the application can understand. We provide a list of
% pre-defined processing functions, such as object detection DNNs. Task experts
% also have the flexibility to add custom processing functions when needed.
% Figure~\ref{fig:sme} shows a sample workflow as a finite state machine. Once the
% task expert finalizes a FSM, SME automatically compile the FSM to generate an
% executable application.
\section{Discussion}
