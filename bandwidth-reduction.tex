\chapter{Application-Agnostic Techniques to Reduce Network Transmission}
\label{chapter: bandwidth}

WCAs continuously stream sensor data to a cloudlet. The richer a sensing
modality is, the more information can be extracted. The core sensing modality of
WCAs is visual data, e.g. egocentric images and videos from wearable cameras.
Compared to other sensors, e.g. microphones and inertial measurement units
(IMUs), cameras capture visual information with rich semantics. As commercial
camera hardware become more affordable, they have become increasingly pervasive.
In 2013, it is estimated that there is one security camera for every eleven
citizens in the UK~\cite{Barrett2013}. In the meantime, deep neural networks
(DNNs) have driven significant advancement in computer vision and have achieved
human-level accuracy in several previously intractable perception problems (e.g.
face recognition, image classification)~\cite{learned2016labeled,
    schroff2015facenet}. The richness and the open-endness of visual data makes
camera the ideal sensor for WCAs.

However, continuous video transmission from many Tier-3 devices places severe
stress on the wireless spectrum.  Hulu estimates that its video streams require
13 Mbps for 4K resolution and 6 Mbps for HD resolution using highly optimized
offline encoding~\cite{Hulu2017}. Live streaming is less bandwidth-efficient, as
confirmed by our measured bandwidth of 10 Mbps for HD feed at 25 FPS. Just 50
users transmitting HD video streams continuously can saturate the theoretical
uplink capacity of 500 Mbps in a 4G LTE cell that covers a large rural
area~\cite{LteWorld2009}.  This is clearly not scalable.

In this chapter, we show how per-user bandwidth demand in WCA-like live video
analytics can be significantly reduced using an application-agnostic approach.
We aim to reduce bandwidth demand without compromising the timeliness or
accuracy of results. In contrast to previous
works~\cite{Wang2017networked,zhang2015design,Wang2016skyeyes}, we leverage
state-of-the-art deep neural networks (DNNs) to selectively transmit interesting
data from a video stream and explore environment-specific optimizations. The
accuracy of the data selection is important, as fewer false positives result in
lower network bandwidth and cloudlet computing cycle consumption.

This chapter is organized as follows. We first discuss the challenges of running
DNNs for visual perception solely on Tier-3 devices in
Section~\ref{bw:challenges}. Next, we propose and compare two
application-agnostic techniques to reduce network transmission. We present our
results first in the context of live video analytics for small autonomous
drones. Both as emerging Tier-3 devices, drones and wearable devices face
similar challenges in live video analysis. Finally, we showcase how these
techniques can be applied to WCAs in Section~\ref{bw:wca}.

\section{Video Processing on Mobile Devices}
\label{bw:challenges}

In the context of real-time video analytics, Tier-3 devices represent
fundamental mobile computing challenges that were identified two decades
ago~\cite{Satya1996}.  Two challenges have specific relevance here. First,
mobile elements are resource-poor relative to static elements.  Second, mobile
connectivity is highly variable in performance and reliability.  We discuss
their implications below.


\subsection{Computation Power on Tier-3 Devices}
\label{bw:payload}

Unfortunately, the hardware needed for deep video stream processing in real time
is larger and heavier than can fit on a typical Tier-3 device. State-of-art
techniques in image processing use DNNs that are compute- and memory-intensive.
Figure~\ref{fig:onboard-dnn-speed} presents experimental results on two
fundamental computer vision tasks, image classification and object detection, on
four different devices. In the figure, MobileNet V1 and ResNet101 V1 are image
classification DNNs. Others are object detection DNNs. Both tasks used publicly
available pretrained DNN models. We carefully choose hardware platforms to
represent a range of computation capabilities of Tier-3 devices. To anticipate
future improvements in smartphone technology, our experiments also consider more
powerful devices such as the Intel$^{\tiny\textregistered}$
Joule~\cite{Hardawar2016} and the NVIDIA Jetson~\cite{NVIDIA2017} that are
physically compact and light enough to be credible as wearable device platforms
in a few years.

Fig.~\ref{fig:onboard-dnn-speed}, we present the best results we could obtain on
each platform. This is not intended to directly compare frameworks and platforms
(as others have been doing~\cite{Zhang2018pcamp}), but rather to illustrate the
differences between wearable platforms and fixed infrastructure servers. 

\begin{figure}
\centering
\begin{flushleft}
M: MobileNet V1; R: ResNet101 V1;
S-M: SSD MobileNet V1; S-I: SSD Inception V2;\\F-I: Faster R-CNN Inception V2;
F-R: Faster R-CNN ResNet101 V1
\end{flushleft}
\sisetup{
    table-format=3,
    table-number-alignment=left
}
\hspace{-0.8in}
\begin{tabular}{|p{1.3cm}|S[table-column-width=1.3cm]|p{3.0cm}|p{1.8cm}|S[table-column-width=1.2cm]|S[table-column-width=1.4cm]|S[table-column-width=1.3cm]|S[table-column-width=1.3cm]|S[table-format=4, table-number-alignment=left, table-column-width=1.5cm]|S[table-format=5, table-number-alignment=left, table-column-width=1.9cm]|}
\hline
{\multirow{2}{*}{}} & {\multirow{2}{*}{}} & {\multirow{2}{*}{}}                                        & {\multirow{2}{*}{}}                                                                        & \multicolumn{2}{c|}{\parbox[t]{1.8cm}{\centering Image\\Classification\\~}}                                & \multicolumn{4}{c|}{Object Detection} \\ \cline{5-10}
                  &  {\parbox[t]{0.9cm}{\centering Weight\\(g)}}
                  & \centering CPU
                  & \centering GPU
                  & {\parbox[t]{0.9cm}{\centering M\\(ms)}}
                  & {\parbox[t]{1.1cm}{\centering R\\(ms)}}
                  & {\parbox[t]{1.0cm}{\centering S-M\\(ms)}}
                  & {\parbox[t]{1.0cm}{\centering S-I\\(ms)}}
                  & {\parbox[t]{1.3cm}{\centering F-I\\(ms)}}
                  & {\parbox[t]{1.6cm}{\centering F-R\\(ms)}} \\ \hline
Nexus~6    & 184                                                      & 4-core 2.7 GHZ Krait 450, 3GB Mem                           & Adreno 420                                                                              & 353 {\scriptsize \ (67)}                                                 & 983 {\footnotesize \ (141)}                                                   & 441 {\footnotesize \ (60)           }                               & 794 {\footnotesize \ (44)            }                                       & {\small ENOMEM}                                                                 & {\small ENOMEM}                                                               \\ \hline
Intel$^{\tiny \textregistered}$ Joule 570x     & 25                                                       & 4-core 1.7 GHz Intel Atom$^{\tiny\textregistered}$ T5700, 4GB Mem                          & Intel$^{\tiny\textregistered}$ HD Graphics (gen 9)                                                               & 37 {\scriptsize \ (1)$\ddagger$ }                                       & 183 {\footnotesize \ (2)$\dagger$$\ddagger$ }                                        & 73  {\footnotesize \ (2)$\ddagger$ }                               & 442 {\footnotesize \ (29)            }                                       & 5125 {\footnotesize \ (750)}                                                             & 9810 {\footnotesize \ (1100)}                                                          \\ \hline
NVIDIA Jetson TX2     & 85                                                       & 2-Core 2.0 GHz Denver2 + 4-Core 2.0 GHz Cortex-A57, 8GB Mem & 256 cuda core 1.3 GHz NVIDIA Pascal                                                          & 13 {\scriptsize \ (0)$\dagger$  }                                       & 92 {\footnotesize \ (2)$\dagger$ }                                          & 192 {\footnotesize \ (18)           }                               & 285 {\footnotesize \ (7)$\dagger$   }                                       & {\small ENOMEM}                                                                 & {\small ENOMEM}                                                               \\ \hline
Rack-mounted Server     &                                                          & 2x 36-core 2.3 GHz Intel$^{\tiny\textregistered}$ Xeon$^{\tiny\textregistered}$ E5-2699v3 Processors, 128GB Mem     & 2880 cuda core 875MHz NVIDIA Tesla K40, 12GB GPU Mem                             & 4 {\scriptsize \ (0)$\ddagger$  }                                       & 33 {\footnotesize \ (0)$\dagger$  }                                          & 12  {\footnotesize \ (2)$\ddagger$ }                               & 70  {\footnotesize \ (6)             }                                       & 229 {\footnotesize \ (4)$\dagger$}                                                               & 438 {\footnotesize \ (5)$\dagger$}                                                               \\ \hline
\end{tabular}
\begin{captiontext}
\vspace{0.1in}
Figures above are means of 3 runs across 100 random images. The time shown
includes only the forward pass time using batch size of 1. ENOMEM indicates
failure due to insufficient memory. Figures in parentheses are standard
deviations. The weight figures for Joule and Jetson include only the modules
without breakout boards. Weight for Nexus~6 includes the complete phone with
battery and screen. Numbers are obtained with TensorFlow (TensorFlow Lite for
Nexus 6) unless indicated otherwise. \\
$\dagger$ indicates GPU is used. $\ddagger$ indicates
Intel$^{\tiny\textregistered}$ Computer Vision SDK beta 3 is used.
\end{captiontext}
\caption{Deep Neural Network Inference Speed}
\label{fig:onboard-dnn-speed}
\end{figure}

Image classification maps an image into categories, with each category
indicating whether one or many particular objects (e.g., a human survivor, a
specific animal, or a car) exist in the image.  The prediction speed using two
different DNNs are shown. MobileNet V1~\cite{Howard2017} is a DNN
designed for mobile devices from the ground-up by reducing the number of
parameters and simplifying the computation using
depth-wise separable convolution. ResNet101 V1~\cite{He2016} is a more accurate
but also more resource-hungry DNN that won the ImageNet
classification challenge in 2015~\cite{Russakovsky15}. 

Object detection is a harder task than image classification, because it requires
bounding boxes to be predicted around the specific areas of an image that
contains a particular class of object. Object detection DNNs are built on top of
image classification DNNs by using image classification DNNs as low-level
feature extractors. Since feature extractors in object detection DNNs can be
changed, the DNN structures excluding feature detectors are referred as object
detection meta-architectures. We benchmarked two object detection DNN
meta-architectures: Single Shot Multibox Detector (SSD)~\cite{Liu2016} and
Faster R-CNN~\cite{Ren2015}. We used multiple feature extractors for each
meta-architecture. The meta-architecture SSD uses simpler methods to identify
potential regions for objects and therefore requires less computation and runs
faster. On the other hand, Faster R-CNN~\cite{Ren2015} uses a separate region
proposal neural network to predict regions of interest and has been shown to
achieve higher accuracy~\cite{Huang2017}. Figure~\ref{fig:onboard-dnn-speed}
presents results in four columns: SSD combined with MobileNet V1 or Inception
V2, and Faster R-CNN combined with Inception V2 or ResNet101 V1~\cite{He2016}.
The combination of Faster R-CNN and ResNet101 V1 is one of the most accurate
object detectors available today~\cite{Russakovsky15}. The entries marked ``{\sc
ENOMEM}'' correspond to experiments that were aborted because of insufficient
memory.

These results demonstrates the computation gap between mobile and static
elements. While the most accurate object detection model Faster R-CNN Resnet101 V1
can achieve more than two FPS on a server GPU, it either takes several seconds
on  mobile platforms or fails to execute due to insufficient memory. In
addition, the figure also confirms that sustaining open-ended real-time video
analytics on smartphone form factor computing devices is well beyond the state
of the art today and may remain so in the near future.  This constrains what is
achievable with Tier-3 devices.

\subsection{Result Latency, Offloading \& Scalability}
\label{bw:offloading}

{\em Result latency} is the delay between first capture of a video frame in
which a particular result (e.g., image of a survivor) is present, and report of
its discovery or feedback based on the discovery after video processing.
Operating totally disconnected, a Tier-3 device can capture and store video, but
defer its processing until the mission is complete.  At that point, the data can
be uploaded from the device to the cloud and processed there.  This approach
completely eliminates the need for real-time video processing, obviating the
challenges of Tier-3 computation power mentioned previously. Unfortunately, this
approach delays the discovery and use of knowledge in the captured data by a
substantial amount (e.g., many tens of minutes to a few hours).  Such delay may
be unacceptable in use cases such as search-and-rescue using drones, or
step-by-step instruction feedback from wearable devices. In this chapter, we focus
on approaches that aim for much smaller result latency: ideally, close to
real-time.

A different approach is to offload video processing during flight over a
wireless link to an edge computing node (cloudlet). With this approach, even a
weak Tier-3 device can leverage the substantial processing capability of a
ground-located cloudlet, without concern for its weight, size, heat dissipation,
or energy usage.  Much lower result latency is now possible.  However, even if
cloudlet resources are viewed as ``free'' from the viewpoint of mobile
computing, the Tier-3 device consumes wireless bandwidth in transmitting video.

Today, 4G LTE offers the most plausible wide-area connectivity from a Tier-3
device to its associated cloudlet.  The much higher bandwidths of 5G are still
many years away, especially at global scale.  More specialized wireless
technologies, such as Lightbridge 2~\cite{LightBridge2} for drones, can also be
used.  Regardless of specific wireless technology, the principles and techniques
described in this chapter apply.

{\em Scalability,} in terms of maximum number of concurrently operating Tier-3
devices within a 4G LTE cell becomes an important metric.  In this chapter we
explore how the limited processing capability on a Tier-3 device can be used to
greatly decrease the volume of data transmitted, thus improving scalability
while minimally impacting result accuracy and result latency.

Note that the uplink capacity of 500 Mbps per 4G LTE cell assumes standard
cellular infrastructure that is undamaged.  In natural disasters and military
combat, this infrastructure may be destroyed. Emergency substitute
infrastructure, such as Google and AT\&T's partnership on balloon-based 4G LTE
infrastructure for Puerto Rico after hurricane Maria~\cite{Morse2017}, can only
sustain much lower uplink bandwidth per cell, e.g. 10Mbps for the balloon-based
LTE~\cite{Sankaran2018}.  Conserving wireless bandwidth from Tier-3 video
transmission then becomes even more important, and the techniques described here
will be even more valuable.

% \emph{Result accuracy} influences a second dimension of scalability, namely
% the ability of one individual to supervise the result streams from
% many drones.  The output of each video processing pipeline should only
% demand occasional human attention.  The accuracy, sophistication, and
% speed of this pipeline determines the cognitive load on mission
% personnel for a given video stream.  For example, a pipeline that has
% virtually no false positives or false negatives in detecting survivors
% will consume less supervisory human attention than a mediocre
% pipeline.  That will allow one person to confidently supervise a large
% swarm that rapidly covers a large search area.

\section{Baseline: {\xc Dumb}  S\lc{trategy}}
\label{sec:dumbdrone}

\subsection{Description}

We first establish and evaluate the baseline case of no image processing
performed at the Tier-3 device.  Instead, all captured video is immediately
transmitted to the cloudlet.  Result latency is very low, merely the sum of
transmission delay and cloudlet processing delay. We use drones as the example
of Tier-3 devices and drone video search as the scenario of video analytics
first. We later demonstrate how to apply the techniques developed for drone
search to WCAs.

\begin{figure}
\centering
\begin{tabular}{|p{1cm}|p{2.5cm}|p{2.5cm}|p{2.5cm}|p{2.5cm}|p{2.5cm}|}
\hline
   & Detection & Data & Data & Training & Testing \\ 
Task& Goal & Source & Attributes & Subset & Subset\\ 
\hline
T1 & {\small People in scenes of daily life}&{\small Okutama Action Dataset~\cite{Barekatain2017}}&\makecell[tl]{\small 33 videos \\\small 59842 fr\\\small 4K@30~fps}&\makecell[tl]{\small 9 videos\\\small 17763 fr}&\makecell[tl]{\small 6 videos\\\small 20751 fr}\\ 
\hline
T2 &{\small Moving cars}&{\small Stanford Drone Dataset~\cite{Robicquet2016}}&\makecell[tl]{\small 60 videos \\\small 522497 fr\\\small 1080p@30~fps}&\makecell[tl]{\small 16 videos\\\small 179992 fr} & \multirow{4}{*}{\parbox{1.5cm}{\centering\small 14 videos 92378 fr\\ Combination of test videos from each dataset.}} \\ \cline{1-5}
  %% \makecell[tl]{\small 14 videos\\\small 92378 fr\\\small Combination of \\\small human, car, raft \\\small and elephant videos\\\small from each datasets}} \\ \cline{1-5}
%% \makecell[tl]{\small 14 videos\\\small 92378 fr}\\ 
%% \hline
T3 &{\small Raft in flooding scene}&{\small YouTube collection~\cite{YouTube1}}&\makecell[tl]{\small 11 videos \\\small 54395 fr\\\small 720p@25~fps}&\makecell[tl]{\small 8 videos\\\small 43017 fr} & \\ \cline{1-5}
%% \makecell[tl]{\small 14 videos\\\small 92378 fr}\\
%% \hline
T4 &{\small Elephants in natural habitat}&{\small YouTube collection~\cite{YouTube2}}&\makecell[tl]{\small 11 videos \\\small 54203 fr\\\small 720p@25~fps}&\makecell[tl]{\small 8 videos\\\small 39466 fr} & \\ \cline{1-5}
%% \makecell[tl]{\small 14 videos\\\small 92378 fr}\\
%% \hline
% T5 &{\small Pushing or pulling Suitcases}&\small Okutama Action Dataset &\small Same as T1 &\small Same as T1 & \\
%% \small Same as T1\\
\hline
\end{tabular}
\vspace{0.1in}
\begin{captiontext}
fr = ``frames''\\
fps = ``frames per second''\\
No overlap between training and testing subsets of data
\end{captiontext}
\caption{Benchmark Suite of Drone Video Traces}
\label{fig:benchmarksuite}
\end{figure}

\begin{figure}
\centering
\begin{tabular}{|c|c|c|c|c|}
\hline
     & Total & Avg & & \\ 
     & Bytes & BW & & \\ 
Task & (MB) & (Mbps) & Recall & Precision \\ 

\hline
T1 & \phantom{0}924 & 10.7 & 74\% & 92\%\\ 
\hline
T2 & 2704 & \phantom{0}7.0 & 66\% & 90\%\\ 
\hline
\end{tabular}
\vspace{0.2in}
\begin{captiontext}
Peak bandwidth demand is same as average since video is transmitted
continuously. Precision and recall are at the maximum F1 score.
\end{captiontext}
\caption{Baseline Object Detection Metrics}
\label{fig:baseline}
\end{figure}

\subsection{Experimental Setup}
\label{sec:dumbdrone-setup}

To ensure experimental reproducibility, our evaluation is based on
replay of a benchmark suite of pre-captured videos rather than on
measurements from live drone flights.  In practice, live results may
diverge slightly from trace replay because of non-reproducible
phenomena.  These can arise, for example, from wireless propagation
effects caused by varying weather conditions, or by seasonal changes
in the environment such as the presence or absence of leaves on trees.
In addition, variability can arise in a  drone's pre-programmed flight
path due to collision avoidance with moving obstacles such as birds,
other drones, or aircraft.

All of the pre-captured videos in the benchmark suite are publicly accessible,
and have been captured from aerial viewpoints. They characterize drone-relevant
scenarios such as surveillance, search-and-rescue, and wildlife conservation.
Figure~\ref{fig:benchmarksuite} presents this benchmark suite of videos,
organized into four tasks. All the tasks involve detection of tiny objects on
individual frames. Although T2 is also nominally about action detection (moving
cars), it is implemented using object detection on individual frames and then
comparing the pixel coordinates of vehicles in successive frames.
% Task T5 additionally involves action detection, which operates on short video segments rather than individual frames. 

\subsection{Evaluation}
\label{sec:dumbdrone-results}

Figure~\ref{fig:baseline} presents the key performance indicators on the object
detection tasks T1 and T2. We use the well-labeled dataset to train and evaluate
Faster-RCNN with ResNet 101. We report the precision and recall at maximum F1
score.  Peak bandwidth is not shown since it is identical to average bandwidth
demand for continuous video transmission.  As shown earlier in
Figure~\ref{fig:onboard-dnn-speed}, the accuracy of this algorithm comes at the
price of very high resource demand.  This can only be met today by server-class
hardware that is available in a cloudlet.  Even on a cloudlet, the figure of 438
milliseconds of processing time per frame indicates that only a rate of two
frames per second is achievable.  Sustaining a higher frame rate will require
striping the frames across cloudlet resources, thereby increasing resource
demand considerably.  Note that the results in
Figure~\ref{fig:onboard-dnn-speed} were based on 1080p frames, while tasks T1
uses the higher resolution of 4K. This will further increase demand on cloudlet
resources.

Clearly, the strategy of blindly shipping all video to the cloudlet
and processing every frame is resource-intensive to the point of being
impractical today.  It may be acceptable as an offline processing
approach in the cloud, but is unrealistic for real-time processing on
cloudlets.  We therefore explore an approach in which a modest amount
of computation on the Tier-3 is able, with high confidence, to avoid
transmitting many video frames and thereby saving wireless bandwidth
as well as cloudlet processing resources.  This leads us to the {\xc
  EarlyDiscard} strategy of the next section.





\section{EarlyDiscard Strategy}
\label{sec:earlydiscard}

\begin{figure}
    \includegraphics[trim={0cm 13cm 14cm 0cm},clip,width=\linewidth]{FIGS/fig-early-discard.pdf}
    \caption{Early Discard on Tier-3 Devices}
    \label{fig:ondrone}
\end{figure}


\subsection{Description}
EarlyDiscard is based on the idea of using on-board processing to filter and
transmit only interesting frames in order to save bandwidth when offloading
computation. Frames are considered to be interesting if they capture objects or
events valuable for processing, for instance, survivors for a search task.
Previous work~\cite{Hu2015,Naderiparizi2017} leveraged pixel-level
features and multiple sensing modalities to select interesting frames from
hand-held or body-worn cameras. In this section, we explore the use of DNNs to
filter frames from aerial views. The benefits of using DNNs are as follows.
First, DNNs, even shallow ones, are capable of understanding some semantically
meaningful visual information. Their decisions of what to send are based on the
reasoning of image content in addition to pixel-level characteristics. Next,
DNNs are trained and specialized for each task, resulting in their high accuracy
and robustness for that particular task. Finally, compared to a sensor fusing
approach that requires other sensing modalities to be present on Tier-3 devices,
no additional hardware is added to the existing platforms.

Although smartphone-class hardware is incapable of supporting the most accurate
object detection algorithms at full frame rate today, it is typically powerful
enough to support less accurate algorithms.  These {\em weak detectors}, for
instance, MobileNet in Table~\ref{fig:onboard-dnn-speed}, are typically
designed for mobile platforms or were the state of the art just a few years ago.
In addition, they can be biased towards high recall with only modest loss of
precision. In other words, many clearly irrelevant frames can be discarded by a
weak detector, without unacceptably increasing the number of relevant frames
that are erroneously discarded.  This asymmetry is the basis of the early
discard strategy.

As shown in Figure~\ref{fig:ondrone}, we envision a choice of weak detectors
being available as early discard filters on Tier-3 devices with the specific
choice of filter being task-specific.  Based on the measurements presented in
Table~\ref{fig:onboard-dnn-speed}, we choose cheap DNNs that can run in
real-time as EarlyDiscard filters on Tier-3 devices. Note that both object
detection and image classification algorithms can yield meaningful early discard
results, as it is not necessary to know exactly where in the frame relevant
objects occur --- just an estimate of key object presence is good enough. This
suggests that MobileNet would be a good choice as a weak detector. For a given
image or partial of an image, it can predict whether the input contains objects
of interests. More importantly, MobileNet's speed of 13 ms per frame on the
Tier-3 platform Jetson yields more than 75 fps. We therefore use MobileNet for
early discard in our experiments.

Pre-trained classifiers for MobileNet are available today for generic objects
such as cars, animals, human faces, human bodies, watercraft, and so on.
However, these DNN classifiers have typically been trained on images that were
captured from a human perspective --- often by a camera held or worn by a
person. These images typically have the objects at the center of the image and
occupy the majority of the image. Many Tier-3 devices, however, capture images
from different viewpoints (e.g. aerial views) and need to recognize rare
task-specific objects different from generic categories. To improve the
classification accuracy for custom objects from different viewpoints, we used
    {\em transfer learning}~\cite{Yosinski2014} to finetune the pre-trained
classifiers on small training sets of images that were captured from correct
viewpoint. The process of fine-tuning involves initial re-training of the last
DNN layer, followed by re-training of the entire network until convergence.
Transfer learning enables accuracy to be improved significantly for custom
objects without incurring the full cost of creating a large training set.

% captured
% from an aerial viewpoint.

% Finetuning not only allows us to adapt pre-trained classifiers to
% drone views, but also makes it possible to target custom objects of interests
% that are not in the original dataset, for instance, survivors in orange life
% jacket. 


\begin{figure}
    \centering
    \includegraphics[width=0.8\linewidth]{FIGS/fig-training.pdf}
    \caption{Tiling and DNN Fine Tuning}
    \label{fig:tiling}
\end{figure}

For live drone video analytics, images are typically captured from a significant
height, and hence objects in such an image are small.  This interacts negatively
with the design of many DNNs, which first transform an input image to a fixed
low resolution --- for example, 224x224 pixels in MobileNet. Many important but
small objects in the original image become less recognizable.  It has been shown
that small object size correlates with poor accuracy in DNNs~\cite{Huang2017}.
To address this problem, we {\em tile} high resolution frames into multiple
sub-frames and then perform recognition on the sub-frames as a batch.  This is
done offline for training, as shown in Figure~\ref{fig:tiling}, and also for
online inference on the drone and on the cloudlet.  The lowering of resolution
of a sub-frame by a DNN is less harmful, since the scaling factor is smaller.
Objects are represented by many more pixels in a transformed sub-frame than if
the entire frame had been transformed. The price paid for tiling is increased
computational demand.  For example, tiling a frame into four sub-frames results
in four times the classification workload. Note that this increase in workload
typically does not translates into the same increase in inference time, as
workloads can be batched together to leverage hardware parallelism for a reduced
total inference time.

\begin{figure}
    \centering
    \includegraphics[width=.8\linewidth]{FIGS/fig-tile-resolution-speed-accuracy.pdf}
    \caption{Speed-Accuracy Trade-off of Tiling}
    \label{fig:earlydiscard-tile-accuracy-speed}
\end{figure}


\subsection{Experimental Setup}

Our experiments on the EarlyDiscard strategy used the same benchmark suite
described in Section~\ref{sec:dumbdrone-setup}. We used Jetson TX2 as the Tier-3
device platform. We run MobileNet filters to get predictions on whether
sub-frames contain objects of interests. We compare the predictions with ground
truths (e.g. whether a sub-frame is indeed interesting) to evaluate the
effectiveness of EarlyDiscard. Both frame-based and event-based metrics are used
in the evaluation.

\begin{figure}
    \centering
    \includegraphics[width=\linewidth]{FIGS/fig-event-recall-frame-percentage-legend.pdf}\\
    \vspace{.5in}
    \begin{minipage}[]{0.45\linewidth}
        \centering
        \includegraphics[width=\linewidth]{FIGS/fig-event-recall-frame-percentage-vs-threshold-okutama.pdf}\\
        {(a) T1}
    \end{minipage}
    \begin{minipage}[]{0.45\linewidth}
        \centering
        \includegraphics[width=\linewidth]{FIGS/fig-event-recall-frame-percentage-vs-threshold-stanford.pdf}\\
        {(b) T2}
    \end{minipage}

    \vspace{.5in}

    \begin{minipage}[]{0.45\linewidth}
        \centering
        \includegraphics[width=\linewidth]{FIGS/fig-event-recall-frame-percentage-vs-threshold-raft.pdf}\\
        {(c) T3}
    \end{minipage}
    \begin{minipage}[]{0.45\linewidth}
        \centering
        \includegraphics[width=\linewidth]{FIGS/fig-event-recall-frame-percentage-vs-threshold-elephant.pdf}\\
        {(c) T4}
    \end{minipage}

    \vspace{.5in}
    \caption{Bandwidth Breakdown}
    \label{fig:earlydiscard-frame-percent-breakdown}
\end{figure}

\subsection{Evaluation}
\label{sec:earlydiscard-result}

EarlyDiscard is able to significantly reduce the bandwidth consumed while
maintaining high result accuracy and low average delay. For three out of four
tasks, the average bandwidth is reduced by a factor of ten. Below we present
our results in detail.

\subsubsection{Effects of Tiling}
% \noindent{\textbf{Effects of Tiling}}: 
Tiling is used to improve the accuracy
for high resolution aerial images. We used the Okutama Action Dataset, whose
attributes are shown in row T1 of Table~\ref{fig:benchmarksuite}, to explore
the effects of tiling.  For this dataset,
Figure~\ref{fig:earlydiscard-tile-accuracy-speed} shows how speed and accuracy
change with tile size.  Accuracy improves as tiles become smaller, but the
sustainable frame rate drops.  We group all tiles from the same frame in a
single batch to leverage parallelism, so the processing does not change linearly
with the number of tiles. The choice of an operating point will need to strike a
balance between the speed and accuracy.  In the rest of the chapter, we use two
tiles per frame by default.

\subsubsection{EarlyDiscard Filter Accuracy}

The output of a Tier-3 filter is the probability of the current tile being
``interesting.''  A tunable {\em cutoff threshold} parameter specifies the
threshold for transmission to the cloudlet. All tiles, whether deemed
interesting or not, are still stored in the Tier-3 storage for offline processing.

Since objects have temporal locality in videos, we define an event (of an
object) in a video to be consecutive frames containing the same object of
interests. For example, the appearance of the same red raft in T3 in consecutive
45 frames constitutes a single event. A correct detection of an event is defined
as at least one of the consecutive frames being transmitted to the cloudlet.

Figure~\ref{fig:earlydiscard-frame-percent-breakdown} shows our results on all
four tasks. Blue lines show how the event recalls of EarlyDiscard filters for different
tasks change as a function of cutoff threshold. The MobileNet DNN filter we used
is able to detect all the events for T1 and T4 even at a high cutoff threshold.
For T2 and T3, the majority of the events are detected. Achieving high recall on
T2 and T3 (on the order of 0.95 or better) requires setting a low cutoff
threshold.  This leads to the possibility that many of the transmitted frames
are actually uninteresting (i.e., false positives).

\subsubsection{False negatives}
As discussed earlier, false negatives are
a source of concern with early discard.  Once the Tier-3 device drops a frame
containing an important event, improved cloudlet processing cannot help. The
results in the third column of Table~\ref{fig:early-discard-results} confirm
that there are no false negatives for T1 and T4 at a cutoff threshold of 0.5.
For T2 and T3, lower cutoff thresholds are needed to achieve perfect recalls.

\subsubsection{Result latency}
The contribution of early discard processing to total result latency
is calculated as the average time difference between the first frame
in which an object occurs (i.e., first occurrence in ground truth) and
the first frame containing the object that is transmitted to the
backend (i.e., first detection).  The results in the fourth column of
Table~\ref{fig:early-discard-results} confirm that early discard
contributes little to result latency.  The amounts range from 0.1~s
for T1 to 12.7~s for T3.

% At the timescale of human actions
% such as dispatching of a rescue team, these are negligible delays.

\begin{table}
    \centering
    \begin{tabular}{|c|c|c|c|c|c|c|}
        \hline
           & Task Total Events & Detected Events  & Avg Delay      & Total Data     & Avg B/W & Peak B/W       \\
           &                   &                  & (s)            & (MB)           & (Mbps)  & (Mbps)         \\

        \hline
        T1 & \phantom{0}62     & 100~\%           & \phantom{0}0.1 & \phantom{0}441 & 5.10    & 10.7           \\
        \hline
        T2 & \phantom{0}11     & \phantom{0}73~\% & \phantom{0}4.9 & \phantom{00}13 & 0.03    & \phantom{0}7.0 \\ % 100% recall at 47% of frames, 82% recall at 21% of frames
        \hline
        T3 & \phantom{0}31     & \phantom{0}90~\% & 12.7           & \phantom{00}93 & 0.24    & \phantom{0}7.0 \\ % 100% recall at 9% frames
        \hline
        T4 & \phantom{0}25     & 100~\%           & \phantom{0}0.3 & \phantom{0}167 & 0.43    & \phantom{0}7.0 \\
        \hline
    \end{tabular}\\
    \caption{Recall, Event Latency and Bandwidth at Cutoff Threshold 0.5}
    \label{fig:early-discard-results}
\end{table}


\subsubsection{Bandwidth}
Columns 5--7 of Table~\ref{fig:early-discard-results} pertain to wireless
bandwidth demand for the benchmark suite with early discard.  The figures shown
are based on H.264 encoding of each individual frames in the video transmission.
Average bandwidth is calculated as the total data transmitted divided by mission
duration.  Comparing column 5 of Table~\ref{fig:early-discard-results} with
column 2 of Table~\ref{fig:baseline}, we see that all videos in the benchmark
suite are benefited by early discard (Note T3 and T4 have the same test dataset
as T2). For T2, T3, and T4, the bandwidth is reduced by more than 10x. The
amount of benefit is greatest for rare events (T2 and T3).  When events are
rare, the Tier-3 device can drop many frames.

Figure~\ref{fig:earlydiscard-frame-percent-breakdown} provides deeper insight
into the effectiveness of cutoff-threshold on event recall. It also shows how
many true positives (violet) and false positives (aqua) are
transmitted. Ideally, the aqua section should be zero.  However for T2, most
frames transmitted are false positives, indicating the early discard filter has
low precision.  The other tasks exhibit far fewer false positives.  This
suggests that the opportunity exists for significant bandwidth savings if
precision could be further improved, without hurting recall.

\subsection{Use of Sampling}

\begin{figure}
    \centering
    \includegraphics[width=.7\linewidth]{FIGS/fig-random-select-interval-recall-hatch.pdf}
    \caption{Event Recall at Different Sampling Intervals}
    \label{fig:sampling-only}
\end{figure}


\begin{figure}
    \centering
    \includegraphics[width=0.7\linewidth]{FIGS/fig-recall-frame-aggregated-legend.pdf}

    \begin{minipage}[]{0.47\linewidth}
        \centering
        \includegraphics[trim={0.5cm 0.5cm 0 0},clip,width=\linewidth]{FIGS/fig-random-select-and-filter-recall-frame-okutama-aggregated.pdf}\\
        {(a) T1}
    \end{minipage}
    \begin{minipage}[]{0.47\linewidth}
        \centering
        \includegraphics[trim={0.5cm 0.5cm 0 0},clip,width=\linewidth]{FIGS/fig-random-select-and-filter-recall-frame-stanford-aggregated.pdf}
        {(b) T2}
    \end{minipage}
    \begin{minipage}[]{0.47\linewidth}
        \centering
        \includegraphics[trim={0.5cm 0.5cm 0 0},clip,width=\linewidth]{FIGS/fig-random-select-and-filter-recall-frame-raft-aggregated.pdf}
        {(c) T3}
    \end{minipage}
    \begin{minipage}[]{0.47\linewidth}
        \centering
        \includegraphics[trim={0.5cm 0.5cm 0 0},clip,width=\linewidth]{FIGS/fig-random-select-and-filter-recall-frame-elephant-aggregated.pdf}
        {(c) T4}
    \end{minipage}
    \caption{Sample with Early Discard. Note the log scale on y-axis.}
    \label{fig:sampling-discard}
\end{figure}

\begin{table}
    \centering
    \begin{tabular}{|c|c|c|c|c|}
        \hline
        JPEG Frame Sequence & H264 High Quality & H264 Medium Quality & H264 Low Quality \\
        (MB)                & (MB)              & (MB)                & (MB)             \\
        \hline
        5823                & 3549              & 1833                & 147              \\
        \hline
    \end{tabular}\\
    \vspace{0.1in}
    \begin{captiontext}
        H264 high quality uses Constant Rate Factor (CRF) 23. Medium
        uses CRF 28 and low uses CRF 40~\cite{Merritt2007}.
    \end{captiontext}
    \caption{Test Dataset Size With Different Encoding Settings}
    \label{fig:video-vs-images}
\end{table}

Given the relatively low precision of the weak detectors, a significant number
of false positives are transmitted.  Furthermore, the occurrence of an object will
likely last through many frames, so true positives are also often redundant for
simple detection tasks.  Both of these result in excessive
consumption of precious bandwidth.
This suggests that simply restricting the number of transmitted
frames by sampling may help reduce bandwidth consumption.

Figure~\ref{fig:sampling-only} shows the effects of
sending a sample of frames from Tier-3, without any
content-based filtering.  Based on these results, we can reduce
the frames sent as little as one per second and still get
adequate recall at the cloudlet.  Note that this result is very
sensitive to the actual duration of the events in the videos.
For the detection tasks outlined here, most of the events (e.g.,
presences of a particular elephant) last for many seconds (100's
of frames), so such sparse sampling does not hurt recall.
However, if the events were of short duration, e.g., just a few
frames long, then this method would be less effective, as
sampling may lead to many missed events (false negatives).

Can we use content-based filtering along with sampling to further
reduce bandwidth consumption?  Figure~\ref{fig:sampling-discard}
shows results when running early discard on a sample of the
frames. This shows that for the same recall, we can reduce the
bandwidth consumed by another factor of 5 on average over sampling alone.
This effective combination can reduce the average bandwidth
consumed for our test videos to just a few hundred kilobits
per second.  Furthermore, more processing time is available per
processed frame, allowing more sophisticated algorithms to be
employed, or to allow smaller tiles to be used, improving
accuracy of early discard.

One case where sampling is not an effective solution is when all frames
containing an object need to be sent to the cloudlet for some form of activity
or behavior analysis from a complete video sequence.  In this case, bandwidth
will not reduce much, as all frames in the event sequence must be sent.
However, the processing time benefits of sampling may still be exploited,
provided all frames in a sample interval are transmitted on a match.


\subsection{Effects of Video Encoding}

One advantage of the Dumb strategy is that since all
frames are transmitted, one can use a modern video encoding to
reduce transmission bandwidth.  With early discard, only a subset
of disparate frames are sent.  These will likely need to be
individually compressed images, rather than a video stream.  How
much does the switch from video to individual frames affect
bandwidth?

In theory, this can be a significant impact. Video encoders leverage the
similarity between consecutive frames, and model motion to efficiently encode
the information across a set of frames. Image compression can only exploit
similarity within a frame, and cannot efficiently reduce number of bits needed
to encode redundant content across frames. To evaluate this difference, we start
with extracted JPEG frame sequences of our video data set. We encode the frame
sequence with different H.264 settings. Table~\ref{fig:video-vs-images}
compares the size of frame sequences in JPEG and the encoded video file sizes.
We see only about 3x difference in the data size for the medium quality. We can
increase the compression (at the expense of quality) very easily, and are able
to reduce the video data rate by another order of magnitude before quality
degrades catastrophically.

However, this compression does affect analytics. Even at medium quality level,
visible compression artifacts, blurring, and motion distortions begin to appear.
Initial experiments analyzing compressed videos show that these distortions do
have a negative impact on accuracy of analytics. Using average precision
analysis, a standard method to evaluate accuracy, we estimate that the
most accurate model (Faster-RCNN ResNet101) on low quality videos performs similarly
to the less accurate model (Faster-RCNN InceptionV2) on high quality
JPEG images. This negates the benefits of using the state-of-art models.

In our EarlyDiscard design, we pay a penalty of sending frames instead of a
compressed low quality video stream. This overhead (approximately 30x) is
compensated by the 100x reduction in frames transmitted due to sampling with
early discard. In addition, the selective frame transmission preserves the
accuracy of the state-of-art detection techniques.

Finally, one other option is to treat the set of disparate frames as a sequence
and employ video encoding at high quality. This can ultimately eliminate the per
frame overhead while maintaining accuracy. However, this will require a complex setup with
both low-latency encoders and decoders, which can generate output data
corresponding to a frame as soon as input data is ingested, with no buffering,
and can wait arbitrarily long for additional frame data to arrive.

For the experiments in the rest of this chapter, we only account for the
fraction of frames transmitted, rather than the choice of specific encoding
methods used for those frames.
\section{{\xc Just-in-time-Learning} Strategy To Improve Early Discard}
\label{sec:jitl}

\begin{figure}
    \centering
    \includegraphics[trim={0 1.8cm 0 0},clip,width=0.7\linewidth]{FIGS/fig-jitl-legend.pdf}\\
    \vspace{.5in}
    \begin{subfigure}[b]{.48\linewidth}
    \centering
    \includegraphics[width=\linewidth]{FIGS/fig-jitl-okutama-eventrecall-step.pdf}
    \caption{T1}
    \end{subfigure}
    \begin{subfigure}[b]{.48\linewidth}
    \centering
    \includegraphics[width=\linewidth]{FIGS/fig-jitl-stanford-eventrecall-step.pdf}
    \caption{T2}
    \end{subfigure}

    \vspace{.5in}

    \begin{subfigure}[b]{.48\linewidth}
    \centering
        \includegraphics[width=\linewidth]{FIGS/fig-jitl-raft-eventrecall-step.pdf}
    \caption{T3}
    \end{subfigure}
    \begin{subfigure}[b]{.48\linewidth}
    \centering
        \includegraphics[width=\linewidth]{FIGS/fig-jitl-elephant-eventrecall-step.pdf}
    \caption{T4}
    \end{subfigure}

    \vspace{.5in}
\caption{JITL Fraction of Frames under Different Event Recall}
\label{fig:jitl-eventrecall}
\end{figure}

While EarlyDiscard filters are customized and optimized for specific tasks (e.g.
detecting a human with red life jacket), we observe that EarlyDiscard filters do
not leverage context information within a specific video stream. Opportunities
exist if we could further specialize the computer vision processing to the
characteristics of video streams.

We propose Just-in-time-learning  (JITL), which tunes the Tier-3 processing
pipeline to the characteristics of the current mission in order to reduce
transmitted false positives from the Tier-3 device, and therefore reduce wasted
bandwidth.  Intuitively, JITL leverages temporal locality in video streams to
quickly adapt processing outcomes based on recent feedback. 

It is inspired by the ideas of cascade architecture from the computer vision
community~\cite{Viola2001}, but is different in construction. A JITL filter is a
cheap cascade filter that distinguishes between the EarlyDiscard DNN's
\emph{true positives} (frames that are actually interesting) and \emph{false
positives} (frames that are wrongly considered interesting). Specifically, when
a frame is reported as positive by EarlyDiscard, it is then passed through a
JITL filter. If the JITL filter reports negative, the frame is regarded as a
false positive and will not be sent. Ideally, all \emph{true positives} from
EarlyDiscard are marked \emph{positive} by the JITL filter, and all \emph{false
positives} from EarlyDiscard are marked \emph{negative}.  Frames dropped by
EarlyDiscard are not processed by the JITL filter, so this approach can only
serve to improve precision, but not recall.


Periodically during a drone mission, a JITL filter is trained on the cloudlet
using the frames transmitted from the drone.  The frames received on the
cloudlet are predicted positive by the EarlyDiscard filter. The cloudlet, with
more processing power, is able to run more accurate DNNs to identify true
positives and false positives. Using this information as a feedback on how well
current Tier-3 processing pipeline is doing, a small and lightweight JITL filter
is trained to distinguish true positives and false positives of EarlyDiscard
filters. These JITL filters are then pushed to the drone to run as a cascade
filter after the EarlyDiscard DNN.

True/false positive frames have high temporal locality throughout a drone
mission. The JITL filter is expected to pick up the features that confused the
EarlyDiscard DNN in the immediate past and improve the pipeline's accuracy in
the near future. These features are usually specific to the current flight, and
may be affected by terrain, shades, object colors, and particular shapes or
background textures.

JITL can be used with EarlyDiscard DNNs of different cutoff probabilities to
strike different trade-offs. In a bandwidth-favored setting, JITL can work with
an aggressively selective EarlyDiscard DNN to further reduce wasted bandwidth. In
a recall-favored setting, JITL can be used with a lower-cutoff DNN to preserve
recall.

In our implementation, we use a linear support vector machine
(SVM)~\cite{Friedman2001} as the JITL filter. Linear SVM has several advantages:
1) short training time in the order of seconds; 2) fast inference; 3) only
requires a few training examples; 3) small in size to transmit, usually on the
order of 50KB in our experiments. The input features to the JITL SVM filter are
the image features extracted by the EarlyDiscard DNN filter. In our case, since
we are using MobileNet as our EarlyDiscard filter, they are the 1024-dimensional
vector elements from the second last layer of MobileNet. This vector, also
called ``bottleneck values'' or ``transfer values'' captures high-level features
that represents the content of an image. Note that the availability of such
image feature vector is not tied to a particular image classification DNN nor
unique to MobileNet. Most image classification DNNs can be used as a feature
extractor in this way.

\subsection{JITL Experimental Setup}
We used Jetson TX2 as our Tier-3 device platform and evaluated the JITL strategy
on four tasks, T1 to T4. For the test videos in each task, we began with the
EarlyDiscard filter alone and gradually trained and deployed JITL filters.
Specifically, every ten seconds, we trained an SVM using the frames transmitted
from the drone and the ground-truth labels for these frames. In a real
deployment, the frames would be marked as true positives or false positives by
an accurate DNN running on the cloudlet since ground-truth labels are not
available. In our experiments, we used ground-truth labels to control variables
and remove the effect of imperfect prediction of DNN models running on the
cloudlet. 

In addition, we used the true and false positives from all previous intervals,
not just the last ten seconds when training the SVM. The SVM, once trained, is
used as a cascade filter running after the EarlyDiscard filter on the drone to
predict whether the output of the EarlyDiscard filter is correct or not. For a
frame, if the EarlyDiscard filter predicts it to be interesting, but the JITL
filter predicts the EarlyDiscard filter is wrong, it would not be transmitted to
the cloudlet. In other words, following two criteria need to be satisfied for a
frame to be transmitted to the cloudlet: 1) EarlyDiscard filter predicts it to
be interesting 2) JITL filter predicts the EarlyDiscard filter is correct on
this frame.

\subsection{JITL Results}

From our experiments, JITL is able to filter out more than 15\% of remaining
frames after EarlyDiscard without loss of event recall for three of four tasks.
Figure~\ref{fig:jitl-eventrecall} details the fraction of frames saved by JITL.
The x-axis presents event recall. Y-axis represents the fraction of total
frames. The blue region presents the achievable fraction of frames by
EarlyDiscard. The orange region shows the additional savings using JITL. For T1,
T3, and T4, at the highest event recall, JITL filters out more than 15\% of
remaining frames. This shows that JITL is effective at reducing the false
positives thus improving the precision of the drone filter. However,
occasionally, JITL predicts wrongly and removes true positives. For example, for
T2, JITL does not achieve a perfect event recall. This is due to shorter event
duration in T2, which results in fewer positive training examples to learn
from. Depending on tasks, getting enough positive training examples for JITL
could be difficult, especially when events are short or occurrences are few. To
overcome this problem in practice, techniques such as synthetic data
generation~\cite{Dwibedi2017} could be explored to synthesize true positives
from the background of the current flight.

\section{Applying EarlyDiscard and JITL to Wearable Cognitive Assistants}
\label{bw:wca}

While the experiments in previous sections
(~\ref{sec:earlydiscard}~\ref{sec:jitl}) are performed in a drone video
analytics context, EarlyDiscard and JITL approaches can be applied more
generally to live video analytics offloading from Tier-3 devices to Tier-2 edge
data-centers. In this section, we use the LEGO
application~\cite{chen2018application} to showcase how to apply these bandwidth
saving approaches to WCAs.

\begin{figure}
    \centering
    \begin{minipage}[]{0.45\linewidth}
        \centering
        \includegraphics[width=\linewidth]{FIGS/lego-search}\\
        {(a) Searching for Lego Blocks}
    \end{minipage}
    \begin{minipage}[]{0.45\linewidth}
        \centering
        \includegraphics[width=\linewidth]{FIGS/lego-assembled}\\
        {(b) Assembling Lego Pieces}
    \end{minipage}
    \caption{Example Images from a Lego Assembly Video}
    \label{fig:wca-lego-example-images}
\end{figure}

The LEGO wearable cognitive assistant helps a user put together a specific Lego
pattern by providing step-by-step audiovisual instructions. The application
works as follows. The assistant first prompts a user an animated image showing
the Lego block to use and asks the user to put it on the Lego board or assemble
it with previous pieces. Following the guidance, the user searches for the
particular Lego block, assemble it, and put the assembled piece on the Lego
board for the next instruction. Figure~\ref{fig:wca-lego-example-images} shows
the first-person view images captured from the wearable device during this
process. The assistant analyzes the assembled Lego piece on the Lego board by
identifying its shape and color using computer vision and provides the suitable
instruction.

\begin{figure}
    \centering
    \begin{minipage}[]{0.31\linewidth}
        \centering
        \includegraphics[width=\linewidth]{FIGS/lego-dataset-1}\\
    \end{minipage}
    \begin{minipage}[]{0.31\linewidth}
        \centering
        \includegraphics[width=\linewidth]{FIGS/lego-dataset-2}\\
    \end{minipage}
    \begin{minipage}[]{0.31\linewidth}
        \centering
        \includegraphics[width=\linewidth]{FIGS/lego-dataset-3}\\
    \end{minipage}
    \caption{Example Images from LEGO Dataset}
    \label{fig:wca-lego-dataset}
\end{figure}

Intuitively, to the assistant, frames capturing the assembled piece on the Lego
board, (for example Figure~\ref{fig:wca-lego-example-images} (b)) are the
crucial frames to process, as they reflect the user's working progress.
Figure~\ref{fig:wca-lego-example-images} (a), on the other hand, is less
interesting as it does not contain information on user progress. If some cheap
processing on the wearable device could distinguish
(a) from (b), bandwidth consumption can be
reduced as we can discard Figure~\ref{fig:wca-lego-example-images} (a) early on
the wearable device without transmitting the frame to the cloudlet for processing.
This provides opportunities to apply EarlyDiscard and JITL.

We collect a LEGO dataset of twelve videos, in which users assemble Lego pieces
in three environments with different background, lighting, and viewpoints.
Figure~\ref{fig:wca-lego-dataset} shows example images from the dataset. We run
the LEGO WCA on these videos to get pseudo ground truth labels. Specifically,
for each frame, based on the outputs of the LEGO WCA vision processing, we
categorize the frame to be either ``interesting'' or ``not interesting''. A
frame is considered to be interesting if a LEGO board is found in the frame,
otherwise considered not interesting.

\begin{figure}
    \centering
    \includegraphics[width=.6\linewidth]{FIGS/earlydiscard-cm}\\
    \caption{EarlyDiscard Filter Confusion Matrix}
    \label{fig:wca-early-discard}
\end{figure}

We use this dataset to finetune a MobileNet DNN in order to automatically
distinguish interesting frames from the boring ones for EarlyDiscard. For each
of the three environments, we randomly select two videos for training, one video
for validation, and one video for testing. We randomly sample 2000 interesting
images and 2000 boring images from the six training videos as the training
data. Similarly, we random sample 200 interesting images and 200 boring images
from the three validation videos as the validation data. We implement
MobileNet transfer learning using the PyTorch
framework~\cite{paszke2019pytorch}. We train the model for 20 epochs and
select the model weights that give the highest accuracy on the validation set as
the model for inference.

Our test sets have in total 14725 frames. Figure~\ref{fig:wca-early-discard}
shows the confusion matrix of our trained EarlyDiscard classifier. X-axis
represents the predicted results: ``Transmit'' means the frame is predicted to
be interesting and should be transmitted to cloudlet for processing while
``Discard'' means the frame is predicted to be boring and should not be
transmitted. Similarly, Y-axis represents the ground truth results. As we can
see, the classifier correctly predicts 2260 out of 14725 frames to be
interesting and correctly suppresses 11971 frames. With EarlyDiscard in place,
only 19\% of all the frames are transmitted. Meanwhile, the false negative is 0
frame, meaning no ``interesting'' frame is wrongly discarded. This is the result
of biasing the classifier towards recall instead of precision.

\begin{figure}
    \centering
    \begin{minipage}[b]{.45\linewidth}
        \centering
        \includegraphics[width=\linewidth]{FIGS/jitl-earlydiscard-cm}\\
        {(a) EarlyDiscard}
    \end{minipage}
    \begin{minipage}[b]{.45\linewidth}
        \centering
        \includegraphics[width=\linewidth]{FIGS/jitl-combined-cm}\\
        {(b) EarlyDiscard + JITL}
    \end{minipage}
    \caption{JITL Confusion Matrix}
    \label{fig:wca-jitl}
\end{figure}

Among all the frames that are transmitted, 18\% of them are false positives.
These 494 false positives suggest that there are room to improve using JITL. For
each of the test videos, we use the first half of the video as training examples
for JITL to train a SVM that produces a confidence score for EarlyDiscard
prediction. Figure~\ref{fig:wca-jitl} compares the confusion matrix of using
EarlyDiscard alone with EarlyDiscard + JITL. As we can see, JITL reduces 13\% of
the false positives at the cost of 2 false negatives. Note that these 2 false
negative frames do not result in missing instructions as adjacent interesting
frames are still transmitted.

% To reduce bandwidth consumption with EarlyDiscard and JITL, we first need to
% identify what frames should be considered interesting.

% the cheap computer vision processing that can identify ``interesting'' frames on
% Tier-3 devices. Figure~\ref{fig:wca-lego-example-images} provides intuitions on how to
% apply EarlyDiscard and JITL to the Lego application
\section{Related Work}
\label{bw:relatedwork}

In the context of drone video analytics, Wang et al.~\cite{Wang2017networked}
shares our concern for wireless bandwidth, but focuses on coordinating a network
of drones to capture and broadcast live sport event. In addition, Wang et
al~\cite{Wang2016skyeyes} explored adaptive video streaming with drones using
content-based compression and video rate adaptation. While we share their
inspiration, our work leverages characteristics of DNNs to enable
mission-specific optimization strategies.

Much previous work on static camera networks explored efficient use of compute
and network resources at scale. Zhang et al.~\cite{zhang2017live} studied
resource-quality trade-off under result latency constraints in video analytics
systems. Kang et al.~\cite{kang2017noscope} worked on optimizing DNN queries
over videos at scale. While they focus on supporting a large number of computer
vision workload, our work optimizes for the first hop wireless bandwidth. In
addition, Zhang et al.~\cite{zhang2015design} designed a wireless distributed
surveillance system that supports a large geographical area through frame
selection and content-aware traffic scheduling. In contrast, our work does not
assume static cameras. We explore techniques that tolerate changing scenes in
video feeds and strategies that work for moving cameras.

Some previous work on computer vision in mobile settings has relevance to
aspects of our system design.  Chen et al.~\cite{chen2015glimpse} explore how
continuous real-time object recognition can be done on mobile devices. They meet
their design goals by combining expensive object detection with computationally
cheap object tracking.  Although we do not use object tracking in our work, we
share the resource concerns that motivate that work.  Naderiparizi et
al.~\cite{naderiparizi2017glimpse} describe a programmable early-discard camera
architecture for continuous mobile vision.  Our work shares their emphasis on
early discard, but differs in all other aspects.  In fact, our work can be
viewed as complementing that work: their programmable early-discard camera would
be an excellent choice for Tier-3 devices. Lastly, Hu et al~\cite{Hu2015} have
investigated the approach of using lightweight computation on a mobile device to
improve the overall bandwidth efficiency of a computer vision pipeline that
offloads computation to the edge.  We share their concern for wireless
bandwidth, and their use of early discard using inexpensive algorithms on the
mobile device.

\section{Chapter Summary and Discussion}
\label{bw:discussion}

In this chapter, we address the bandwidth challenge of running many WCAs at
scale. We propose two application-agnostic methods to reduce bandwidth
consumption when offloading computation to edge servers.

The EarlyDiscard technique employs on-board filters to select interesting frames
and suppress the transmission of mundane frames to save bandwidth. In
particular, cheap yet effective DNN filters are trained offline to fully
leverage the large quantity of training data and the high learning capacities of
DNNs. Building on top of EarlyDiscard, JITL adapts an EarlyDiscard filter to a
specific environment online. While a WCA is running, JITL continuously evaluates
the EarlyDiscard filter and reduces the number of false positives by predicting
whether an EaryDiscard decision is made correctly. These two techniques together
reduce the total number of unnecessary frames transmitted.

We evaluate these two strategies first in the drone live video analytics context
for search tasks in domains such as search-and-rescue, surveillance, and
wildlife conservation, and then for WCAs. Our experimental results show that
this judicious combination of Tier-3 processing and edge-based processing can
save substantial wireless bandwidth and thus improve scalability, without
compromising result accuracy or result latency.