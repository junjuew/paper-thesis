\chapter{Application-Agnostic Techniques to Reduce Network Transmission}

WCAs continuously stream sensor data to the cloudlet. The richer a sensing
modality is, the more information can be extracted. Visual data, e.g. images and
videos, from cameras is the main sensing modalities of wearable cognitive
assistance. However, continuous video transmission from many wearable devices
places severe stress on the wireless spectrum.  Hulu estimates that its video
streams require 13 Mbps for 4K resolution and 6 Mbps for HD resolution using
highly optimized offline encoding~\cite{Hulu2017}. Live streaming is less
bandwidth-efficient, as confirmed by our measured bandwidth of 10 Mbps for HD
feed at 25 FPS. Just 50 users transmitting HD video streams continuously can
saturate the theoretical uplink capacity of 500 Mbps in a 4G LTE cell that
covers a large rural area~\cite{LteWorld2009}.  This is clearly not scalable.

In this section, we show how per-user bandwidth demand can be significantly
reduced in an application-agnostic fashion, without compromising the timeliness
or accuracy of results. We present techniques for an adaptive computer vision
pipeline for WCAs that leverages edge computing to enable dynamic optimizations.
In contrast to previous
works~\cite{Wang2017networked}~\cite{Zhang2015design}~\cite{Wang2016skyeyes}, we
leverage state-of-the-art deep neural networks (DNNs) to selectively transmit
interesting data from a video stream and explore mission-specific optimizations.

% \begin{enumerate}
%     \item talk about weak and accurate detectors
%     \item evaluate in drone context
%     \item evaluate in WCA contexts
% \end{enumerate}

\input{bandwidth-challenges.tex}

\section{EarlyDiscard Strategy}
\label{sec:earlydiscard}

\begin{figure}
    \includegraphics[trim={0cm 13cm 14cm 0cm},clip,width=\linewidth]{FIGS/fig-early-discard.pdf}
    \caption{Early Discard on Tier-3 Devices}
    \label{fig:ondrone}
\end{figure}


\subsection{Description}
EarlyDiscard is based on the idea of using on-board processing to filter and
transmit only interesting frames in order to save bandwidth when offloading
computation. Frames are considered to be interesting if they capture objects or
events valuable for processing, for instance, survivors for a search task.
Previous work~\cite{Hu2015,Naderiparizi2017} leveraged pixel-level
features and multiple sensing modalities to select interesting frames from
hand-held or body-worn cameras. In this section, we explore the use of DNNs to
filter frames from aerial views. The benefits of using DNNs are as follows.
First, DNNs, even shallow ones, are capable of understanding some semantically
meaningful visual information. Their decisions of what to send are based on the
reasoning of image content in addition to pixel-level characteristics. Next,
DNNs are trained and specialized for each task, resulting in their high accuracy
and robustness for that particular task. Finally, compared to a sensor fusing
approach that requires other sensing modalities to be present on Tier-3 devices,
no additional hardware is added to the existing platforms.

Although smartphone-class hardware is incapable of supporting the most accurate
object detection algorithms at full frame rate today, it is typically powerful
enough to support less accurate algorithms.  These {\em weak detectors}, for
instance, MobileNet in Table~\ref{fig:onboard-dnn-speed}, are typically
designed for mobile platforms or were the state of the art just a few years ago.
In addition, they can be biased towards high recall with only modest loss of
precision. In other words, many clearly irrelevant frames can be discarded by a
weak detector, without unacceptably increasing the number of relevant frames
that are erroneously discarded.  This asymmetry is the basis of the early
discard strategy.

As shown in Figure~\ref{fig:ondrone}, we envision a choice of weak detectors
being available as early discard filters on Tier-3 devices with the specific
choice of filter being task-specific.  Based on the measurements presented in
Table~\ref{fig:onboard-dnn-speed}, we choose cheap DNNs that can run in
real-time as EarlyDiscard filters on Tier-3 devices. Note that both object
detection and image classification algorithms can yield meaningful early discard
results, as it is not necessary to know exactly where in the frame relevant
objects occur --- just an estimate of key object presence is good enough. This
suggests that MobileNet would be a good choice as a weak detector. For a given
image or partial of an image, it can predict whether the input contains objects
of interests. More importantly, MobileNet's speed of 13 ms per frame on the
Tier-3 platform Jetson yields more than 75 fps. We therefore use MobileNet for
early discard in our experiments.

Pre-trained classifiers for MobileNet are available today for generic objects
such as cars, animals, human faces, human bodies, watercraft, and so on.
However, these DNN classifiers have typically been trained on images that were
captured from a human perspective --- often by a camera held or worn by a
person. These images typically have the objects at the center of the image and
occupy the majority of the image. Many Tier-3 devices, however, capture images
from different viewpoints (e.g. aerial views) and need to recognize rare
task-specific objects different from generic categories. To improve the
classification accuracy for custom objects from different viewpoints, we used
    {\em transfer learning}~\cite{Yosinski2014} to finetune the pre-trained
classifiers on small training sets of images that were captured from correct
viewpoint. The process of fine-tuning involves initial re-training of the last
DNN layer, followed by re-training of the entire network until convergence.
Transfer learning enables accuracy to be improved significantly for custom
objects without incurring the full cost of creating a large training set.

% captured
% from an aerial viewpoint.

% Finetuning not only allows us to adapt pre-trained classifiers to
% drone views, but also makes it possible to target custom objects of interests
% that are not in the original dataset, for instance, survivors in orange life
% jacket. 


\begin{figure}
    \centering
    \includegraphics[width=0.8\linewidth]{FIGS/fig-training.pdf}
    \caption{Tiling and DNN Fine Tuning}
    \label{fig:tiling}
\end{figure}

For live drone video analytics, images are typically captured from a significant
height, and hence objects in such an image are small.  This interacts negatively
with the design of many DNNs, which first transform an input image to a fixed
low resolution --- for example, 224x224 pixels in MobileNet. Many important but
small objects in the original image become less recognizable.  It has been shown
that small object size correlates with poor accuracy in DNNs~\cite{Huang2017}.
To address this problem, we {\em tile} high resolution frames into multiple
sub-frames and then perform recognition on the sub-frames as a batch.  This is
done offline for training, as shown in Figure~\ref{fig:tiling}, and also for
online inference on the drone and on the cloudlet.  The lowering of resolution
of a sub-frame by a DNN is less harmful, since the scaling factor is smaller.
Objects are represented by many more pixels in a transformed sub-frame than if
the entire frame had been transformed. The price paid for tiling is increased
computational demand.  For example, tiling a frame into four sub-frames results
in four times the classification workload. Note that this increase in workload
typically does not translates into the same increase in inference time, as
workloads can be batched together to leverage hardware parallelism for a reduced
total inference time.

\begin{figure}
    \centering
    \includegraphics[width=.8\linewidth]{FIGS/fig-tile-resolution-speed-accuracy.pdf}
    \caption{Speed-Accuracy Trade-off of Tiling}
    \label{fig:earlydiscard-tile-accuracy-speed}
\end{figure}


\subsection{Experimental Setup}

Our experiments on the EarlyDiscard strategy used the same benchmark suite
described in Section~\ref{sec:dumbdrone-setup}. We used Jetson TX2 as the Tier-3
device platform. We run MobileNet filters to get predictions on whether
sub-frames contain objects of interests. We compare the predictions with ground
truths (e.g. whether a sub-frame is indeed interesting) to evaluate the
effectiveness of EarlyDiscard. Both frame-based and event-based metrics are used
in the evaluation.

\begin{figure}
    \centering
    \includegraphics[width=\linewidth]{FIGS/fig-event-recall-frame-percentage-legend.pdf}\\
    \vspace{.5in}
    \begin{minipage}[]{0.45\linewidth}
        \centering
        \includegraphics[width=\linewidth]{FIGS/fig-event-recall-frame-percentage-vs-threshold-okutama.pdf}\\
        {(a) T1}
    \end{minipage}
    \begin{minipage}[]{0.45\linewidth}
        \centering
        \includegraphics[width=\linewidth]{FIGS/fig-event-recall-frame-percentage-vs-threshold-stanford.pdf}\\
        {(b) T2}
    \end{minipage}

    \vspace{.5in}

    \begin{minipage}[]{0.45\linewidth}
        \centering
        \includegraphics[width=\linewidth]{FIGS/fig-event-recall-frame-percentage-vs-threshold-raft.pdf}\\
        {(c) T3}
    \end{minipage}
    \begin{minipage}[]{0.45\linewidth}
        \centering
        \includegraphics[width=\linewidth]{FIGS/fig-event-recall-frame-percentage-vs-threshold-elephant.pdf}\\
        {(c) T4}
    \end{minipage}

    \vspace{.5in}
    \caption{Bandwidth Breakdown}
    \label{fig:earlydiscard-frame-percent-breakdown}
\end{figure}

\subsection{Evaluation}
\label{sec:earlydiscard-result}

EarlyDiscard is able to significantly reduce the bandwidth consumed while
maintaining high result accuracy and low average delay. For three out of four
tasks, the average bandwidth is reduced by a factor of ten. Below we present
our results in detail.

\subsubsection{Effects of Tiling}
% \noindent{\textbf{Effects of Tiling}}: 
Tiling is used to improve the accuracy
for high resolution aerial images. We used the Okutama Action Dataset, whose
attributes are shown in row T1 of Table~\ref{fig:benchmarksuite}, to explore
the effects of tiling.  For this dataset,
Figure~\ref{fig:earlydiscard-tile-accuracy-speed} shows how speed and accuracy
change with tile size.  Accuracy improves as tiles become smaller, but the
sustainable frame rate drops.  We group all tiles from the same frame in a
single batch to leverage parallelism, so the processing does not change linearly
with the number of tiles. The choice of an operating point will need to strike a
balance between the speed and accuracy.  In the rest of the chapter, we use two
tiles per frame by default.

\subsubsection{EarlyDiscard Filter Accuracy}

The output of a Tier-3 filter is the probability of the current tile being
``interesting.''  A tunable {\em cutoff threshold} parameter specifies the
threshold for transmission to the cloudlet. All tiles, whether deemed
interesting or not, are still stored in the Tier-3 storage for offline processing.

Since objects have temporal locality in videos, we define an event (of an
object) in a video to be consecutive frames containing the same object of
interests. For example, the appearance of the same red raft in T3 in consecutive
45 frames constitutes a single event. A correct detection of an event is defined
as at least one of the consecutive frames being transmitted to the cloudlet.

Figure~\ref{fig:earlydiscard-frame-percent-breakdown} shows our results on all
four tasks. Blue lines show how the event recalls of EarlyDiscard filters for different
tasks change as a function of cutoff threshold. The MobileNet DNN filter we used
is able to detect all the events for T1 and T4 even at a high cutoff threshold.
For T2 and T3, the majority of the events are detected. Achieving high recall on
T2 and T3 (on the order of 0.95 or better) requires setting a low cutoff
threshold.  This leads to the possibility that many of the transmitted frames
are actually uninteresting (i.e., false positives).

\subsubsection{False negatives}
As discussed earlier, false negatives are
a source of concern with early discard.  Once the Tier-3 device drops a frame
containing an important event, improved cloudlet processing cannot help. The
results in the third column of Table~\ref{fig:early-discard-results} confirm
that there are no false negatives for T1 and T4 at a cutoff threshold of 0.5.
For T2 and T3, lower cutoff thresholds are needed to achieve perfect recalls.

\subsubsection{Result latency}
The contribution of early discard processing to total result latency
is calculated as the average time difference between the first frame
in which an object occurs (i.e., first occurrence in ground truth) and
the first frame containing the object that is transmitted to the
backend (i.e., first detection).  The results in the fourth column of
Table~\ref{fig:early-discard-results} confirm that early discard
contributes little to result latency.  The amounts range from 0.1~s
for T1 to 12.7~s for T3.

% At the timescale of human actions
% such as dispatching of a rescue team, these are negligible delays.

\begin{table}
    \centering
    \begin{tabular}{|c|c|c|c|c|c|c|}
        \hline
           & Task Total Events & Detected Events  & Avg Delay      & Total Data     & Avg B/W & Peak B/W       \\
           &                   &                  & (s)            & (MB)           & (Mbps)  & (Mbps)         \\

        \hline
        T1 & \phantom{0}62     & 100~\%           & \phantom{0}0.1 & \phantom{0}441 & 5.10    & 10.7           \\
        \hline
        T2 & \phantom{0}11     & \phantom{0}73~\% & \phantom{0}4.9 & \phantom{00}13 & 0.03    & \phantom{0}7.0 \\ % 100% recall at 47% of frames, 82% recall at 21% of frames
        \hline
        T3 & \phantom{0}31     & \phantom{0}90~\% & 12.7           & \phantom{00}93 & 0.24    & \phantom{0}7.0 \\ % 100% recall at 9% frames
        \hline
        T4 & \phantom{0}25     & 100~\%           & \phantom{0}0.3 & \phantom{0}167 & 0.43    & \phantom{0}7.0 \\
        \hline
    \end{tabular}\\
    \caption{Recall, Event Latency and Bandwidth at Cutoff Threshold 0.5}
    \label{fig:early-discard-results}
\end{table}


\subsubsection{Bandwidth}
Columns 5--7 of Table~\ref{fig:early-discard-results} pertain to wireless
bandwidth demand for the benchmark suite with early discard.  The figures shown
are based on H.264 encoding of each individual frames in the video transmission.
Average bandwidth is calculated as the total data transmitted divided by mission
duration.  Comparing column 5 of Table~\ref{fig:early-discard-results} with
column 2 of Table~\ref{fig:baseline}, we see that all videos in the benchmark
suite are benefited by early discard (Note T3 and T4 have the same test dataset
as T2). For T2, T3, and T4, the bandwidth is reduced by more than 10x. The
amount of benefit is greatest for rare events (T2 and T3).  When events are
rare, the Tier-3 device can drop many frames.

Figure~\ref{fig:earlydiscard-frame-percent-breakdown} provides deeper insight
into the effectiveness of cutoff-threshold on event recall. It also shows how
many true positives (violet) and false positives (aqua) are
transmitted. Ideally, the aqua section should be zero.  However for T2, most
frames transmitted are false positives, indicating the early discard filter has
low precision.  The other tasks exhibit far fewer false positives.  This
suggests that the opportunity exists for significant bandwidth savings if
precision could be further improved, without hurting recall.

\subsection{Use of Sampling}

\begin{figure}
    \centering
    \includegraphics[width=.7\linewidth]{FIGS/fig-random-select-interval-recall-hatch.pdf}
    \caption{Event Recall at Different Sampling Intervals}
    \label{fig:sampling-only}
\end{figure}


\begin{figure}
    \centering
    \includegraphics[width=0.7\linewidth]{FIGS/fig-recall-frame-aggregated-legend.pdf}

    \begin{minipage}[]{0.47\linewidth}
        \centering
        \includegraphics[trim={0.5cm 0.5cm 0 0},clip,width=\linewidth]{FIGS/fig-random-select-and-filter-recall-frame-okutama-aggregated.pdf}\\
        {(a) T1}
    \end{minipage}
    \begin{minipage}[]{0.47\linewidth}
        \centering
        \includegraphics[trim={0.5cm 0.5cm 0 0},clip,width=\linewidth]{FIGS/fig-random-select-and-filter-recall-frame-stanford-aggregated.pdf}
        {(b) T2}
    \end{minipage}
    \begin{minipage}[]{0.47\linewidth}
        \centering
        \includegraphics[trim={0.5cm 0.5cm 0 0},clip,width=\linewidth]{FIGS/fig-random-select-and-filter-recall-frame-raft-aggregated.pdf}
        {(c) T3}
    \end{minipage}
    \begin{minipage}[]{0.47\linewidth}
        \centering
        \includegraphics[trim={0.5cm 0.5cm 0 0},clip,width=\linewidth]{FIGS/fig-random-select-and-filter-recall-frame-elephant-aggregated.pdf}
        {(c) T4}
    \end{minipage}
    \caption{Sample with Early Discard. Note the log scale on y-axis.}
    \label{fig:sampling-discard}
\end{figure}

\begin{table}
    \centering
    \begin{tabular}{|c|c|c|c|c|}
        \hline
        JPEG Frame Sequence & H264 High Quality & H264 Medium Quality & H264 Low Quality \\
        (MB)                & (MB)              & (MB)                & (MB)             \\
        \hline
        5823                & 3549              & 1833                & 147              \\
        \hline
    \end{tabular}\\
    \vspace{0.1in}
    \begin{captiontext}
        H264 high quality uses Constant Rate Factor (CRF) 23. Medium
        uses CRF 28 and low uses CRF 40~\cite{Merritt2007}.
    \end{captiontext}
    \caption{Test Dataset Size With Different Encoding Settings}
    \label{fig:video-vs-images}
\end{table}

Given the relatively low precision of the weak detectors, a significant number
of false positives are transmitted.  Furthermore, the occurrence of an object will
likely last through many frames, so true positives are also often redundant for
simple detection tasks.  Both of these result in excessive
consumption of precious bandwidth.
This suggests that simply restricting the number of transmitted
frames by sampling may help reduce bandwidth consumption.

Figure~\ref{fig:sampling-only} shows the effects of
sending a sample of frames from Tier-3, without any
content-based filtering.  Based on these results, we can reduce
the frames sent as little as one per second and still get
adequate recall at the cloudlet.  Note that this result is very
sensitive to the actual duration of the events in the videos.
For the detection tasks outlined here, most of the events (e.g.,
presences of a particular elephant) last for many seconds (100's
of frames), so such sparse sampling does not hurt recall.
However, if the events were of short duration, e.g., just a few
frames long, then this method would be less effective, as
sampling may lead to many missed events (false negatives).

Can we use content-based filtering along with sampling to further
reduce bandwidth consumption?  Figure~\ref{fig:sampling-discard}
shows results when running early discard on a sample of the
frames. This shows that for the same recall, we can reduce the
bandwidth consumed by another factor of 5 on average over sampling alone.
This effective combination can reduce the average bandwidth
consumed for our test videos to just a few hundred kilobits
per second.  Furthermore, more processing time is available per
processed frame, allowing more sophisticated algorithms to be
employed, or to allow smaller tiles to be used, improving
accuracy of early discard.

One case where sampling is not an effective solution is when all frames
containing an object need to be sent to the cloudlet for some form of activity
or behavior analysis from a complete video sequence.  In this case, bandwidth
will not reduce much, as all frames in the event sequence must be sent.
However, the processing time benefits of sampling may still be exploited,
provided all frames in a sample interval are transmitted on a match.


\subsection{Effects of Video Encoding}

One advantage of the Dumb strategy is that since all
frames are transmitted, one can use a modern video encoding to
reduce transmission bandwidth.  With early discard, only a subset
of disparate frames are sent.  These will likely need to be
individually compressed images, rather than a video stream.  How
much does the switch from video to individual frames affect
bandwidth?

In theory, this can be a significant impact. Video encoders leverage the
similarity between consecutive frames, and model motion to efficiently encode
the information across a set of frames. Image compression can only exploit
similarity within a frame, and cannot efficiently reduce number of bits needed
to encode redundant content across frames. To evaluate this difference, we start
with extracted JPEG frame sequences of our video data set. We encode the frame
sequence with different H.264 settings. Table~\ref{fig:video-vs-images}
compares the size of frame sequences in JPEG and the encoded video file sizes.
We see only about 3x difference in the data size for the medium quality. We can
increase the compression (at the expense of quality) very easily, and are able
to reduce the video data rate by another order of magnitude before quality
degrades catastrophically.

However, this compression does affect analytics. Even at medium quality level,
visible compression artifacts, blurring, and motion distortions begin to appear.
Initial experiments analyzing compressed videos show that these distortions do
have a negative impact on accuracy of analytics. Using average precision
analysis, a standard method to evaluate accuracy, we estimate that the
most accurate model (Faster-RCNN ResNet101) on low quality videos performs similarly
to the less accurate model (Faster-RCNN InceptionV2) on high quality
JPEG images. This negates the benefits of using the state-of-art models.

In our EarlyDiscard design, we pay a penalty of sending frames instead of a
compressed low quality video stream. This overhead (approximately 30x) is
compensated by the 100x reduction in frames transmitted due to sampling with
early discard. In addition, the selective frame transmission preserves the
accuracy of the state-of-art detection techniques.

Finally, one other option is to treat the set of disparate frames as a sequence
and employ video encoding at high quality. This can ultimately eliminate the per
frame overhead while maintaining accuracy. However, this will require a complex setup with
both low-latency encoders and decoders, which can generate output data
corresponding to a frame as soon as input data is ingested, with no buffering,
and can wait arbitrarily long for additional frame data to arrive.

For the experiments in the rest of this chapter, we only account for the
fraction of frames transmitted, rather than the choice of specific encoding
methods used for those frames.
\section{{\xc Just-in-time-Learning} Strategy To Improve Early Discard}
\label{sec:jitl}

\begin{figure}
    \centering
    \includegraphics[trim={0 1.8cm 0 0},clip,width=0.7\linewidth]{FIGS/fig-jitl-legend.pdf}\\
    \vspace{.5in}
    \begin{subfigure}[b]{.48\linewidth}
    \centering
    \includegraphics[width=\linewidth]{FIGS/fig-jitl-okutama-eventrecall-step.pdf}
    \caption{T1}
    \end{subfigure}
    \begin{subfigure}[b]{.48\linewidth}
    \centering
    \includegraphics[width=\linewidth]{FIGS/fig-jitl-stanford-eventrecall-step.pdf}
    \caption{T2}
    \end{subfigure}

    \vspace{.5in}

    \begin{subfigure}[b]{.48\linewidth}
    \centering
        \includegraphics[width=\linewidth]{FIGS/fig-jitl-raft-eventrecall-step.pdf}
    \caption{T3}
    \end{subfigure}
    \begin{subfigure}[b]{.48\linewidth}
    \centering
        \includegraphics[width=\linewidth]{FIGS/fig-jitl-elephant-eventrecall-step.pdf}
    \caption{T4}
    \end{subfigure}

    \vspace{.5in}
\caption{JITL Fraction of Frames under Different Event Recall}
\label{fig:jitl-eventrecall}
\end{figure}

While EarlyDiscard filters are customized and optimized for specific tasks (e.g.
detecting a human with red life jacket), we observe that EarlyDiscard filters do
not leverage context information within a specific video stream. Opportunities
exist if we could further specialize the computer vision processing to the
characteristics of video streams.

We propose Just-in-time-learning  (JITL), which tunes the Tier-3 processing
pipeline to the characteristics of the current mission in order to reduce
transmitted false positives from the Tier-3 device, and therefore reduce wasted
bandwidth.  Intuitively, JITL leverages temporal locality in video streams to
quickly adapt processing outcomes based on recent feedback. 

It is inspired by the ideas of cascade architecture from the computer vision
community~\cite{Viola2001}, but is different in construction. A JITL filter is a
cheap cascade filter that distinguishes between the EarlyDiscard DNN's
\emph{true positives} (frames that are actually interesting) and \emph{false
positives} (frames that are wrongly considered interesting). Specifically, when
a frame is reported as positive by EarlyDiscard, it is then passed through a
JITL filter. If the JITL filter reports negative, the frame is regarded as a
false positive and will not be sent. Ideally, all \emph{true positives} from
EarlyDiscard are marked \emph{positive} by the JITL filter, and all \emph{false
positives} from EarlyDiscard are marked \emph{negative}.  Frames dropped by
EarlyDiscard are not processed by the JITL filter, so this approach can only
serve to improve precision, but not recall.


Periodically during a drone mission, a JITL filter is trained on the cloudlet
using the frames transmitted from the drone.  The frames received on the
cloudlet are predicted positive by the EarlyDiscard filter. The cloudlet, with
more processing power, is able to run more accurate DNNs to identify true
positives and false positives. Using this information as a feedback on how well
current Tier-3 processing pipeline is doing, a small and lightweight JITL filter
is trained to distinguish true positives and false positives of EarlyDiscard
filters. These JITL filters are then pushed to the drone to run as a cascade
filter after the EarlyDiscard DNN.

True/false positive frames have high temporal locality throughout a drone
mission. The JITL filter is expected to pick up the features that confused the
EarlyDiscard DNN in the immediate past and improve the pipeline's accuracy in
the near future. These features are usually specific to the current flight, and
may be affected by terrain, shades, object colors, and particular shapes or
background textures.

JITL can be used with EarlyDiscard DNNs of different cutoff probabilities to
strike different trade-offs. In a bandwidth-favored setting, JITL can work with
an aggressively selective EarlyDiscard DNN to further reduce wasted bandwidth. In
a recall-favored setting, JITL can be used with a lower-cutoff DNN to preserve
recall.

In our implementation, we use a linear support vector machine
(SVM)~\cite{Friedman2001} as the JITL filter. Linear SVM has several advantages:
1) short training time in the order of seconds; 2) fast inference; 3) only
requires a few training examples; 3) small in size to transmit, usually on the
order of 50KB in our experiments. The input features to the JITL SVM filter are
the image features extracted by the EarlyDiscard DNN filter. In our case, since
we are using MobileNet as our EarlyDiscard filter, they are the 1024-dimensional
vector elements from the second last layer of MobileNet. This vector, also
called ``bottleneck values'' or ``transfer values'' captures high-level features
that represents the content of an image. Note that the availability of such
image feature vector is not tied to a particular image classification DNN nor
unique to MobileNet. Most image classification DNNs can be used as a feature
extractor in this way.

\subsection{JITL Experimental Setup}
We used Jetson TX2 as our Tier-3 device platform and evaluated the JITL strategy
on four tasks, T1 to T4. For the test videos in each task, we began with the
EarlyDiscard filter alone and gradually trained and deployed JITL filters.
Specifically, every ten seconds, we trained an SVM using the frames transmitted
from the drone and the ground-truth labels for these frames. In a real
deployment, the frames would be marked as true positives or false positives by
an accurate DNN running on the cloudlet since ground-truth labels are not
available. In our experiments, we used ground-truth labels to control variables
and remove the effect of imperfect prediction of DNN models running on the
cloudlet. 

In addition, we used the true and false positives from all previous intervals,
not just the last ten seconds when training the SVM. The SVM, once trained, is
used as a cascade filter running after the EarlyDiscard filter on the drone to
predict whether the output of the EarlyDiscard filter is correct or not. For a
frame, if the EarlyDiscard filter predicts it to be interesting, but the JITL
filter predicts the EarlyDiscard filter is wrong, it would not be transmitted to
the cloudlet. In other words, following two criteria need to be satisfied for a
frame to be transmitted to the cloudlet: 1) EarlyDiscard filter predicts it to
be interesting 2) JITL filter predicts the EarlyDiscard filter is correct on
this frame.

\subsection{JITL Results}

From our experiments, JITL is able to filter out more than 15\% of remaining
frames after EarlyDiscard without loss of event recall for three of four tasks.
Figure~\ref{fig:jitl-eventrecall} details the fraction of frames saved by JITL.
The x-axis presents event recall. Y-axis represents the fraction of total
frames. The blue region presents the achievable fraction of frames by
EarlyDiscard. The orange region shows the additional savings using JITL. For T1,
T3, and T4, at the highest event recall, JITL filters out more than 15\% of
remaining frames. This shows that JITL is effective at reducing the false
positives thus improving the precision of the drone filter. However,
occasionally, JITL predicts wrongly and removes true positives. For example, for
T2, JITL does not achieve a perfect event recall. This is due to shorter event
duration in T2, which results in fewer positive training examples to learn
from. Depending on tasks, getting enough positive training examples for JITL
could be difficult, especially when events are short or occurrences are few. To
overcome this problem in practice, techniques such as synthetic data
generation~\cite{Dwibedi2017} could be explored to synthesize true positives
from the background of the current flight.

\section{Evaluation}
\section{Discussion}