\section{T\lc{echniques} \lc{for} W\lc{orkload} R\lc{eduction}}

\label{sec: workload-reduction}

In this section, we focus on techniques to reduce cloudlet workload. We first
discuss the intuition and mechanisms of the techniques and then present
micro-benchmarks applying these techniques to example applications.

\subsection{Adaptive Sampling}


The processing demands and latency bounds of a WCA application can
vary considerably during task execution because of human speed
limitations.  When the user is awaiting guidance, it is desirable to
sample input at the highest rate to rapidly determine task state and
thus minimize guidance latency.  However, while the user is performing
a task step, the application can stay in a passive state and sample at
a lower rate.  For a short period of time immediately after guidance
is given, the sampling rate can be very low because it is not humanly
possible to be done with the step.  As more time elapses, the
sampling rate has to increase because the user may be nearing
completion of the step.  Although this active-passive phase
distinction is most characteristic of WCA applications that provide
step-by-step task guidance (the blue cluster in
the lower right of Figure~\ref{figs:design-space}), most WCA
applications exhibit this behavior to some degree.  As shown in the
rest of this section, adaptive sampling rates can reduce processing
load without impacting application latency or accuracy.

We use task-specific heuristics to define application active and
passive phases.  In an active application phase, a user is likely to
be waiting for instructions or comes close to needing instructions,
therefore application needs to be ``active`` by sampling and
processing at high frequencies. On the other hand, applications can
run at low frequency during passive phases when an instruction is
unlikely to occur.

We use the LEGO application to show the effectiveness of adaptive
sampling. By default, the LEGO application runs at active phase. The
application enters passive phases immediately following the delivery
of an instruction, since the user is going to take a few seconds
searching and assembling LEGO blocks. The length and sampling rate of
a passive phase is provided by the application to the framework. We
provide the following system model as an example of what can be
provided. We collect five LEGO traces with 13739 frames as our
evaluation dataset.

\textbf{Length of a Passive Phase: }
We model the time it takes to finish each step as a Gaussian distribution. We
use maximum likelihood estimation to calculate the parameters of the Guassian
model.

\begin{figure}[]
\centering
\includegraphics[width=0.95\linewidth,trim=2em 3em 28em 20em, clip]{FIGS/fig-lego-sampling-model.pdf}
\caption{\small Dynamic Sampling Rate for LEGO}
\label{fig:lego-sampling-model}
\end{figure}

\textbf{Lowest Sampling Rate in Passive Phase: }
The lowest sampling rate in passive phase still needs to meet application's
latency requirement. Figure~\ref{fig:lego-sampling-model} shows the system model
to calculate the largest sampling period S that still meets the latency bound.
In particular,
$$(k-1)S + processing\_delay \leq latency\_bound $$ $k$ represents the
cumulative number of frames an event needs to be detected in order to be
certain an event actually occurred. The LEGO application empirically sets this
value to be 5. 

\textbf{Adaptation Algorithm: }
At the start of a passive phase, we set the sampling rate to the
minimum calculated above.  As time progresses, we gradually increase
the sampling rate.  The idea behind this is that the initial low
sampling rates do not provide good latency, but this is acceptable, as
the likelihood of an event is low.  As the likelihood increases (based
on the Gaussian distribution described earlier), we increase sampling
rate to decrease latency when events are likely.
Figure~\ref{fig:adaptive-sampling-example}(a) shows the sampling rate
adaptation our system employs during a passive phase.
The sampling rate is calculated as $$sr = \\
%
 min\_sr + \alpha * (max\_sr - min\_sr) * cdf\_Gaussian(t)$$ 
%
$sr$ is the sampling rate. $t$ is the time after an instruction has been given. $\alpha$ is
a recovery factor which determines how quickly the sampling rate
rebounds to active phase rate. 
 
\begin{figure}
\small\centering
\begin{subfigure}{.45\linewidth}
  \centering
  \includegraphics[width=\linewidth, trim=0em 0em 0em 0em, clip]{FIGS/fig-lego-adaptive-sr.pdf}
  {\small (a) Passive Sampling Rate}
\end{subfigure}
\begin{subfigure}{.45\linewidth}
    \centering
    \includegraphics[width=0.92\linewidth, trim=0em 0em 0em 0em, clip]{FIGS/fig-lego-example-sr.pdf}
   {\small (b) Trace Sampling Rate}
\end{subfigure}
\caption{\small Adaptive Sampling Rate}
\label{fig:adaptive-sampling-example}
\end{figure}


Figure~\ref{fig:adaptive-sampling-example}(b) shows the sampling rate
for a trace as the application runs. The video captures a user doing 7
steps of a LEGO assembly task. Each drop in sampling rate happens
after an instruction has been delivered to the user.
Table~\ref{tab:adaptive-sample-eval} shows the percentage of frames
sampled and guidance latency comparing adaptive sampling with naive
sampling at half frequency. Our adaptive sampling scheme requires
processing fewer frames while achieving a lower guidance latency.

\begin{table}[]
\small\centering
\begin{tabular}{|c|c|c|}
\hline
Trace  & \begin{tabular}[c]{@{}c@{}}Sample\\ Half Freq\end{tabular} & \begin{tabular}[c]{@{}c@{}}Adaptive\\ Sampling\end{tabular} \\ \hline
1 & 50\%          & 25\%              \\ \hline
2 & 50\%          & 28\%              \\ \hline
3 & 50\%          & 30\%              \\ \hline
4 & 50\%          & 30\%              \\ \hline
5 & 50\%          & 43\%              \\ \hline
\end{tabular}\\[0.1in]
{\small (a) Percentage of Frames Sampled}\\[0.2in]

\begin{tabular}{|c|c|}
\hline
                                                            & \begin{tabular}[c]{@{}c@{}}Guidance Delay \\ (frames$\pm$stddev)\end{tabular}
                                                            \\ \hline
\begin{tabular}[c]{@{}c@{}}Sample Half Freq\end{tabular}      & 7.6 $\pm$ 6.9                                                                  \\ \hline
\begin{tabular}[c]{@{}c@{}}Adaptive Sampling\end{tabular} &  5.9 $\pm$ 8.2                                                                  \\ \hline
% \begin{tabular}[c]{@{}c@{}}Theoretical Minimum\end{tabular} &  ?? $$                                                                  \\ \hline
\end{tabular}\\[0.1in]
{\small (b) Guidance Latency}\\[0.1in]
\caption{\small Frames Sampled and Guidance Latency}
\label{tab:adaptive-sample-eval}
\vspace{-0.2in}
\end{table}


\begin{figure}
\begin{center}
\includegraphics[width=1.0\linewidth]{FIGS/fig-imu-trace-lego.pdf}\\
{\small (a) LEGO}
\includegraphics[width=1.0\linewidth]{FIGS/fig-imu-trace-pingpong.pdf}\\
{\small (b) PING PONG}
\end{center}
\vspace{-0.1in}
\caption{\small Accuracy of IMU-based Frame Suppression}
\label{fig:imu-trace-example}
\vspace{-0.1in}
\end{figure}


\begin{table}
\small\centering
\begin{tabular}{| l | c | c |}
   \hline
        & Suppressed  &  Max Delay of \\ 
        & Passive Frames (\%)   & State Change Detection \\ \hline
    Trace 1 & 17.9\%    & 0 \\
    Trace 2 & 49.9\%    & 0 \\
    Trace 3 & 27.1\%    & 0 \\
    Trace 4 & 37.0\%    & 0 \\
    Trace 5 & 34.1\%    & 0 \\
    \hline
\end{tabular}\\[0.1in]
{\small (a) LEGO}\\[0.1in]

\begin{tabular}{| l | c | c |}
    \hline
        & Suppressed        & Loss of  \\
        & Passive Frames (\%)    & Active Frames (\%) \\ \hline
    Trace 1 & 21.5\%    &   0.8\%   \\
    Trace 2 & 30.0\%    &   1.5\%   \\
    Trace 3 & 26.2\%    &   1.9\%   \\
    Trace 4 & 29.8\%    &   1.0\%   \\
    Trace 5 & 38.4\%    &   0.2\%   \\
    \hline
\end{tabular}\\[0.1in]
{\small (b) PING PONG}\\[0.1in]
\caption{\small Effectiveness of IMU-based Frame Suppression}
\label{tab:imu-result}
\end{table}



\subsection{IMU-based Passive Phase Suppression}

In many applications, passive phases can often be associated with the
user's head movement. We illustrate with two applications here. In
LEGO, during the passive phase, which begins after the user receives
the next instruction, a user typically turns away from the LEGO board
and starts searching for the next brick to use in a parts box. During
this period, the computer vision algorithm would detect no meaningful
task states if the frames are transmitted.  In PING PONG, an active
phase lasts throughout a rally.  Passive phases are in between actual
game play, when the user takes a drink, switches sides, or, most
commonly, tracks down and picks up a wayward ball from the floor.
These are associated with much large range of head movements than
during a rally when the player generally looks toward the opposing
player.  Again, the frames can be suppressed on the client to reduce
wireless transmission and load on the cloudlet.  In both scenarios,
significant body movement can be detected through Inertial Measurement
Unit (IMU) readings on the wearable device, and used to predict those
passive phases.

For each frame, we get a six-dimensional reading from the IMU:
rotation in three axes, and acceleration in three axes.  We train an
application-specific SVM to predict active/passive phases based on IMU
readings, and suppress predicted passive frames on the client.
Figure~\ref{fig:imu-trace-example}(a) and (b) show an example trace
from LEGO and PING PONG, respectively.  Human-labeled ground truth
indicating passive and active phases is shown in blue.  The red dots
indicate predictions of passive phase frames based on the IMU
readings; these frames are suppressed at the client and not
transmitted.  Note that in both traces, the suppressed frames also
form streaks. In other words, a number of frames in a row can be
suppressed. As a result, the saving we gain from IMU is orthogonal to
that from adaptive sampling.

Although the IMU approach does not capture all of the passive frames
(e.g., in LEGO, the user may hold his head steady while looking for
the next part), when a passive frame is predicted, this is likely
correct (i.e., high precision, moderate recall).  Thus, we expect
little impact on event detection accuracy or latency, as few if any
active phase frames are affected.  This is confirmed in
Table~\ref{tab:imu-result}, which summarizes results for five traces
from each application.  We are able to suppress up to 49.9\% of
passive frames for LEGO and up to 38.4\% of passive frames in case of
PING PONG on the client, while having minimal impact on application
quality --- incurring no delay in state change detection in LEGO, and
less than 2\% loss of active frames in PING PONG.


