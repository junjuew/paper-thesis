\section{R\lc{elated} W\lc{ork}}
\label{sec: resource-management-related}

Deadline-based scheduling algorithms and mechanisms have been studied
extensively in the literature. Many work~\cite{tokuda1990real, kao1993deadline,
stankovic2012deadline, steiger2004operating} exist in the context of real-time
systems. Some use utility functions for scheduling tasks with soft
deadlines~\cite{ravindran2005recent, li2004utility}. While we draw
inspiration from these systems, our resource allocation focuses on application
adaptation characteristics in addition to meeting latency requirements.

Many mobile systems leverage application adaptation. Notably,
Odyssey~\cite{Noble1997} and extensions~\cite{Flinn1999} proposed upcall-based
collaboration between a mobile's operating system and its applications to adapt
to variable wireless connectivity and limited battery. Such adaption was purely
reactive by the mobile device; in our context, adaptation for a collection of
devices can be centrally managed by their cloudlet, with failover to reactive
methods as needed. Exploration of tradeoffs between application fidelity and
resource demand led to the concept of {\em multi-fidelity
applications}~\cite{Satya1999}; such concepts are relevant to our work, but the
critical computing resources in our setting are those of the cloudlet rather
than the mobile device.

More recently,
~\cite{boutin2014apollo,yao2014haste,verma2015large,ousterhout2013sparroW,delimitrou2014quasar,schwarzkopf2013omega}
study cluster scheduling in data-center context. These systems target a diverse
range of workload, take resource reservation demands, and focus on the
scalability of scheduling. In contrast, our resource allocation scheme focuses
on edge-native applications, in particular, wearable cognitive assistance, and
profile their characteristics for better outcomes in oversubscribed edge
scenarios.

Closely related, dynamic resource management in the cloud for video analytics
have been explored by~\cite{sembiring2013dynamic, fu2015drs, kaseb2015cloud}.
Some also leverage profile-based adaptation for more efficient video analytics
resource management~\cite{zhang2017live, hung2018videoedge, jiang2018chameleon}.
However, most of these systems focus on throughput-oriented analytics
application on large clusters. In contrast, we target interactive performance on
relatively small edge deployments.


