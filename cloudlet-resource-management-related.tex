\section{R\lc{elated} W\lc{ork}}
\label{sec: resource-management-related}

Although edge computing is new, the techniques for scalability
examined in this paper bear some resemblance to work that was done in
the early days of mobile computing, and more recent cloud management
work.

Odyssey~\cite{Noble1997} and extensions~\cite{Flinn1999} proposed upcall-based
collaboration between a mobile's operating system and its applications to adapt
to variable wireless connectivity and limited battery. Such adaption was purely
reactive by the mobile device; in our context, adaptation for a collection of
devices can be centrally managed by their cloudlet, with failover to reactive
methods as needed. Exploration of tradeoffs between application fidelity and
resource demand led to the concept of {\em multi-fidelity
applications}~\cite{Satya1999}; such concepts are relevant to our work, but the
critical computing resources in our setting are those of the cloudlet rather
than the mobile device.

Several different approaches to adapting application fidelity have been studied.
Dynamic sampling rate with various heuristics for adaptation have been tried
primarily in the contexts of individual mobile devices for energy
efficiency~\cite{lorincz2009mercury, lorincz2008resource, vallina2012energy,
lane2010survey}. Semantic deduplication to reduce redundant processing of frames
have been suggested by ~\cite{Hu2015, kang2017noscope, hsieh2018focus,
zhang2015design}. Similarly, previous works have looked at suppression based on
motion either from video content~\cite{naderiparizi2017glimpse,
lebeckcollaborative} or IMUs~\cite{jain2015overlay}. Others have investigated
exploiting multiple deep models with accuracy and resource
tradeoff~\cite{han2016mcdnn,jiang2018chameleon}. While most of these efforts
were in mobile-only, cloud-only, or mobile-cloud context, we explore similar
techniques in an edge-native context.


Partitioning workloads between mobile devices and the cloud have been
studied in sensor networks~\cite{newton2009wishbone},
throughput-oriented systems~\cite{cuervo2010maui, yi2017lavea}, for
interactive applications~\cite{ra2011odessa, chen2015glimpse}, and
from programming model perspective~\cite{balan2003tactics}. We believe
that these approaches will become important techniques to scale
applications on heavily loaded cloudlets.

Dynamic resource allocation schemes on the cloud for video processing have been
well explored~\cite{sembiring2013dynamic, fu2015drs, kaseb2015cloud}.  More
recently, profile-based adaptation of video analytics~\cite{zhang2017live,
hung2018videoedge, jiang2018chameleon} focused on throughput-oriented analytics
application on large clusters or cloud. In contrast, our goals focus on
interactive performance on relatively small edge deployments.


