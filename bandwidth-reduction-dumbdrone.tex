\section{Baseline: Dumb  S\lc{trategy}}
\label{sec:dumbdrone}

\subsection{Description}

We first establish and evaluate the baseline case of no image processing
performed at the Tier-3 device.  Instead, all captured video is immediately
transmitted to the cloudlet.  Result latency is very low, merely the sum of
transmission delay and cloudlet processing delay. We use drones as the example
of Tier-3 devices and drone video search as the scenario of video analytics
first. We later demonstrate how to apply the techniques developed to WCAs.

\begin{table}
\centering
\begin{tabular}{|p{1cm}|p{2.5cm}|p{2.5cm}|p{2.5cm}|p{2.5cm}|p{2.5cm}|}
\hline
   & Detection & Data & Data & Training & Testing \\ 
Task& Goal & Source & Attributes & Subset & Subset\\ 
\hline
T1 & {\small People in scenes of daily life}&{\small Okutama Action Dataset~\cite{Barekatain2017}}&\makecell[tl]{\small 33 videos \\\small 59842 fr\\\small 4K@30~fps}&\makecell[tl]{\small 9 videos\\\small 17763 fr}&\makecell[tl]{\small 6 videos\\\small 20751 fr}\\ 
\hline
T2 &{\small Moving cars}&{\small Stanford Drone Dataset~\cite{Robicquet2016}}&\makecell[tl]{\small 60 videos \\\small 522497 fr\\\small 1080p@30~fps}&\makecell[tl]{\small 16 videos\\\small 179992 fr} & \multirow{4}{*}{\parbox{1.5cm}{\centering\small 14 videos 92378 fr\\ Combination of test videos from each dataset.}} \\ \cline{1-5}
  %% \makecell[tl]{\small 14 videos\\\small 92378 fr\\\small Combination of \\\small human, car, raft \\\small and elephant videos\\\small from each datasets}} \\ \cline{1-5}
%% \makecell[tl]{\small 14 videos\\\small 92378 fr}\\ 
%% \hline
T3 &{\small Raft in flooding scene}&{\small YouTube collection~\cite{YouTube1}}&\makecell[tl]{\small 11 videos \\\small 54395 fr\\\small 720p@25~fps}&\makecell[tl]{\small 8 videos\\\small 43017 fr} & \\ \cline{1-5}
%% \makecell[tl]{\small 14 videos\\\small 92378 fr}\\
%% \hline
T4 &{\small Elephants in natural habitat}&{\small YouTube collection~\cite{YouTube2}}&\makecell[tl]{\small 11 videos \\\small 54203 fr\\\small 720p@25~fps}&\makecell[tl]{\small 8 videos\\\small 39466 fr} & \\ \cline{1-5}
%% \makecell[tl]{\small 14 videos\\\small 92378 fr}\\
%% \hline
% T5 &{\small Pushing or pulling Suitcases}&\small Okutama Action Dataset &\small Same as T1 &\small Same as T1 & \\
%% \small Same as T1\\
\hline
\end{tabular}
\vspace{0.1in}
\begin{captiontext}
fr = ``frames''\\
fps = ``frames per second''\\
No overlap between training and testing subsets of data
\end{captiontext}
\caption{Benchmark Suite of Drone Video Traces}
\label{fig:benchmarksuite}
\end{table}

\begin{table}
\centering
\begin{tabular}{|c|c|c|c|c|}
\hline
     & Total & Avg & & \\ 
     & Bytes & BW & & \\ 
Task & (MB) & (Mbps) & Recall & Precision \\ 

\hline
T1 & \phantom{0}924 & 10.7 & 74\% & 92\%\\ 
\hline
T2 & 2704 & \phantom{0}7.0 & 66\% & 90\%\\ 
\hline
\end{tabular}
\vspace{0.2in}
\begin{captiontext}
Peak bandwidth demand is same as average since video is transmitted
continuously. Precision and recall are at the maximum F1 score.
\end{captiontext}
\caption{Baseline Object Detection Metrics}
\label{fig:baseline}
\end{table}

\subsection{Experimental Setup}
\label{sec:dumbdrone-setup}

To ensure experimental reproducibility, our evaluation is based on
replay of a benchmark suite of pre-captured videos rather than on
measurements from live drone flights.  In practice, live results may
diverge slightly from trace replay because of non-reproducible
phenomena.  These can arise, for example, from wireless propagation
effects caused by varying weather conditions, or by seasonal changes
in the environment such as the presence or absence of leaves on trees.
In addition, variability can arise in a  drone's pre-programmed flight
path due to collision avoidance with moving obstacles such as birds,
other drones, or aircraft.

All of the pre-captured videos in the benchmark suite are publicly accessible,
and have been captured from aerial viewpoints. They characterize drone-relevant
scenarios such as surveillance, search-and-rescue, and wildlife conservation.
Table~\ref{fig:benchmarksuite} presents this benchmark suite of videos,
organized into four tasks. All the tasks involve detection of tiny objects on
individual frames. Although T2 is also nominally about action detection (moving
cars), it is implemented using object detection on individual frames and then
comparing the pixel coordinates of vehicles in successive frames.
% Task T5 additionally involves action detection, which operates on short video segments rather than individual frames. 

\subsection{Evaluation}
\label{sec:dumbdrone-results}

Table~\ref{fig:baseline} presents the key performance indicators on the object
detection tasks T1 and T2. We use the well-labeled dataset to train and evaluate
Faster-RCNN with ResNet 101. We report the precision and recall at maximum F1
score.  Peak bandwidth is not shown since it is identical to average bandwidth
demand for continuous video transmission.  As shown earlier in
Table~\ref{fig:onboard-dnn-speed}, the accuracy of this algorithm comes at the
price of very high resource demand.  This can only be met today by server-class
hardware that is available in a cloudlet.  Even on a cloudlet, the figure of 438
milliseconds of processing time per frame indicates that only a rate of two
frames per second is achievable.  Sustaining a higher frame rate will require
striping the frames across cloudlet resources, thereby increasing resource
demand considerably.  Note that the results in
Table~\ref{fig:onboard-dnn-speed} were based on 1080p frames, while tasks T1
uses the higher resolution of 4K. This will further increase demand on cloudlet
resources.

Clearly, the strategy of blindly shipping all video to the cloudlet and
processing every frame is resource-intensive to the point of being impractical
today.  It may be acceptable as an offline processing approach in the cloud, but
is unrealistic for real-time processing on cloudlets.  We therefore explore an
approach in which a modest amount of computation on the Tier-3 is able, with
high confidence, to avoid transmitting many video frames and thereby saving
wireless bandwidth as well as cloudlet processing resources.  This leads us to
the EarlyDiscard strategy of the next section.



