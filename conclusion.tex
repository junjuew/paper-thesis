\chapter{Conclusion and Future Work}
\label{chapter: conclusion}

This dissertation addresses the problem of \textit{scaling wearable cognitive
assistance} for widespread deployment. We propose system optimizations that
reduce network consumption, leverage application characteristics to adapt client
behaviors, and provide an adaptation-centric resource management mechanism. In
addition, we design and develop a suite of development tools that lower the
barrier of entry and improve developer productivity. This chapter concludes the
dissertation with a summary of contributions, and discusses future research
directions and challenges in this area.

\section{Contributions}

This dissertation claims the following thesis.

\noindent\fbox{%
    \parbox{\textwidth}{%
\textbf{Two critical challenges to the widespread adoption of wearable cognitive
  assistance are 1) the need to operate cloudlets and wireless network at low
  utilization to achieve acceptable end-to-end latency 2) the level of specialized
  skills and the long development time needed to create new applications. These
  challenges can be effectively addressed through system optimizations,
  functional extensions, and the addition of new software development tools to
  the Gabriel platform.}
    }%
}

To validate this thesis, we first introduce two example wearable cognitive
assistants and present measurement studies on how they would saturate existing
wireless network bandwidth. We propose two application agnostic techniques to
reduce network transmission. Then, leveraging WCA application characteristics,
we provide an adaptation taxonomy and demonstrate techniques to reduce offered
load on the client device. With these adaptation mechanism, we design and
evaluate a adaptation-centric resource allocation mechanism at the edge server
that takes advantages of application degradation profiles. In the end, we offer
a new application development methodology and provide a suite of tools to reduce
development difficulty and speed up application development process.

% \subsection{Application Agnostic Methods to Reduce Network Transmission}

% Wearable cognitive assistance, as a latency-sensitive analytics applications
% poses several difficult mobile computing challenges that arise in performing
% real-time video analytics on small wearable devices. These challenges lie at the
% intersection of wireless bandwidth, result accuracy, and timeliness of results.

% To address these challenges, we have developed an adaptive computer vision
% pipeline for reducing network transmission without application assistance.  We
% explore an early discard strategy to selectively send the most interesting
% frames and reduce precious bandwidth between the drone and a ground-based
% cloudlet. We propose just-in-time learning to further improve bandwidth
% efficiency. Our experimental results show that this judicious combination of
% client-side processing and edge-based processing can save substantial wireless
% bandwidth and thus improve scalability, without compromising result accuracy or
% result latency. 

% \subsection{Application-Aware Techniques to Reduce Offered 
% Load and Adaptation Centric Cloudlet Resource Management}

% More than a decade ago, the emergence of cloud computing led to the
% realization that applications had to be written in a certain way to
% take full advantage of elasticity of the cloud.  This led to the
% concept of ``cloud-native applications'' whose scale-out capabilities
% are well matched to the cloud, as well as tools and techniques to
% easily create such applications.

% The emergence of edge computing leads to another inflection point in
% application design.  In particular, it leads to ``edge-native
% applications'' that are deeply dependent on attributes such as low
% latency or bandwidth scalability that can only be obtained at the edge.
% However, as this paper has shown, edge-native applications have to be 
% written in a way that is very different from cloud-native applications
% if they are to be scalable.

% This is the first work to show that cloud-native implementation
% strategies that focus primarily on dynamic scale-out are unlikely to
% be effective for scalability in edge computing.  Instead, edge-native
% applications need to adapt their network and cloudlet resource demand
% to system load.  As the total number of Tier-3 devices associated with
% a cloudlet increases, the per-device network and cloudlet load has to
% decrease.  This is a fundamental difference between cloud-native and
% edge-native approaches to scalability. 

% We explore client workload reduction and server resource allocation to manage
% application quality of service in the face of contention for cloudlet resources.
% We demonstrate that our system is able to ensure that in overloaded situations,
% a subset of users are still served with good quality of service rather than
% equally sharing resources and missing latency requirements for all.

% \subsection{Wearable Cognitive Assistance Development Tools}

% Wearable cognitive assistance used to be difficult to develop. We propose a
% unifying development methodology to streamline the development process. Together
% with the methodology, we have built and provided a suite of development tools
% that lower the barrier of application development and speed up the
% implementation process. With these tools, we have shown the productivity
% improvement can be 10x-20x for application developers.

\section{Future Work}

\subsection{Advanced Computer Vision For Wearable Cognitive Assistance}

Most WCAs developed and discussed in this dissertation are frame-based. Namely,
they assume that complete user states can be captured within a single frame.
While symbolic states can be aggregated, computer vision advancement in video
analysis (e.g. activity recognition), can significantly broaden the horizon and
improve the robustness of WCAs. For instance, in many assembly tasks, screws
become occluded when placed correctly. Current solutions, as a workaround, often
consider the screw in-place as a separate object. This is not robust to wrong
user actions (e.g. clockwise tightening vs counter-clockwise tightening).
Activity recognition that remembers a history of objects and human actions can
help solve the problem and thus enable many new WCA domains.

\subsection{Fine-grained Online Resource Management}

This dissertation opens up many potential directions to explore in practical
resource management for edge-native applications. We have alluded to some of
these topics earlier. 

One example we briefly mentioned is dynamic partitioning of work between Tier-3
and Tier-2 to further reduce offered load on cloudlets.  In addition, other
resource allocation policies, especially fairness-centered policies, such as
max-min fairness and static priority can be explored when optimizing overall
system performance. These fairness-focused policies could also be used to
address aggressive users, which are not considered in this dissertation.  While
we have shown offline profiling is effective for predicting demand and utility
for WCA applications, for a broader range of edge-native applications, with ever
more aggressive and variable offload management, online estimation may prove to
be necessary. 

Another area worth exploring is the particular set of control and coordination
mechanisms to allow cloudlets to manage client-offered load directly. Finally,
the implementation to date only controls allocation of resources but allows the
cloudlet operating system to arbitrarily schedule application processes. Whether
fine-grained control of application scheduling on cloudlets can help scale
services remains an open question.


\subsection{WCA Synthesis from Example Videos}

The authoring tools presented in this dissertation can be considered as a first
step towards an ambitious goal of synthesizing cognitive assistants
automatically from crowd-sourced expert videos. The critical missing piece is
the ability to analyze and summarize a consistent task model from multiple
videos. Some work~\cite{pham2018unsupervised} has started to study this
challenge, although there is still a long road ahead. Much needed is signficant
improvement of domain-specific video understanding. Nonetheless, many tools and
techniques discussed in this dissertation could still apply and serve as
stepping stones toward fully automated creation of WCAs. 