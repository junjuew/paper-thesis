\section{Profiling-based Resource Allocation}
\label{sec: resource-allocation}

Given a workload of concurrent applications running on a cloudlet, and the
number of clients requesting service from each application, our resource
allocator determines how many instances to launch and how much resource (CPU
cores, memory, etc.) to allocate for each application instance.  We assume
queueing delays are limited by the token mechanism used in Original Gabriel,
which limits the number of outstanding requests on a per-client basis.


\subsection{Maximizing Overall System Utility}

As described earlier, for each application $a \in $ \{FACE, LEGO, PING PONG, POOL, \ldots \},
we construct a resource to utility mapping
$u_a: \mathbf{r} \rightarrow \mathbb{R}$ for one instance of the application on cloudlet,
where $\mathbf{r}$ is a resource vector of allocated CPU, memory, etc. We formulate the
following optimization problem which maximizes the system-wide total utility,
subject to a tunable maximum per-client limit:

\begin{equation}
  \begin{aligned}
    \max_{\{k_a, \mathbf{r}_a\}} \quad & \sum_a{k_a \cdot u_a(\mathbf{r}_a)}                              \\
    \textrm{s.t.} \quad                & \sum_a k_a \cdot \mathbf{r}_a \preccurlyeq \hat{\mathbf{r}}      \\
                                       & 0 \preccurlyeq \mathbf{r}_a  \quad \forall a                     \\
                                       & k_a \cdot u_a(\mathbf{r}_a) \le \gamma \cdot c_a \quad \forall a \\
                                       & k_a \in \mathbb{Z}
  \end{aligned}
\end{equation}

In above, $c_a$ is the number of mobile clients requesting service from application $a$.
The total resource vector of the cloudlet is  $\hat{\mathbf{r}}$.
For each application $a$, we determine how many instances to launch --- $k_a$, and
allocate resource vector $\mathbf{r}_a$ to each of them.
A tunable knob $\gamma$ regulates the maximum utility allotted
per application, and serves to enforce a form of partial fairness (no application
can be given excessive utility, though some may still receive none).
The larger $\gamma$ is, the more aggressive our scheduling algorithm
will be in maximizing global utility and
suppressing low-utility applications.
%In our system model, the maximum $\gamma$ is 30 for
%clients at 30 FPS. 
By default, we set $\gamma=10$, which, based on our definition of
utility, roughly means resources will be allocated so
no more than one third of frames (from a 30FPS source)
will be processed within satisfactory latency bounds for a given
client.

Solving the above optimization problem is computationally difficult. We thus use
an iterative greedy allocation algorithm as shown in
Figure~\ref{fig:cloudlet-allocation-algorithm}. In our implementation, we
exploit the \texttt{cpu-shares} and \texttt{memory-reservation} control options
of Linux Docker containers. It puts a soft limit on containers' resource
utilization only when they are in contention, but allows them to use as much
left-over resource as needed.

\begin{algorithm}[]
  \SetAlgoLined
  % \KwResult{Write here the result }
  Profile applications under varying resources\;
  $u_a(\mathbf{r})$: resource $\mathbf{r}$ to utility mapping for application
  $a$\;
  \For{each application}{
    find the highest \emph{utility-to-resource} ratio $\frac{u_a(\mathbf{r})}{|\mathbf{r}|}$\;
  }
  \While{leftover system resource}{
    Find the application with the largest \emph{utility-to-resource}
    $\frac{u_a(\mathbf{r}^*_a)}{|\mathbf{r}^*_a|}$, which has not been allocated
    resources\;
    Allocate $k_a$ application instances,
    each with resource $\mathbf{r}^*_a$, such that $k_a$ is the largest integer with
    $k_a \cdot u_a(\mathbf{r}^*_a) \le \gamma \cdot c_a$\;
  }
  \caption{Iterative Allocation Algorithm to Maximize Overall System Utility}
  \label{fig:cloudlet-allocation-algorithm}
\end{algorithm}

% For each application profile $u_a(\mathbf{r})$, we find the resource point that gives
% the highest $\frac{u_a(\mathbf{r})}{|\mathbf{r}|}$, i.e., \emph{utility-to-resource} ratio. 
% Denote this point as $\mathbf{r}^*_a$. We start with the application with the largest 
% $\frac{u_a(\mathbf{r}^*_a)}{|\mathbf{r}^*_a|}$. We allocate $k_a$ application instances,
% each with resource $\mathbf{r}^*_a$, such that $k_a$ is the largest integer with
% $k_a \cdot u_a(\mathbf{r}^*_a) \le \gamma \cdot c_a$. 
% If there is leftover resource, we move to the application with the next highest 
% utility-to-resource ratio and repeat the process.
